
\chapter{sheet-21 : Variational Method}
%% writing by shahnoor
\ifpdf
\graphicspath{{Chapter21/figs/}}
\else
\graphicspath{{Chapter21/figs/}}
\fi

\section{Principles of the Method}
The expectation value of the Hamiltonian $H$ of a system in any state $\ket{\psi}$ is given by
\begin{equation}
\expval{H} = \frac{\expval{H}{\psi}}{\braket{\psi}{\psi}}
\label{chapter21.eqn1-principle}
\end{equation}
If the state $\ket{\psi}$ is normalized, we can write
\begin{equation}
\expval{H} = \expval{H}{\psi} \equiv \qty(\psi, H \psi)
\label{chapter21.eqn2-principle}
\end{equation}

We shall now prove that $\expval{H}$ is an upperbound to the ground state energy $E_0$ of the system, i.e.,
\begin{equation}
\expval{H} \geq E_0
\label{chapter21.eqn3-principle}
\end{equation}

\underline{Proof}\\
	We expand $\ket{\psi}$ as a linear combination of the complete set of states $\{\ket{\phi_i}, i=0,1,2,\ldots\}$ where the $\ket{\phi}$'s are the orthogonal eigenstates of $H$ belonging to the eigenvalue $E_0, E_1, \ldots$ respectively. Thus
	\begin{equation}
		\ket{\psi} = \sum_{i=0}^{\infty} a_i \ket{\phi_i}
		\label{chapter21.eqn4-principle}
	\end{equation}
	Since $\ket{\psi}$ is normalized, i.e., $\braket{\psi}{\psi} = 1$, it follows that
	\begin{equation}
		\sum_{i=0}^{\infty} \abs{a_i}^2 = 1
		\label{chapter21.eqn5-principle}
	\end{equation}
	
	%% page 2
	The expectation value of $H$ in the state $\ket{\psi}$ can now be written as 
	\begin{align*}
		\expval{H} &= \expval{H}{\psi} = \sum_{i,j} a^*_i a_j \mel{\phi_i}{H}{\phi_j}\\
		&=  \sum_{i,j} a^*_i a_j E_j \braket{\phi_i}{\phi_j}
	\end{align*}
	Since the eigenstates $\ket{\phi_i},\ i=1,2,\ldots$ are orthogonal, we have
	\begin{equation*}
		\braket{\phi_i}{\phi_j} = \delta_{i j}
	\end{equation*}
	Hence
	\begin{align*}
		\expval{H} &= \sum_{i,j} a^*_i a_j E_j \delta_{i j} = \sum_{i} \abs{a_i}^2 E_i \\
		&\geq E_0 \sum_{i} \abs{a_i}^2 \quad \text{since } E_i > E_0 \ \text{for} \ i>0 \\
		&= E_0 \quad \text{since } \sum_{i} \abs{a_i}^2=1 \  (\text{equation} \ref{chapter21.eqn5-principle})\\
		Thus, \ \expval{H} &\geq E_0
	\end{align*}
	The equality sign holds if $\ket{\psi}$ is exactly equal to the ground state vector $\ket{\phi}$, in which case $a_0=1$ and all other $a_i$'s are zero. Thus we have shown that the expectation value of $H$ in any state $\ket{\psi}$ gives an upper bound to the ground state energy. This result is the basis of the variational method for finding the ground state energy and the wavefunction.
	
	The inequality $E_0 \leq \expval{H}$ shows that if we choose a number of trial wavefunctions $\psi_1, \psi_2, \ldots$ and calculate the corresponding expectation values $\expval{H}_i$, then each of the expectation values is greater that $E_0$. Therefore, the lowest expectation value is closest to $E_0$. In this variational method we  proceed as follows:
	\begin{enumerate}
		\item 	Choose an appropriate trial wave function $\psi_{\alpha \beta \ldots}$ depending on the parameters $\alpha, \beta, \ldots$.
		
		
		\item Calculate the expectation value $\expval{H}_{\alpha, \beta, \ldots}$ using the wavefunction $\psi_{\alpha, \beta, \ldots}$.
		
		
		\item Vary the trial wavefunctions by varying the parameters $\alpha, \beta, \ldots$ such that $\expval{H}_{\alpha, \beta, \ldots}$ attain its minimum value. To find the values of the parameters for which the expectation value is minimum, we set
		\begin{align*}
			\pdv{\expval{H}_{\alpha, \beta, \ldots}}{\alpha} = 0\\
			\pdv{\expval{H}_{\alpha, \beta, \ldots}}{\beta} = 0
		\end{align*}
		and so on. Solving these equations we obtain $\alpha_0, \beta_0, \ldots$. Thus $\expval{H}_{\alpha_0, \beta_0, \ldots}$ is a minimum and so is the best approximation to the ground state energy. The wavefunction $\psi_{\alpha_0, \beta_0, \ldots}$ is the variational approximation to the ground state wavefunction. Usually, the approximation to the wavefunction is \textbf{poorer} than the approximation to the ground state energy.
	\end{enumerate}
	
	
	\section{Variational Method for Excited States}
	The variational method can also be adapted to obtain approximate values for the energy of an excited state provided that the wavefunctions of the states of lower energy are accurately known. The trial wave function of the $n^{th}$ state is taken to be orthogonal to the known states of lower energy. Thus, the trial wavefunction for the $n^{th}$ state is of the form
	\begin{align*}
		\ket{\psi} = \ket{\chi} - \sum_{i=0}^{n-1} \ket{\phi_i}\braket{\phi_i}{\chi}
	\end{align*}
	Where $\ket{\chi}$ is an arbitrary ket conforming to the general features of Quantum Mechanics. It is obvious that
	\begin{align*}
		\braket{\phi_0}{\psi}  = 		\braket{\phi_1}{\psi} = \ldots = 		\braket{\phi_{n-1}}{\psi} = 0
	\end{align*}
	Therefore, in the expansion of $\ket{\psi}$ in terms of the basis states $\ket{\phi_i}$, we will have $a_i=0$ for $i=0,1,\ldots,n-1$, i.e.,
	\begin{equation*}
		\ket{\psi} = \sum_{i=n}^{\infty} a_i \ket{\phi_i}
	\end{equation*}
	Normalizing $\ket{\psi}$, i.e., $\braket{\psi}{\psi} = 1$, we have $\sum_{i=n}^{\infty} \abs{a_i}^2 = 1$.
	Therefore
	\begin{align*}
		\expval{H} 
		&= \expval{H}{\psi} \\
		&= \sum_{i=n}^{\infty} \sum_{j=n}^{\infty} a^{*}_{i} a_{j} \mel{\phi_i}{H}{\phi_j} \\
		&= \sum_{i=n}^{\infty} \sum_{j=n}^{\infty} a^{*}_{i} a_{j} E_j \delta_{i j} \\
		&= \sum_{i=n}^{\infty} \abs{a_i}^2 E_i \\
		&\geq \sum_{i=n}^{\infty} \abs{a_i}^2 = E_n
	\end{align*}
	Thus
	\begin{equation}
		\expval{H} \geq E_n
	\end{equation}
	i.e., $\expval{H}$ provides an upper bound to the energy $E_n$.
	
	
	\section{Examples}
	%%% page 5
	\subsection{One-dimensional Harmonic Oscillator}
	We consider a one-dimensional harmonics oscillator whose Hamiltonian is 
	\begin{equation*}
		H = - \frac{\hbar^2}{2 m}\dv[2]{}{x} + \frac{1}{2} m \omega_0^2 x^2
	\end{equation*}
	The potential energy $V(x)=\frac{1}{2} m \omega_0^2 x^2$ is an even function of $x$. Therefore, eigenstates of $H$ must be either even or odd.
	
	\subsubsection{Ground State}
	
	 The lowest state in energy, i.e., the ground state is always even. Further, the wavefunction must tend to zero as $\abs{x} \rightarrow 0 $. These properties of the exact ground-state wavefunction suggests that we can choose the trail wavefunction to be of the form
	\begin{equation*}
		\psi\qty(x) = A e ^{-\alpha x^2 / 2}
	\end{equation*}
	Here $\psi$ depends on only one parameter $\alpha$. The constant $A$ is fixed by normalization of $\psi$, i.e.,
	\begin{align*}
		\braket{\psi}{\psi} &= 1 \\
		or, \ \int_{-\infty}^{\infty} \psi^*\qty(x) \psi\qty(x) \dd{x} &= 1\\
		or, \ \abs{A}^2 \int_{-\infty}^{\infty} e^{-\alpha x^2} \dd{x} &= 1\\
		or, \ \abs{A}^2 \qty(\frac{\pi}{\alpha})^{1/2} &= 1
	\end{align*}
	Therefore, we can choose $A$ to be real and positive having the value
	\begin{align*}
		A = \qty(\frac{\alpha}{\pi})^{1/4}
	\end{align*}
	Therefore, the normalized trial wavefunction is 
	\begin{equation*}
		\psi\qty(x) = \qty(\frac{\alpha}{\pi})^{1/4} e^{-\alpha x^2 / 2}
	\end{equation*}
	In the next step, we have to calculate the expectation value of $H$. We have
	\begin{equation*}
		\expval{H}_\alpha = \expval{H}{\psi} = \int_{-\infty}^{\infty} \psi^* \qty(x) H \psi \qty(x) \dd{x} = \expval{T}_\alpha + \expval{V}_\alpha
	\end{equation*}
	Where
	\begin{align*}
		\expval{T}_\alpha 
		&= \int_{-\infty}^{\infty} \psi^* \qty(x) \hat{T} \psi \qty(x) \dd{x} \\
		&= -\frac{\hbar^2}{2 m} \qty(\frac{\alpha}{\pi})^{1/2} \int_{-\infty}^{\infty} e^{-\alpha x^2/2} \dv[2]{}{x} e^{-\alpha x^2 / 2} \dd{x}
	\end{align*}
	
	and
	\begin{align*}
	\expval{V}_\alpha 
	&= \int_{-\infty}^{\infty} \psi^* \qty(x) V \psi \qty(x) \dd{x} \\
	&= \qty(\frac{\alpha}{\pi})^{1/2} \frac{1}{2} m \omega_0^2 \int_{-\infty}^{\infty} x^2 e^{-\alpha x^2} \dd{x} \\
	&= \qty(\frac{\alpha}{\pi})^{1/2} \frac{1}{2} m \omega_0^2 \frac{1}{2 \alpha} \qty(\frac{\pi}{\alpha})^{1/2} \\
	&= \frac{m \omega_0^2}{4 \alpha}
	\end{align*}
	Using the gamma integrals (appendix (\ref{appendix1.Integrals})).
	
	Next, we will calculate $\expval{T}_\alpha$. Integrating by parts we get
	\begin{align*}
		\expval{T}_{\alpha} =  \frac{\hbar^2}{2 m} \qty(\frac{\alpha}{\pi})^{1/2} \int_{-\infty}^{\infty} \qty[\dv{}{x} e^{-\alpha x^2 / 2}]^2 \dd{x} \\
		&=  \frac{\hbar^2}{2 m} \qty(\frac{\alpha}{\pi})^{1/2} \int_{-\infty}^{\infty} \qty[-\alpha x e^{-\alpha x^2 / 2}]^2 \dd{x} \\
		&= \frac{\hbar^2}{2 m} \qty(\frac{\alpha}{\pi})^{1/2} \alpha^2 \int_{-\infty}^{\infty} x^2 e^{-\alpha x^2} \dd{x} \\
		&= \frac{\hbar^2}{2 m} \qty(\frac{\alpha}{\pi})^{1/2} \alpha^2 \frac{1}{2 \alpha} \qty(\frac{\pi}{\alpha})^{1/2}
	\end{align*}
	Therefore
	\begin{align*}
		\expval{H}_\alpha &= \expval{T}_\alpha + \expval{V}_\alpha = \frac{\hbar^2 \alpha}{4 m} + \frac{m \omega_0^2}{4 \alpha}\\
		&= \frac{1}{4} \qty(\frac{\hbar^2 \alpha}{m}  +  \frac{m \omega_0^2}{\alpha})
	\end{align*}
	
	Next, we minimize $\expval{H}_\alpha$ by varying the parameter $\alpha$. To find the value of $\alpha$ for which $\expval{H}_\alpha$ is minimized, we write
	\begin{align*}
		\pdv{}{\alpha} \expval{H}_\alpha &= 0\\
		or, \ \frac{1}{4} \qty(\frac{\hbar^2}{m} - \frac{m \omega_0^2}{\alpha^2}) &= 0\\
		or, \ \alpha^2 = \frac{m^2 \omega_0^2}{\hbar^2} \\
		i.e., \alpha = \alpha_0 = \frac{m \omega_0}{\hbar}
	\end{align*}
	Thus, the minimum value of $\expval{H}_\alpha$ is 
	\begin{align*}
	 \expval{H}_{min}
	 &= \expval{H}_{\alpha_0} = \frac{1}{4} \qty(\frac{\alpha_0 \hbar^2}{m}   +  \frac{m \omega_0^2}{\alpha_0} ) \\
	 &= \frac{1}{4} \qty(\frac{m \omega_0 \hbar^2}{\hbar m}  +  \frac{m \omega_0^2 \hbar}{\omega_0})  \\
	 &= \frac{1}{2} \hbar \omega_0
	\end{align*}
	Therefore
	\begin{align*}
		E_0 &\leq \expval{H}_{\alpha_0}\\
or, \	E_0 &\leq \frac{1}{2} \hbar \omega_0
	\end{align*}
	The variational estimate of the ground state wavefunction is
	\begin{align*}
		\psi_{\alpha_0} \qty(x) = \qty(\frac{m \omega_0}{\pi \hbar}^{1/4}) e^{-m\omega_0 x^2/ \hbar}
	\end{align*}
	
	
	\subsubsection{First excited state}
	%% page 9
	
	The trial wavefunction has to be chosen such that it is orthogonal to the ground state wavefunction. Since the ground state wavefunction is even, the trial wave function mus be chosen as an \textbf{odd} function of $x$. In that case, the trial wavefunction will be orthogonal to the ground state wavefunction.
	
	Let the trial wavefunction be
	\begin{align}
		\psi\qty(x) = B x e^{-\beta x^2/2}
	\end{align}
	Where $B$ is the normalization constant. Normalizing $\psi\qty(x)$ we obtain
	\begin{align}
		\abs{B}^2 = \frac{2 \beta^{3/2}}{\sqrt{\pi}}
	\end{align}
	Hence the normalized trial wavefunction is
	\begin{align}
		\psi\qty(x) = \sqrt{\frac{2 \beta^{3/2}}{\sqrt{\pi}}} x e^{-\beta x^2/2}
	\end{align}
	Next, we have to calculate $\expval{H}$:
	\begin{align}
		\expval{H}_\beta = \expval{T}_\beta + \expval{V}_\beta
	\end{align}
	where
	\begin{align*}
		\expval{T}_\beta 
		&= - \frac{\hbar^2}{2 m} \int_{-\infty}^{\infty} \psi^*\qty(x) \dv[2]{}{x} \psi\qty(x) \dd{x}\\
		&= \frac{\hbar^2}{2 m} \int_{-\infty}^{\infty} \dv{\psi^*}{x} \dv{\psi}{x} \dd{x} \quad \text{Integrating by parts}\\
		&= \frac{\hbar^2 B^2}{2 m} \int_{-\infty}^{\infty} \qty[\dv{}{x} \qty(x e^{-\beta x^2/2})]^2 \dd{x}\\
		&= \frac{\hbar^2 B^2}{2 m} \int_{-\infty}^{\infty} \qty[e^{-\beta x^2/2} - \beta x^2 e^{-\beta x^2/2}]^2 \dd{x} \\
		&= 	\frac{\hbar^2 B^2}{2 m} \int_{-\infty}^{\infty} \qty[1 - 2\beta x^2 + \beta^2 x^4] e^{-\beta x^2} \dd{x} \\
		&= \frac{\hbar^2 B^2}{2 m} \left[
		\sqrt{\frac{\pi}{\beta}} - 2 \beta \cdot \frac{1}{2\beta} \sqrt{\frac{\pi}{\beta}}  + \beta^2 \frac{3}{4 \beta^2} \sqrt{\frac{\pi}{\beta}} 
		\right] \\
		&= \frac{\hbar^2 B^2}{2 m} \sqrt{\frac{\pi}{\beta}} \qty[1 - 1 + \frac{3}{4}] \\
		&= \frac{3}{4} \frac{\hbar^2 B^2}{2 m} \sqrt{\frac{\pi}{\beta}} \\
		&= \frac{3}{4} \frac{\hbar^2}{2 m} \frac{2 \beta^{3/2}}{\sqrt{\pi}}  \sqrt{\frac{\pi}{\beta}} \\
		&= \frac{3}{4} \frac{\beta \hbar^2}{m}
	\end{align*}
	and
	\begin{align*}
		\expval{V}_\beta 
		&= \int_{-\infty}^{\infty} \psi^*\qty(x) V\qty(x) \psi\qty(x) \dd{x} \\
		&= \frac{1}{2} m \omega_0^2 B^2 \int_{-\infty}^{\infty} x e^{-\beta x^2/2} x^2 x e^{-\beta x^2/2} \dd{x}\\
		&= \frac{1}{2} m \omega_0^2 B^2 \int_{-\infty}^{\infty} x^4 e^{-\beta x^2} \dd{x} \\
		&= \frac{1}{2} m \omega_0^2 B^2 \frac{3}{4 \beta^2} \sqrt{\frac{\pi}{\beta}} \\
		&= \frac{1}{2} m \omega_0^2 \frac{2 \beta^{3/2}}{\sqrt{\pi}} \frac{3}{4 \beta^2} \sqrt{\frac{\pi}{\beta}} \\
		&= \frac{3}{4} \frac{m \omega_0}{\beta}
	\end{align*}
	
	\begin{align}
		\therefore\ \expval{H}_\beta = \frac{3}{4} \qty(\frac{\hbar^2 \beta}{m} + \frac{m\omega_0^2}{\beta}) 
	\end{align}
	Next, we minimize $\expval{H}_\beta$ by choosing the approximate value of $\beta$. Se set
	\begin{align*}
		\pdv{}{\beta} \expval{H}_\beta &= 0\\
		or,\ \frac{\hbar^2}{m} - \frac{m \omega_0^2}{\beta^2} &=0\\
		or,\ \beta^2 = \frac{m^2 \omega_0^2}{\hbar^2}
	\end{align*}
	Since $\beta$ is a positive parameter, we have
	\begin{align*}
		\beta = \beta_0 = \frac{m \omega_0}{\hbar}
	\end{align*}
	Thus, the minimize vale of $\expval{H}_\beta$ is
	\begin{align*}
		\expval{H}_{min} = \expval{H}_{\beta_0} 
		&= \frac{3}{4} \left[\frac{\hbar^2  m\omega_0/\hbar}{m}  +  \frac{m \omega_0^2}{m \omega_0/\hbar}\right]\\
		&= \frac{3}{4} \left[\hbar \omega_0 + \hbar \omega_0\right]\\
		&= \frac{3}{2} \hbar \omega_0
	\end{align*}
	Therefore,
	\begin{align*}
		E_1 &\leq \expval{H}_{min}\\
		i.e., \ E_1 &\leq \frac{3}{2} \hbar \omega_0
	\end{align*}
	
	
	
	\subsection{Hydrogen Atom}
	%%% page 14
	\subsubsection{Ground State}
	The ground state is spherically symmetric. therefore let us choose a trial wavefunction of the form
	\begin{align}
		\psi\qty(\vec{r}) = A e^{-\beta r}
	\end{align}
	We normalize $\psi\qty(\vec{r})$, i.e.,
	\begin{align*}
		\int \psi^*\qty(\vec{r}) \psi\qty(\vec{r}) \dd[3]{r} &= 1 \\
		or, \ \abs{A}^2 \int_{0}^{\infty} e^{-2\beta r} r^2 \dd{r} \int_{\Omega} \dd{\Omega} &= 1 \\
		or, \ 4\pi \abs{A}^2 \int_{0}^{\infty} e^{-2\beta r} r^2 \dd{r} &= 1 \\
		or, \ 4\pi \abs{A}^2 \frac{2 !}{\qty(2\beta)^3} &= 1 \\
		or, \ \abs{A}^2 = \frac{\beta^3}{\pi}
	\end{align*}
	We have used the solid angle $\dd{\Omega} = \sin\theta \dd{\theta} \dd{\phi}$.
	
	Choosing $A$ to be real and positive we have
	\begin{equation}
		A = \sqrt{\frac{\beta^3}{\pi}}
	\end{equation}
	The normalized trial wavefunction for the ground state is then
	\begin{equation}
		\psi\qty(\vec{r}) = \sqrt{\frac{\beta^3}{\pi}} e ^{-\beta r}
	\end{equation}
	Now, the Hamiltonian for the hydrogen atom is
	\begin{align}
		H = - \frac{\hbar^2}{2 m} \laplacian{} - \frac{e^2}{4\pi\epsilon}\frac{1}{r}
	\end{align}
	Therefore
	\begin{align}
		\expval{H} = \expval{- \frac{\hbar^2}{2 m} \laplacian{}}  + \expval{- \frac{e^2}{4\pi\epsilon}\frac{1}{r}}  = \expval{T}  + \expval{V} 
	\end{align}
	Consider the expectation value of kinetic energy
	\begin{align}
		\expval{T}  =  - \frac{\hbar^2}{2 m}\expval{\laplacian{}} = - \frac{\hbar^2}{2 m} \int \psi^*\qty(r) \laplacian{\psi\qty(r)} \dd[3]{r}
	\end{align}
	We use the identity
	\begin{align}
		\divergence \qty(\psi^* \grad{\psi}) = \psi^*\laplacian{\psi} + \grad{\psi^*} \cdot \grad{\psi}
	\end{align}
	
	\begin{align}
		\therefore \int \psi^*\qty(r) \laplacian{\psi\qty(r)} \dd[3]{r} = \int \divergence \qty(\psi^* \grad{\psi}) \dd[3]{r} - \int \grad{\psi^*} \cdot \grad{\psi} \dd[3]{r}
		\end{align}
		The first integral on the right can be converted to a surface integral using Gauss's theorem, and since the surface is at infinity, the integrand vanishes since $\psi$ vanishes for $r \rightarrow \infty$.
		
		Hence we have
		\begin{align*}
			\expval{T} 
			&= -\frac{\hbar^2}{2 m} \int \psi^*\laplacian{\psi} \dd[3]{r} \\
			&= \frac{\hbar^2}{2 m} \int \grad{\psi^*} \cdot \grad{\psi} \dd[3]{r} \\
			&= \frac{\hbar^2}{2 m} \int \abs{\grad{\vec{\psi}}^2} \dd[3]{r}
		\end{align*}
		Now, $\psi$ is only a function of $r\equiv \abs{\vec{r}}$. Therefore
		\begin{align*}
			\grad{\psi} = \hat{r} \dv{}{r} \psi = -\hat{r} \beta \psi\qty(r)
		\end{align*}
		Hence
		\begin{align}
			\abs{\grad{\psi}}^2 = \beta^2 \psi^2\qty(r)
		\end{align}
		Thus
		\begin{align}
			\expval{T} = \frac{\hbar^2 \beta^2}{2 m} \int \psi^2\qty(r) \dd[3]{r} = \frac{\hbar^2 \beta^2}{2 m} 
		\end{align}
		Next we calculate the expectation value of the potential energy.
		
		%%%% 17
		Next, we calculate the expectation value of the potential energy.
		\begin{align*}
			\expval{V} &= -\frac{e^2}{4\pi \epsilon_0} \expval{\frac{1}{r}}\\
			&= -\frac{e^2}{4\pi \epsilon_0} \int \psi^*\qty(r) \frac{1}{r} \psi\qty(r) \dd[3]{r}
		\end{align*}
		Since $\psi$ and $\psi^*$ do not depend upon $\theta$ or $\phi$, we can easily integrate over $\theta$ and $\phi$ to get $4\pi$. Hence $\dd[3]{r} = 4 \pi r^2 \dd{r}$. Then
		\begin{align*}
			\expval{V} 
			&= - \frac{e^2}{4 \pi \epsilon_0} \qty(4\pi) \int_{0}^{\infty} \psi^2\qty(r) r \dd{r}\\
			&= - \frac{e^2}{4 \pi \epsilon_0} \qty(4\pi) \frac{\beta^3}{\pi} \int_{0}^{\infty} e^{-2\beta r} r \dd{r}\\
			&= - \frac{e^2}{4 \pi \epsilon_0} \qty(4\pi) \frac{\beta^3}{\pi} \frac{1}{4\beta^2}\\
			&= - \frac{\beta e^2}{4 \pi \epsilon_0}
		\end{align*}
		Hence, we obtain
		\begin{align*}
			\expval{H}_\beta = \frac{\hbar^2 \beta^2}{2 m } - \frac{\beta e^2}{4\pi \epsilon_0}
		\end{align*}
		Now we minimize $\expval{H}_\beta$
		\begin{align*}
			\pdv{}{\beta} \expval{H}_\beta &= 0\\
			or, \ \frac{\hbar^2 \beta}{m} - \frac{e^2}{4\pi\epsilon_0} &= 0 \\
			or, \ \beta=\beta_0 = \frac{m e^2}{4\pi \epsilon_0 \hbar^2} = \frac{1}{a_0}
		\end{align*}
		Where $a_0$ is the Bohr radius. Thus, the minimum value of $\expval{H}$ is 
		\begin{align*}
			\expval{H}_{min} =  \expval{H}_{\beta_0} &= \frac{\hbar^2\beta_0^2}{2 m} - \frac{\beta_0 e^2}{4 \pi \epsilon_0}\\
			&= \frac{\hbar^2}{2 m a_0^2} - \frac{e^2}{4\pi \epsilon_0 a_0}\\
			&= \qty(\frac{\hbar^2}{2 m a_0}  -  \frac{e^2}{4\pi \epsilon_0}) \frac{1}{a_0} \\
			&= \qty(\frac{\hbar^2}{2 m \frac{4\pi \epsilon_0 \hbar^2}{m e^2}}  -  \frac{e^2}{4\pi \epsilon_0}) \frac{1}{a_0} \\
			&= \frac{1}{4\pi \epsilon_0} \qty(\frac{e^2}{2} - e^2) \frac{1}{a_0}\\
			&= - \frac{e^2}{\qty(4\pi \epsilon_0) 2 a_0}
		\end{align*}
		Therefore
		\begin{align*}
			E_0 \leq - \frac{e^2}{\qty(4\pi \epsilon_0) 2 a_0}
		\end{align*}
		
		The variational estimate of the ground state wavefunction is
		\begin{align}
			\psi = \qty(\frac{\beta^3}{\pi})^{1/2} e^{-\beta r} = \qty(\frac{1}{\pi a_0^3})^{1/2} e^{-r/a_0}
		\end{align}
		
		In this example, variational estimates of the ground state energy and ground state wavefunction coincide with the exact values.
		
		
	
	\subsection{Helium Atom}
	%%% page 21
\begin{figure}
	%%%%%%% TODO
	\centering
	\includegraphics[width=0.5\linewidth]{Pictures/not-found.jpg}
	\caption{Helium Atom}
\end{figure}
	The Hamiltonian is
	\begin{equation}
		H = - \frac{\hbar^2}{2 m} \qty(\nabla_1^2 + \nabla_2^2) - \frac{2 e^2}{4 \pi \epsilon_0} \qty(\frac{1}{r_1} + \frac{1}{r_2})  +  \frac{e^2}{4 \pi \epsilon_0 r_{1 2}}
		\label{chapter21.eqn1-example-helium}
	\end{equation}
	Next, we have to choose an approximate trial wave function. We note that, for a helimum ion, the exact ground state wave function is
	\begin{align}
		\psi_{1 0 0}\qty(\vec{r}) = \sqrt{\frac{Z^3}{\pi a_0^3}} e^{-Z r / a_0} \quad (Z=2)
	\end{align}
	Where $a_0$ is the Bohr radius. If we neglect the interaction $e^2 / 4\pi \epsilon_0 r_{1 2}$ between the two electrons of the helium atom, the wave function of the helium atom can be written as
	\begin{align*}
		\psi\qty(\vec{r_1}, \vec{r_2}) 
		&= \psi_{1 0 0}\qty(\vec{r_1}) \psi_{1 0 0}\qty(\vec{r_2}) \\
		&= \frac{Z^3}{\pi a_0^3} e^{-Z \qty(r_1 + r_2)/a_0}
	\end{align*}
	
	Of course $\psi\qty(\vec{r_1}, \vec{r_2}) $ cannot be the exact wavefunction of the helium atom since we have left  out the electron-electron interaction. However, we can use $\psi\qty(\vec{r_1}, \vec{r_2}) $ as the trial wave function, and to tka e the mutual interaction between the two electron into account, we will take $Z$ appearing in $\psi\qty(\vec{r_1}, \vec{r_2}) $ to be a free parameter (nor $Z=2$). However $Z=2$ appearing in the Hamiltonian is kept unchanged.
	
	Next, we calculate the expectation value of $H$ using the trial wave function $\psi\qty(\vec{r_1}, \vec{r_2}) $.
	
	The expectation value of the Hamiltonian is 
	\begin{align*}
		\expval{H}_Z 
		&= \int \psi^* H \psi \dd[3]{r_1}  \dd[3]{r_2} \\
		&= \int \psi^*\qty(\vec{r_1}, \vec{r_2}) \left[
		- \frac{\hbar^2}{2 m} \qty(\nabla_1^2 + \nabla_2^2) - \frac{2 e^2}{4 \pi \epsilon_0} \qty(\frac{1}{r_1} + \frac{1}{r_2})  +  \frac{e^2}{4 \pi \epsilon_0 r_{1 2}}
		\right] \psi\qty(\vec{r_1}, \vec{r_2}) \dd[3]{r_1} \dd[3]{r_2}
	\end{align*}
	\underline{Expectation values of $\laplacian{}_1$ and $\laplacian{}_2$}
	\begin{align*}
		\expval{\nabla_1^2} 
		&= \int \psi^*\qty(\vec{r_1}, \vec{r_2}) \nabla_1^2 \psi\qty(\vec{r_1}, \vec{r_2}) \dd[3]{r_1} \dd[3]{r_2} \\
		&= \int \psi^*_{1 0 0}\qty(\vec{r_1}) \nabla_1^2 \psi_{1 0 0}\qty(\vec{r_1})  \dd[3]{r_1}\\
		&= \frac{Z^3}{\pi a_0^3} \int e^{-Z r/a_0} \nabla^2 e^{-Z r/a_0} \dd[3]{r} \\
		&= -\frac{Z^3}{\pi a_0^3} \int \qty(\grad{e^{-Z r/a_0}}) \cdot \qty(\grad{e^{-Z r/a_0}}) r^2 \dd{r}\dd{\Omega} \\
		&= -\frac{Z^3}{\pi a_0^3} \qty(4\pi) \frac{Z^2}{a_0^2} \int e^{-2 Z /a_0}  r^2 \dd{r}\\
		&= -\frac{Z^3}{\pi a_0^3} \qty(4\pi) \frac{Z^2}{a_0^2} \frac{2}{\qty(2 Z / a_0)^3} = - \frac{Z^2}{a_0^2}
	\end{align*}
	%%% page 23
	The expectation value of $\nabla_2^2$ is the same as that of $\nabla_1^2$ because the trial wave function is symmetric under interchange of $r_1$ and $r_2$. Thus
	\begin{align}
		\expval{\nabla_1^2} = 		\expval{\nabla_2^2} &= - \frac{Z^2}{a_0^2} \nonumber\\
		\therefore \ \expval{-\frac{\hbar^2}{2 m}\nabla_1^2} = \expval{-\frac{\hbar^2}{2 m}\nabla_2^2} &= \frac{\hbar^2 Z^2}{2 m a_0^2} \nonumber\\
		&= \frac{\hbar^2 Z^2}{2 m \frac{4 \pi \epsilon_0 \hbar^2}{m e^2} a_0} \nonumber\\
		&= \frac{Z^2 e^2}{ \qty(4\pi \epsilon_0) 2 a_0}
		\label{chapter21.eqn2-example-helium}
	\end{align}
	
	%%% page 25
	\underline{Expectation values of $\frac{1}{r_1}$ and $\frac{1}{r_2}$}\\
	We first note that expectation values of $1/r_1$ and $1/r_2$ are equal since the trial wave function is symmetric under the interchange of $r_1$ and $r_2$. Now
	\begin{align*}
		\expval{\frac{1}{r_1}} 
		&= \int \psi^*\qty(\vec{r_1}, \vec{r_2}) \frac{1}{r_1} \psi\qty(\vec{r_1}, \vec{r_2}) \dd[3]{r_1} \dd[3]{r_2} \\
		&= \int \psi^*_{1 0 0}\qty(\vec{r_1}, \vec{r_2}) \frac{1}{r_1} \psi_{1 0 0}\qty(\vec{r_1}, \vec{r_2}) \dd[3]{r_1} \\
		&= \frac{Z^3}{\pi a_0^3} \int e^{-Z r/a_0} \frac{1}{r} e^{-Z r/a_0} r^2 \dd{r} \dd{\Omega}\\
		&= \frac{Z^3}{\pi a_0^3} \qty(4\pi) \int_{0}^{\infty} e^{-2Z r/a_0} r \dd{r} \\
		&= \frac{Z^3}{\pi a_0^3} \qty(4\pi) \frac{1}{\qty(2 Z / a_0)^2}\\
		&= \frac{Z}{a_0}
	\end{align*}
	Hence
	\begin{align}
		\expval{-\frac{2 e^2}{4\pi \epsilon_0 r_1}} = \expval{-\frac{2 e^2}{4\pi \epsilon_0 r_2}} = - \frac{2 e^2 Z}{4\pi \epsilon_0 a_0}
		\label{chapter21.eqn3-example-helium}
	\end{align}
	
	\underline{Expectation value of $\frac{e^2}{4\pi \epsilon_0 r_{1 2}}$}
	\begin{align}
		\expval{\frac{e^2}{4\pi \epsilon_0 r_{1 2}}} &= \frac{e^2}{4\pi \epsilon_0} \expval{\frac{1}{r_{1 2}}} \\
		&= \frac{e^2}{4\pi \epsilon_0} \int \psi^*\qty(\vec{r_1}, \vec{r_2}) \frac{1}{r_{1 2}} \psi\qty(\vec{r_1}, \vec{r_2}) \dd[3]{r_1} \dd[3]{r_2} \\
		&= \frac{e^2}{4\pi \epsilon_0} \qty(\frac{Z^3}{\pi a_0^3})^{2} \int e^{-2 Z \qty(r_1 + r_2)/a_0} \frac{1}{r{1 2}} \dd[3]{r_1} \dd[3]{r_2}
	\end{align}
	
	We will now make a change of variable:
	Let
	\begin{align*}
		\frac{2 Z}{a_0} \vec{r_1} &= \vec{\rho_1} \\
		\frac{2 Z}{a_0} \vec{r_2} &= \vec{\rho_2} \\
	\end{align*}
	Therefore
	\begin{align*}
		\frac{2 Z}{a_0} \qty(\vec{r_1} - \vec{r_2}) &= \vec{\rho_1}  -  \vec{\rho_2} \\
		i.e., \quad \frac{2 Z}{a_0} \vec{r}_{12} &= \vec{\rho}_{12}
	\end{align*}

	Hence
	\begin{align}
		\expval{\frac{e^2}{4\pi \epsilon_0 r_{1 2}}} &= \frac{e^2}{4\pi \epsilon_0}  \frac{Z^6}{\pi^2 a_0^6} \frac{2 Z}{a_0} \qty(\frac{a_0}{2 Z})^6  \int \frac{e^{-\qty(\rho_1 + \rho_2)}}{\rho_{12}} \dd[3]{\rho_1} \dd[3]{\rho_2} \nonumber\\
		&= \frac{e^2}{4\pi \epsilon_0} \frac{Z}{32 \pi^2 a_0} \int \frac{e^{-\qty(\rho_1 + \rho_2)}}{\rho_{12}} \dd[3]{\rho_1} \dd[3]{\rho_2} \nonumber\\
		&= \frac{e^2}{4\pi \epsilon_0} \frac{Z}{32 \pi^2 a_0} \cdot 20 \pi^2 \nonumber\\
		&= \frac{e^2}{4\pi \epsilon_0} \frac{5 Z}{8 a_0} 
		\label{chapter21.eqn4-example-helium}
	\end{align}
	
	Substituting equations (\ref{chapter21.eqn2-example-helium}, \ref{chapter21.eqn3-example-helium},\ref{chapter21.eqn4-example-helium}) in equation (\ref{chapter21.eqn1-example-helium}) we get
	\begin{align*}
		\expval{H}_Z 
		&= \frac{Z^2 e^2}{4\pi \epsilon_0 a_0} - \frac{4 Z^2 e^2}{4\pi \epsilon_0 a_0} + \frac{ 5 Z^2 e^2}{\qty(4\pi \epsilon_0) 8 a_0}\\
		&= frac{e^2}{4\pi \epsilon_0 a_0} \qty(Z^2 - 4 Z + \frac{5 Z}{8}) \\
		&= frac{e^2}{4\pi \epsilon_0 a_0} \qty(Z^2 - \frac{27 Z}{8})
	\end{align*}
	Next, we minimize $\expval{H}_Z$ by varying $Z$
	\begin{align*}
		\pdv{}{Z} \expval{H}_Z & = 0\\
		2 Z - \frac{27}{8} &= 0\\
	\therefore \	Z &= Z_0 = \frac{27}{16}
	\end{align*}
	
	Thus the lowest upper bound for the ground state energy of the helium atom is
	\begin{align*}
		E_0 &\leq \expval{H}_{min} \\
		&= \expval{H}_{Z_0}\\
		&= \frac{e^2}{4\pi \epsilon_0 a_0} \left[
		\qty(\frac{27}{16})^2 - \frac{27}{8} \frac{27}{16} \right]\\
		&= - \qty(\frac{27}{16})^2 \frac{e^2}{4\pi \epsilon_0 a_0} \\
		&= -5.7 \frac{e^2}{\qty(4\pi \epsilon_0) 2 a_0}		
	\end{align*}
	Experimental value for the ground state energy of the helium atom is
	\begin{equation*}
		\qty(E_0)_{expt} = - 5.81 \frac{e^2}{\qty(4\pi \epsilon_0) 2 a_0}
	\end{equation*}
	The disagreement is only $2\%$.\\
	
	In the variational calculation, the hydrogenic wave function gives the best value for the ground state energy of the helium atom when  $Z=27/16$ rather than $Z=2$. This indicates that each electron screens the nucleus from the other electron. Therefore, the effective nuclear charge is reduced.
	
	
	\section{Note}
	%%% page 31
	The variational method is in general more accurate for estimation of energy than for the wave function. Suppose we choose a trial ground state $\ket{\psi}$ which differs from the exact ground state $\psi_0$ by $\ket{\delta \psi}$, i.e.,
	\begin{align}
		\label{chapter21.eqn1-note}
		\ket{\psi} &= \ket{\psi_0} \braket{\psi_0}{\psi} + \ket{\delta \psi}\\
		\label{chapter21.eqn2-note}
		i.e,\ \ket{\psi} &= C_0 \ket{\psi_0}  + \ket{\delta \psi}
	\end{align}
	Where
	\begin{align}
		\label{chapter21.eqn3-note}
		C_0 = \braket{\psi_0}{\psi}
	\end{align}
	is the component of the trial wave function along the exact ground state $\ket{\psi_0}$. The deviation of $\ket{\psi}$ from $\ket{\psi_0}$, i.e., $\ket{\delta \psi}$, is orthogonal to exact ground state $\ket{\psi_0}$ as can be seen by taking the scalar product of (\ref{chapter21.eqn1-note}) with $\bra{\psi_0}$ and noting that $\braket{\psi_0}{\psi_0} = 1$.
	
	
	Now, the variational estimate to the ground state energy is
	\begin{align*}
		E &= \frac{\expval{H}{\psi}}{\braket{\psi}{\psi}} \\
		i.e., E &= \frac{\qty(C^*_0 \bra{\psi_0} + \bra{\delta \psi})  H  \qty(C_0 \ket{\psi_0}  +  \ket{\delta \psi})}{\qty(C^*_0 \bra{\psi_0} + \bra{\delta \psi}) \qty(C_0 \ket{\psi_0}  +  \ket{\delta \psi})}\\
		&= \frac{\abs{C_0}^2 \expval{H}{\psi_0} + C^*_0 \mel{\psi_0}{H}{\delta\psi} + C_0 \mel{\delta\psi}{H}{\psi_0} + \order{\delta\psi^2}}
		{\abs{C_0}^2 \braket{\psi_0}{\psi_0} + C^*_0 \braket{\psi_0}{\delta\psi} + C_0 \braket{\delta\psi}{\psi_0} + \order{\delta\psi^2}} \\
		&= \frac{\abs{C_0}^2 E_0  +  \order{\delta\psi^2}}{\abs{C_0}^2  +  \order{\delta\psi^2}} \\
		&= E_0 + \order{\delta\psi^2}
	\end{align*}
	i.e., $E$ differs from $E_0$ in the second order in $\delta\psi$. Hence $E$ is an accurate estimate of $E_0$.
	
	
	
	
	
	
	\section{Appendix}
	Show that
	\begin{align}
		\label{chapter21.eqn1-appendix}
		I \equiv \iint \frac{e^{-\qty(r_1 + r_2)}}{r_{1 2}} \dd[3]{r_1} \dd[3]{r_2} = 20 \pi^2
	\end{align}
	We will do the $\vec{r_2}$ integral first. For this purpose $\vec{r_1}$ is fixed and we align the coordinate system so that $\vec{r_1}$ lies along the $z$-axis.
	%%% page 33
	\begin{figure}
		%%%%%%% TODO
		\centering
		\includegraphics[width=0.5\linewidth]{Pictures/not-found.jpg}
		\caption{Alignment of $\vec{r_1}$ in a coordinate system.}
	\end{figure}
	Now
	\begin{align}
		I &= \int e^{-r_1} \dd[3]{r} \int \frac{e^{-r_2}}{\abs{\vec{r_1} - \vec{r_2}}} \dd[3]{r_2} \nonumber \\
		&= \int e^{-r_1} \dd[3]{r} J 
		\label{chapter21.eqn2-appendix}
	\end{align}
	Where
	\begin{align}
		J &= \int \frac{e^{-r_2}}{\abs{\vec{r_1} - \vec{r_2}}} \dd[3]{r_2} \nonumber\\
		&= \int \frac{e^{-r_2} r_2^2 \dd{r_2} \sin\theta_2 \dd{\theta_2} \dd{\phi_2}}{\sqrt{r_1^2 + r_2^2 - 2 r_1 r_2 \cos\theta_2}} \nonumber\\
		&= 2\pi \int_{0}^{\infty} e^{-r_2} r_2^2 \dd{r_2} \int_{0}^{\pi} \frac{\sin\theta_2\dd{\theta_2}}{\sqrt{r_1^2 + r_2^2 - 2 r_1 r_2 \cos\theta_2}} \nonumber\\
		\label{chapter21.eqn3-appendix}
		&= 2\pi \int_{0}^{\infty} e^{-r_2} r_2^2 \dd{r_2} K 
	\end{align}
	Now we do the $\theta_2$ integral
	\begin{align*}
		K = \int_{0}^{\pi} \frac{\sin\theta_2\dd{\theta_2}}{\sqrt{r_1^2 + r_2^2 - 2 r_1 r_2 \cos\theta_2}}
	\end{align*}
	Let
	\begin{align*}
		z &= r_1^2 + r_2^2 - 2r_1 r_2 \cos\theta_2 \\
		dz &= 2 r_1 r_2 \sin\theta_2 \dd{\theta_2}
	\end{align*}
	Hence 
	\begin{align*}
		K &= \int_{\theta_2=0}^{\theta_2=\pi} \frac{\dd{z}}{2 r_1 r_2} z^{-1/2} \\
		or, \ K &= \frac{1}{2 r_1 r_2}  \int_{\theta_2=0}^{\theta_2=\pi} z^{-1/2} \dd{z} \\
		&= \frac{1}{2 r_1 r_2}  \eval{\frac{z^{1/2}}{-\frac{1}{2} + 1}}_{\theta_2=0}^{\theta_2=\pi}\\
		&= \frac{1}{r_1 r_2} \eval{z^{1/2}}_{\theta_2=0}^{\theta_2=\pi}  \\
		&= \frac{1}{r_1 r_2} \eval{\sqrt{r_1^2 + r_2^2 - 2r_1 r_2 \cos\theta_2}}_{\theta_2=0}^{\theta_2=\pi} \\
		&= \frac{1}{r_1 r_2} \left[\sqrt{r_1^2 + r_2^2 + 2 r_1 r_2}  -  \sqrt{r_1^2 + r_2^2 - 2 r_1 r_2}\right] \\
		&= \frac{1}{r_1 r_2} \left[\qty(r_1 + r_2)  -  \abs{r_1  -  r_2}\right] \\
		&= \begin{cases}
			\frac{1}{r_1 r_2} \left[\qty(r_1 + r_2)  -  \qty(r_1  -  r_2)\right] \ \text{if} \ r_2 < r_1
			\\
			\frac{1}{r_1 r_2} \left[\qty(r_1 + r_2)  -  \qty(r_2  -  r_1)\right] \ \text{if} \ r_2 > r_1
		\end{cases} \\
		K &= \begin{cases}
			\frac{2 r_2}{r_1 r_2} \ \text{if} \ r_2 < r_1\\
			\frac{2 r_1}{r_1 r_2}  \ \text{if} \ r_2 > r_1
		\end{cases}
	\end{align*}
	Thus
	\begin{align}
		\label{chapter21.eqn4-appendix}
		K &= \begin{cases}
		\frac{2}{r_1} \ \text{if} \ r_2 < r_1\\
		\frac{2}{r_2}  \ \text{if} \ r_2 > r_1
		\end{cases}
	\end{align}
	
	Substituting equation (\ref{chapter21.eqn4-appendix}) in equation (\ref{chapter21.eqn3-appendix}). We get
	
	\begin{align}
		J &= \qty(2 \pi) \int_{0}^{\infty} e^{-r_2} r_2^2 \dd{r_2} 
		\begin{cases}
			\frac{2}{r_1} \ \text{if} \ r_2 < r_1 \\
			\frac{2}{r_2} \ \text{if} \ r_2 > r_1
		\end{cases}\nonumber\\
		&= 4\pi \left[
		\frac{1}{r_1} \int_{0}^{r_1} e^{-r_2} r_2^2 \dd{r_2} + \int_{r_1}^{\infty} e^{-r_2} r_2 \dd{r_2}
		\right]
		\label{chapter21.eqn5-appendix}
	\end{align}
	Using the standard integrals from appendix (\ref{appendix1.Integrals}), equation (\ref{chapter21.eqn5-appendix}) becomes
	\begin{align*}
		J &= 4 \pi \left[
		\frac{1}{r_1} \eval[-\qty(2 + 2 r_2 + r_2^2) e^{-r_2}|_{0}^{r_1} + \eval[-\qty( 1 + r_2) e^{-r_2}|_{r_1}^{0}
		\right]\\
		&= 4\pi \left[
		-\frac{1}{r_1} \qty(2 + 2 r_1 + r_1^2) e^{-r_1}
		+ \frac{2}{r_1} + \qty(1 + r_1) e^{-r_1}
		\right] \\
		&= \frac{4\pi}{r_1} \left[
		- \qty(2 + 2 r_1 + r_1^2) e^{-r_1}
		+ 2 + \qty(r_1 + r_1^2) e^{-r_1}
		\right] \\
		&= \frac{4\pi}{r_1} \left[
		-2 e^{-r_1} - r_1 e^{-r_1} + 2
		\right]
	\end{align*}
	\begin{equation}
		J = \frac{4\pi}{r_1} \left[
		-\qty(r_1  +  2) e^{-r_1} + 2
		\right]
		\label{chapter21.eqn6-appendix}
	\end{equation}
	
	Finally substitute equation (\ref{chapter21.eqn6-appendix}) in equation (\ref{chapter21.eqn2-appendix}). We get
	\begin{align*}
		I 
		&= \int_{0}^{\infty} e^{-r_1} r_1^2 \dd{r_1} \int \dd{\Omega} \frac{4\pi}{r_1} \left[2 - \qty(r_1 + 2) e ^{-r_1} \right]\\
		&= 16 \pi^2 \int_{0}^{\infty} \dd{r_1} e^{-r_1} r_1 \left[2 - \qty(r_1 + 2)e^{-r_1}\right] \\
		&= 16\pi^2 \left[
		2 \int_{0}^{\infty} e^{-r_1} r_1 \dd{r_1}
		- \int_{0}^{\infty} e^{-2 r_1} \qty(r_1^2 + 2 r_1) \dd{r_1}
		\right]\\
		&= 16 \pi^2 \left[2 - \left[\frac{2}{8} + 2 \cdot \frac{1}{4}\right]\right] \\
		&= 16 \pi^2 \left[2 - \frac{3}{4}\right] \\
		&= 20 \pi^2
	\end{align*}
	
	Hence equation (\ref{chapter21.eqn1-appendix}) is proved.
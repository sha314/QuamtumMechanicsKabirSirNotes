%% writing by shahnoor

\chapter{sheet-18 : Rotations and Angular Momentum}

\ifpdf
\graphicspath{{Chapter18/figs/}}
\else
\graphicspath{{Chapter18/figs/}}
\fi

reference : Sakurai, Cohen-Tanondgi\\

A rotation of a physical system is specified by the angle of rotation and the axis of rotation. The rotation can be either positive or negative. If a right-handed screw turned in the direction of rotation proceeds along the positive direction of the axis, the rotation is said to be positive. Thus, for examp, $\phi\hat{z}$ denotes a positive rotation by an angle $\phi$ about the z-axis Fig. (\ref{chapter18.fig1})

\begin{figure}
	%%%%%%% TODO
	\centering
	\includegraphics[width=0.5\linewidth]{Pictures/not-found.jpg}
	\caption{A positive rotation of the physical system by an angle $\phi$ about the $z$-axis.}
	\label{chapter18.fig1}
\end{figure}
In our subsequent discussions we will consider active rotation\index{active and passive rotation}, i.e., rotation of the physical system rather than the rotation of the coordinates.\\

Now, finite rotations about different axes do not commute, i.e., the change in the coordinate of a point in the physical system depends on the order the rotations are performed. To work out quantitatively the extent in which rotations about different axes fail to commute, we have to construct the matrices corresponding to rotations in the three-dimensional real space $(x,y,z)$.\\

In each rotation a vector $\vec{r}$ with coordinates $(x, y, z)$ changes to a new vector $\vec{r^\prime}$ with coordinate $(x^\prime,y^\prime,z^\prime)$. The matrix connecting $(x^\prime,y^\prime,z^\prime)$ with $(x,y,z)$ is the matrix corresponding to the rotation.

%% page 3
Thus
\begin{equation}
\label{chapter18.eqn1}
\mqty[x^\prime\\ y^\prime \\ z^\prime] = \mqty[R]  \mqty[x\\ y \\ z]
\end{equation}
Where $R$ is the $3\times3$ square matrix corresponding to the rotation. In a rotation, the length of the vector $\vec{r}$ remains unchanged, i.e.,
\begin{equation}
\norm{\vec{r^\prime}} = \sqrt{{x^{\prime}}^2 + {y^{\prime}}^2 + {z^{\prime}}^2} = \sqrt{x^2 + y^2 + z^2} =\norm{\vec{r}}
\end{equation}
Therefore, the matrix $R$ must be orthogonal, i.e.,
\begin{equation}
R^T R = R R^T = \mathcal{1}
\end{equation}
Next, we will construct explicitly the rotation matrices $R$ in the three-dimensional space corresponding to positive rotations about the $x$-axis, $y$-axis and $z$-axis.

First, consider a finite rotation by an angle $\phi$ in a positive sense about the $z$-axis (Fig. \ref{chapter18.fig2})
\begin{figure}
	%%%%%%% TODO
	\centering
	\includegraphics[width=0.5\linewidth]{Pictures/not-found.jpg}
	\caption{A positive rotation by an angle $\phi$ about the $z$-axis.}
	\label{chapter18.fig2}
\end{figure}
We have
\begin{align*}
x^\prime = \cos\phi x - \sin\phi y \\
y^\prime = \sin\phi x + \cos\phi y
z^\prime = z
\end{align*}
In matrix form
\begin{equation*}
	\mqty[x^\prime\\ y^\prime \\ z^\prime] = \mqty[\cos\phi & - \sin\phi& 0 \\
		\sin\phi & \cos\phi & 0 \\
		0& 0& 1]  
	\mqty[x\\ y \\ z]
\end{equation*}
Thus, the rotation matrix corresponding to a positive rotation by an angle $\phi$ about the $z$-axis is 
\begin{equation}
\label{chapter18.eqn4}
R_z\qty(\phi) = \mqty[\cos\phi & - \sin\phi& 0 \\
\sin\phi & \cos\phi& 0 \\
0& 0& 1]
\end{equation}
Next, consider a rotation about the $x$-axis. The corresponding matrix is
\begin{equation}
\label{chapter18.eqn5}
R_x\qty(\phi) = \mqty[1& 0& 0 \\
0& \cos\phi & - \sin\phi\\
0& \sin\phi & \cos\phi]
\end{equation}

Similarly
\begin{equation}
\label{chapter18.eqn6}
R_y\qty(\phi) = \mqty[\cos\phi& 0& \sin\phi \\
0& 1& 0 \\
-\sin\phi& 0 &\cos\phi]
\end{equation}

%% page 6
For infinitesimal rotation, i.e., $\phi=\epsilon$, the rotation matrices, up to second order in $\epsilon$ are
\begin{align}
\label{chapter18.eqn7}
\begin{split}
R_x\qty(\epsilon) = \mqty[1 & 0 & 0 \\ 0 & 1-\frac{\epsilon^2}{2} &- \epsilon \\ 0 & \epsilon & 1 - \frac{\epsilon^2}{2}] \\
R_y\qty(\epsilon) = \mqty[1 - \frac{\epsilon^2}{2} & 0 & \epsilon \\ 0 & 1 & 0 \\ -\epsilon & 0 & 1 - \frac{\epsilon^2}{2}] \\
R_z\qty(\epsilon) = \mqty[1 - \frac{\epsilon^2}{2} & -\epsilon & 0 \\ \epsilon & 1 - \frac{\epsilon^2}{2} & 0 \\ 0 & 0 & 1]
\end{split}
\end{align}
Now the multiplication leads to (up to $\order{\epsilon^2}$)
\begin{equation}
\label{chapter18.eqn8a}
R_x\qty(\epsilon) R_y\qty(\epsilon) = \mqty[1-\frac{\epsilon^2}{2} & 0 & \epsilon\\ \epsilon^2 & 1- \frac{\epsilon^2}{2} & -\epsilon\\ -\epsilon & \epsilon & 1-\epsilon^2] + \order{\epsilon^3}
\end{equation}
and
\begin{equation}
\label{chapter18.eqn8b}
R_y\qty(\epsilon) R_x\qty(\epsilon) = \mqty[1-\frac{\epsilon^2}{2} & \epsilon^2 & \epsilon\\ 0 & 1- \frac{\epsilon^2}{2} & -\epsilon\\ -\epsilon & \epsilon & 1-\epsilon^2] + \order{\epsilon^3}
\end{equation}
From (\ref{chapter18.eqn8a}) and (\ref{chapter18.eqn8b}) we have
\begin{equation}
\label{chapter18.eqn9}
R_x(\epsilon) R_y(\epsilon) - R_y(\epsilon)  R_x(\epsilon) = \mqty[0 & -\epsilon^2 & 0 \\ \epsilon^2 & 0 & 0 \\ 0 & 0 & 0] = R_z(\epsilon^2) - 1
\end{equation}
In these calculations all terms higher than $\epsilon^2$ have been ignored. Eq. (\ref{chapter18.eqn9}) leads to the import result that infinitesimal rotations about different axes do commute up to first order. Now, we have
\begin{equation}
\label{chapter18.eqn10}
1 = R_{any}(0)
\end{equation}
where $any$ stands for any rotation axis. Thus Eq. (\ref{chapter18.eqn9}) can be written as
\begin{equation}
\label{chapter18.eqn11}
\comm{R_x(\epsilon)}{R_y(\epsilon)} = R_z(\epsilon^2) - R_{any}(0)
\end{equation}

\begin{align}
\label{chapter18.eqn12}
\comm{R_y(\epsilon)}{R_z(\epsilon)} = R_x(\epsilon^2) - R_{any}(0)\\
\label{chapter18.eqn13}
\comm{R_z(\epsilon)}{R_x(\epsilon)} = R_y(\epsilon^2) - R_{any}(0)
\end{align}

Eqs. (\ref{chapter18.eqn11}) to (\ref{chapter18.eqn13}) are examples of the commutation relations between rotational matrices in three dimensional real space. These commutation relations will be used later to derived the angular momentum commutation relations.



\section{Rotations in Hilbert Space}
%% page 9

Consider a physical system with state vector $\ket{\psi}$ in Hilbert space.
\begin{figure}
	%%%%%%% TODO
	\centering
	\includegraphics[width=0.5\linewidth]{Pictures/not-found.jpg}
	\caption{rotation in Hilbert space}
	\label{chapter18.fig3}
\end{figure}
If the system is now rotated by a certain angle about a certain axis, the state vector changes to $\ket{\psi}_R$. Thus there exists an operator $U(R)$ in Hilbert space which carries the state $\ket{\psi}$ to $\ket{\psi}_R$, i.e.,
\begin{equation}
\label{chapter18.eqn14}
\ket{\psi}_R = U(R) \ket{\psi}
\end{equation}
The operator $U(R)$ is unitary so that normalization of the states remain unaltered.


What we have done is to established a correspondence between a rotation in real three-dimensional space and a unitary operator $U(R)$ in the Hilbert space,
\begin{align*}
R &\longleftrightarrow U(R) \\
\qq{Rotation in 3-space}& \longleftrightarrow \qq{Transformation in Hilbert space}
\end{align*}
Note that $R$ is a $3\times 3$ orthogonal matrix acting on the components of a classical vector in $3$-space while $U(R)$ is a unitary operator acting on the \textit{vectors} of a Hilbert space (ket space). We could also find a matrix representation of the operator $U(R)$ in the Hilbert space by choosing an appropriate set of basis kets. If the number of kets in the basis set is $N$, then the matrix representation of $U(R)$ would be $N\times N$ dimensional.

%% page 11
For example, if we consider a spin-$1/2$ particle with no other degrees of freedom, then $N=2$ and $U(R)$ would be a $2\times 2$ unitary matrix, for a spin-$3/2$ particle with no other degrees of freedom, $N=4$, and $U(R)$ would be a $4\times 4$ unitary matrix.\\


Now, we will construct the unitary operator $U(R)$. To do so, it is advantageous to consider infinitesimal rotations of the physical system. To be specific, suppose the physical system is rotated by an infinitesimal angle $\dd{\phi}$ about the $z$-axis. Therefore
\begin{equation}
R = \hat{z} \dd{\phi}
\end{equation}
We can write $U(R)$ as
\begin{equation}
\label{chapter18.eqn15}
U(\dd{\phi} \hat{z}) = 1 - \frac{J_z}{\hbar} \dd{\phi}
\end{equation}
where $J_z$ is a hermitian operator with dimensions of action (i.e., dimensions of $\hbar$ : $\qq{Energy}\times \qq{Time}$ or $\qq{Position}\times \qq{Momentum}$). At this state $J_z$ is \textbf{not} yet identified with the $z$-component of the total angular momentum operator. This identification will be made later after we derive the commutations of $J_z$ with other generators.\\

In Eq. (\ref{chapter18.eqn15}), $J_z$ is the generator of the unitary operator $U(\dd{\phi} \hat{z})$. The operator $U(\phi \hat{z})$ corresponding to a finite positive rotation $\phi$ about the $z$-axis can be obtained by successively compounding infinitesimal rotations about the same axis. Thus
\begin{align}
U_z(\phi) 
&= \lim\limits_{N \rightarrow \infty} \qty[1 - \frac{\iu J_z}{\hbar} \frac{\phi}{N}]^N \nonumber\\
\label{chapter18.eqn16a}
&= e^{-\iu J_z \phi/\hbar} \\
\label{chapter18.eqn16b}
&= 1 - \frac{\iu J_z \phi}{\hbar} - \frac{J_z^2 \phi^2}{2 \hbar^2}
\end{align}
Similarly we can write
\begin{align}
\label{chapter18.eqn17}
U_x(\phi) &= e^{-\iu J_x \phi/\hbar} \\
\label{chapter18.eqn18}
U_y(\phi) &= e^{-\iu J_y \phi/\hbar}
\end{align}
In general, for a positive rotation by an angle $\phi$ about an axis $\hat{n}$, we have
\begin{align}
\label{chapter18.eqn19}
U_{\hat{n}}(\phi) &= e^{-\iu \vec{J}\cdot\hat{n} \phi/\hbar}
\end{align}
From Eqs. (\ref{chapter18.eqn16a}), (\ref{chapter18.eqn17}) and (\ref{chapter18.eqn18}) we note that the hermitian operators $J_x,\ J_y$ and $J_z$ are the generators of the unitary transformation operaors in the Hilbert space if the system is rotated about the $x$, $y$ and $z$-axis, respectively. We will now show that the three generators $J_x,\ J_y$ and $J_z$ obey the commutation relations of angular momentum operators.



\section{Commutation relations of $J_x,J_y,J_z$}
%% page 14

To obtain the commutation relations between $J_x,\ J_y$ and $J_z$, we need the concept of a group (see Appendix (\ref{appendix4.group})). Now, the rotations form a group. The group multiplication is the application of two rotations successively. To see that the set of all rotations of a physical system form a group, we note that two successive rotations is equivalent to one single rotation. The inverse of a rotation $\phi \hat{n}$ is $-\phi\hat{n}$. The unit element is no rotation at all.\\


To every rotation of the physical system there corresponds a $3\times 3$ orthogonal matrix $R$ acting on the coordinates of a classical vectors, and a unitary operator $U(R)$ acting on the state kets in the Hilbert space. We say that the $3\times 3$ orthogonal matrices $R$ is a representation of the rotation group in the ordinary $3$-space. The set of unitary operator $U(R)$ is also a representation of the rotation group but in the Hilbert space of state vectors. Thus we may postulate $U(R)$ has the same group properties as $R$
\begin{table}
	\begin{tabular}{c|c|c}
		identity & $1\cdot R = R\cdot 1 = R$ & $\mathbb{1} U(R) = U(R) \mathbb{1} = U(R)$ \\ \hline
		closure & $R_1 R_2 = R_3$ & $U(R_1) U(R_2) = U(R_3)$ \\ \hline
		inverse & $R \cdot R^{-1} = 1 = R^{-1}\cdot R$ & 
			\begin{minipage}{5cm}
				\begin{align*}
				U(R) U(R^{-1}) = \mathbb{1} =  U(R^{-1})  U(R)\\
				\therefore  U(R^{-1}) =  U^{-1}(R) 
				\end{align*}
			\end{minipage}\\ \hline
		associativity & 
		\begin{minipage}{5cm}
			\begin{align*}
			R_1 \cdot (R_2 \cdot R_3) \\
			= (R_1 \cdot R_2) \cdot R_3 \\
			= R_1 \cdot R_2 \cdot R_3
			\end{align*}
		\end{minipage}
			 & 
		\begin{minipage}{5cm}
			\begin{align*}
			U(R_1)\qty(U(R_2) U(R_3)) \\
			= \qty(U(R_1)U(R_2)) U(R_3)\\
			= U(R_1)U(R_2) U(R_3)
			\end{align*}
		\end{minipage}
	\end{tabular}
\end{table}

Therefore, to any equation involving the unitary operator $U(R)$. The analogue of Eq. (\ref{chapter18.eqn9}) in Hilbert space is
\begin{equation}
\label{chapter18.eqn20}
U_x(\epsilon) U_y(\epsilon) - U_y(\epsilon) U_x(\epsilon) = U_z(\epsilon^2) - 1
\end{equation}

Eqs. (\ref{chapter18.eqn9}) and (\ref{chapter18.eqn20}) are valid up to second order in $\epsilon$. We therefore expand Eq. (\ref{chapter18.eqn20}) up to second order obtaining
\begin{equation}
\label{chapter18.eqn21}
\begin{split}
\qty(1 - \frac{\iu J_x \epsilon}{\hbar} - \frac{J_x^2 \epsilon^2}{2 \hbar^2}) 
\qty(1 - \frac{\iu J_y \epsilon}{\hbar} - \frac{J_y^2 \epsilon^2}{2 \hbar^2})
-
\qty(1 - \frac{\iu J_y \epsilon}{\hbar} - \frac{J_y^2 \epsilon^2}{2 \hbar^2})
\qty(1 - \frac{\iu J_x \epsilon}{\hbar} - \frac{J_x^2 \epsilon^2}{2 \hbar^2})\\
= \qty(1 - \frac{\iu J_z \epsilon^2}{\hbar}) - 1
\end{split}
\end{equation}
Terms of the order $\epsilon$ automatically drop out. Equating terms of order  $\epsilon^2$ on both sides of Eq. (\ref{chapter18.eqn21}) we obtain

\begin{equation}
\label{chapter18.eqn22}
\comm{J_x}{J_y} = \iu \hbar J_z
\end{equation}

Repeating this kind of arguments with rotations about other axes, we obtain
\begin{align}
\label{chapter18.eqn23}
\comm{J_y}{J_z} = \iu \hbar J_x \\
\label{chapter18.eqn24}
\comm{J_z}{J_x} = \iu \hbar J_y
\end{align}
Equations (\ref{chapter18.eqn22}) to (\ref{chapter18.eqn24}) are the fundamental commutation relation of angular momentum operators.
We can combine Eqs. (\ref{chapter18.eqn22}) to (\ref{chapter18.eqn24}) as
\begin{equation}
	\comm{J_i}{J_j} = \iu \hbar \varepsilon_{i j k} J_k
\end{equation}
Where $\varepsilon_{i j k}$ is the levi-civita symbol.


 We thus conclude that the generators of the unitary transformations of state vectors in Hilbert space corresponding to rotations of the physical system are nothing but the angular momentum operators.




\section{Rotation operator applied to a spinless particle}
%%% page 21
Suppose that a system is rotated by an angle $\phi$ about an axis $\hat{n}$ in the positive sense. The state of the system changes from $\ket{\psi}$ to $\ket{\psi^\prime}$ according to 
\begin{equation}
\label{chapter18.eqn1-spinless}
\ket{\psi^\prime} = U_{\hat{n}}\qty(\phi) \ket{\psi}
\end{equation}
where
\begin{equation}
\label{chapter18.eqn2-spinless}
	 U_{\hat{n}}\qty(\phi) = e^{-i \vec{J}\cdot\hat{n} \phi/\hbar}
\end{equation}
Now, if the system is a spinless particle $(\vec{S}=0)$ then $\vec{J} = \vec{L}$ and the corresponding rotation operator is
\begin{equation}
\label{chapter18.eqn3-spinless}
U_{\hat{n}}\qty(\phi) = e^{-i \vec{L}\cdot\hat{n} \phi/\hbar}
\end{equation}
For simplicity, we will consider rotation about the $z$-axis and see how the wavefunction changes under such a rotation.
\begin{figure}
	%%%%%%% TODO
	\centering
	\includegraphics[width=0.5\linewidth]{Pictures/not-found.jpg}
	\caption{rotation in Hilbert space}
	\label{chapter18.fig4}
\end{figure}
In this case we have
\begin{align}
\ket{\psi^\prime} 
&= U_{z}\qty(\phi) \ket{\psi} \nonumber\\
\label{chapter18.eqn4-spinless}
&= e^{-i L_z \phi/\hbar} \ket{\psi}
\end{align}
In coordinate representation Eq. (\ref{chapter18.eqn4-spinless}) is
\begin{equation}
\label{chapter18.eqn5-spinless}
\braket{\vec{r}}{\psi^\prime} = \mel{\vec{r}}{e^{-i L_z \phi/\hbar}}{\psi}
\end{equation}
%% page 23
Now
\begin{equation*}
\bra{\vec{r}} \hat{L}_z = \bra{\vec{r}} \qty(\hat{\vec{r}} \times \hat{\vec{p}})_z = \qty(\hat{\vec{r}} \times \frac{\hbar}{\iu} \vec{\nabla})_z \bra{\vec{r}}
\end{equation*}
Therefore, Eq. (\ref{chapter18.eqn5-spinless}) can be written as
\begin{equation}
\braket{\vec{r}}{\psi^\prime} = e^{-\frac{\iu}{\hbar} \phi \qty(\vec{r} \times \frac{\hbar}{\iu} \vec{\nabla})_z} \braket{\vec{r}}{\psi}
\end{equation}
For infinitesimal rotations $\dd{\phi}$ we have
\begin{align*}
\psi^\prime\qty(\vec{r}) 
&= \qty[1 - \frac{\iu}{\hbar} \dd{\phi} \qty(\vec{r} \times \frac{\hbar}{\iu} \vec{\nabla})_z] \psi\qty(\vec{r}) \\
&= \qty[1 - \dd{\phi} \qty(x \pdv{}{y} - y \pdv{}{x})] \psi\qty(\vec{r}) \\
&= \psi\qty(\vec{r}) - \dd{\phi} x \pdv{\psi(x,y,z)}{y} + \dd{\phi} y \pdv{\psi(x,y,z)}{x} \\
&= \psi\qty(x + y\dd{\phi}, y - x \dd{\phi}, z)
\end{align*}

Now
\begin{align}
\label{chapter18.eqn1.infinitesimal-3d}
R_z(\dd{\phi}) &= \mqty[1 & -\dd{\phi} & 0 \\ \dd{\phi} & 1 & 0 \\ 0 & 0 & 1] \\
\label{chapter18.eqn2.infinitesimal-3d}
\therefore R^{-1}_z(\dd{\phi}) &= \mqty[1 & \dd{\phi} & 0 \\ -\dd{\phi} & 1 & 0 \\ 0 & 0 & 1] \\
\end{align}
Therefore
\begin{equation*}
R^{-1}_z(\dd{\phi}) \vec{r} = \mqty[1 & \dd{\phi} & 0 \\ -\dd{\phi} & 1 & 0 \\ 0 & 0 & 1]  \mqty[x\\y\\z] = \mqty[x+y\dd{\phi} \\ y - x\dd{\phi} \\ z]
\end{equation*}
Hence
\begin{equation}
\label{chapter18.eqn6-spinless}
\psi^\prime(\vec{r}) = \psi(R^{-1} \vec{r})
\end{equation}
Since $\vec{r}$ is arbitrary in Eq. (\ref{chapter18.eqn6-spinless}), we can rewrite (\ref{chapter18.eqn6-spinless}) in the form
\begin{equation}
\label{chapter18.eqn7-spinless}
\psi^\prime(R \vec{r}) = \psi(\vec{r})
\end{equation}
Eq. (\ref{chapter18.eqn7-spinless}) is intuitively obvious. If the system is rotated, the old wave function at any point $\vec{r}$ must be equal to the new wave function at the rotated point $R\vec{r}$.

We have derived the transformation equation of the wavefunction (Eq. (\ref{chapter18.eqn6-spinless}) or (\ref{chapter18.eqn7-spinless})) of a spinless particle using the expression rotation operator given in Eq. (\ref{chapter18.eqn3-spinless}). We could \textit{work backwards}, i.e., starting from the obvious transformation of the wavefunction under rotation, i.e., Eq. (\ref{chapter18.eqn6-spinless}) or (\ref{chapter18.eqn7-spinless}),  we could work out the expression for the rotation operator in the Hilbert space of spinless particles.


Thus, if the system is rotated by an angle $\phi$ about the $z$-axis, the wave function changes from $\phi(\vec{r})$ to $\psi^\prime(\vec{r})$ in such a manner that
\begin{equation}
\label{chapter18.eqn8-spinless}
	\psi^\prime(\vec{r}) = \psi\qty(R^{-1}_z(\phi) \vec{r})
\end{equation}
We cast Eq. (\ref{chapter18.eqn8-spinless}) in the form
\begin{equation}
\label{chapter18.eqn9-spinless}
	\psi^\prime(\vec{r}) = U_z(\phi) \psi(\vec{r})
\end{equation}
where $U_z(\phi)$ is the rotation operator in the Hilbert space. To express Eq. (\ref{chapter18.eqn8-spinless}) in the form Eq. (\ref{chapter18.eqn9-spinless}), it is convenient to consider infinitesimal rotations in Eq. (\ref{chapter18.eqn2.infinitesimal-3d}) and Eq. (\ref{chapter18.eqn8-spinless}) is

\begin{align*}
\psi^\prime(\vec{r}) 
&= \psi(x + y\dd{\phi}, y - x\dd{\phi}, z) \\
&= \psi\qty(\vec{r}) - \dd{\phi} x \pdv{\psi(x,y,z)}{y} + \dd{\phi} y \pdv{\psi(x,y,z)}{x} \\
&= \qty[1 - \dd{\phi} \qty(x \pdv{}{y} - y \pdv{}{x})] \psi\qty(\vec{r}) \\
&= \qty[1 - \frac{\iu}{\hbar} \dd{\phi} \qty(\vec{r} \times \frac{\hbar}{\iu} \vec{\nabla})_z] \psi\qty(\vec{r}) \\
&= \qty[1 - \frac{\iu}{\hbar} \dd{\phi} \qty(\vec{r} \times \vec{p})_z] \psi(x,y,z) \\
&= \qty[1 - \frac{\iu}{\hbar} \dd{\phi} \hat{L}_z] \psi(x,y,z)
\end{align*}
For a finite rotation we have
\begin{equation}
\label{chapter18.eqn11-spinless}
\psi^\prime(\vec{r}) = e^{-\frac{\iu}{\hbar} \phi \hat{L}_z}  \psi(\vec{r})
\end{equation}
Hence, the rotation operator in Hilbert space  of a spin particle is given by Eq. (\ref{chapter18.eqn3-spinless}).



\section{Rotation operator in spin space}
\label{chapter28.sect.rotation-in-spin-space}
%% page 27
Consider a particle with only spin degrees of freedom, i.e., the particle's spatial degrees of freedom are suppressed. In this case $\vec{L} = 0$, and, therefore, $\vec{J}=\vec{S}$ where $\vec{S}$ is the spin angular momentum operator of the particle. The rotation operator in the spin Hilbert space is then 
\begin{equation}
\label{chapter18.eqn13-spin}
U^{(s)}_{\hat{n}}\qty(\phi) = e^{- \frac{\iu}{\hbar} \vec{S}\cdot\hat{n} \phi}
\end{equation}

Consider now a rotation by a finite angle $\phi$ about the $z$-axis. If the ket of a spin-$1/2$ particle before rotation in $\ket{\alpha}$, the ket after rotation in given by
\begin{equation}
\label{chapter18.eqn14-spin}
\ket{\alpha}_R = U^{(s)}_{z}\qty(\phi) \ket{\alpha} = e^{- \frac{\iu}{\hbar} S_z \phi} \ket{\alpha}
\end{equation}

To show that the operator (\ref{chapter18.eqn13-spin}) really rotates the physical system, let us look at its effects on $\expval{S_x}$. Under the rotation about the $z$-axis, this expectation value changes to
\begin{equation}
\expval{S_x} \rightarrow \expval{S_x}^\prime = \tensor[_R]{\mel{\alpha}{S_x}{\alpha}}{_R} = \mel{\alpha}{U^{\dagger}_z(\phi) S_x U_z(\phi)}{\alpha}
\end{equation}
We must therefore compute $U^{\dagger}_z(\phi) S_x U_z(\phi)$. Using the identity
\begin{equation}
e^{A} B e^{-A} = B + \comm{A}{B} + \frac{1}{2 !} \comm{A}{\comm{A}{B}} + \ldots
\end{equation}

we have
\begin{align*}
U^{\dagger}_z(\phi) S_x U_z(\phi) 
&= e^{ \frac{\iu}{\hbar} S_z \phi} S_x e^{- \frac{\iu}{\hbar} S_z \phi} \\
&= S_x + \qty(\frac{\iu \phi}{\hbar}) \comm{S_z}{S_x} + \frac{1}{2 !} \qty(\frac{\iu \phi}{\hbar})^2 \comm{S_z}{\comm{S_z}{S_x}} + \frac{1 }{3 !} \qty(\frac{\iu \phi}{\hbar})^3 \comm{S_z}{\comm{S_z}{\comm{S_z}{S_x}}}  + \ldots \\
&= S_x + \qty(\frac{\iu \phi}{\hbar}) \iu \hbar S_y + \frac{1}{2 !} \qty(\frac{\iu \phi}{\hbar})^2 \hbar^2 S_x + \frac{1}{3 !} \qty(\frac{\iu \phi}{\hbar})^3 \iu \hbar^3 S_y + \ldots \\
&= S_x \qty(1 - \frac{\phi^2}{2 !} + \ldots)  - S_y \qty(\phi - \frac{\phi^3}{3 !} + \ldots) \\
&= S_x \cos\phi - S_y \sin\phi
\end{align*}

Thus
\begin{equation}
\expval{S_x}^\prime = \expval{S_x} \cos\phi - \expval{S_y}\sin\phi
\end{equation}
Similarly
\begin{align}
\label{chapter18.eqn15-spin}
\expval{S_y}^\prime = \expval{S_x} \sin\phi + \expval{S_y}\cos\phi \\
\expval{S_z}^\prime = \expval{S_z}
\end{align}
This shows that the \textit{rotation} operator (\ref{chapter18.eqn13-spin}) when applied to the state ket does rotate the expectation value of $\vec{S}$ around the $z$-axis by an angular $\phi$. In other words, the expectation values of the spin operator behaves as though it ware a classical vector!

Up to now we have dealt with the expectation values of the spin operator in the rotated and unrotated spin states. Now let us look at the effect of the rotation $U_z(\phi)$ on a general spin state $\ket{\alpha}$. For a spin $1/2$ particle we write
\begin{equation}
\label{chapter18.eqn16-spin}
\ket{\alpha} = \ket{+} \braket{+}{\alpha} + \ket{-} \braket{-}{\alpha}
\end{equation}
where $\ket{\pm}$ are the eigenstates of $S_z$ with eigenvalues $\pm \frac{1}{2} \hbar$. Now
\begin{equation}
e^{-\iu S_z \phi/\hbar} \ket{\alpha} = e^{-\iu \phi/2} \ket{+} \braket{+}{\alpha} + e^{\iu \phi/2} \ket{-} \braket{-}{\alpha}
\end{equation}
The appearance of the half angle $\phi/2$ here has an extremely interesting consequence. Let us consider a rotation by an angle $2\pi$. We then have
\begin{equation}
\label{chapter18.eqn17-spin}
\ket{\phi} \substack{\rightarrow \\ R_z(2 \pi)} \ket{\alpha}_R = - \ket{\alpha}
\end{equation}
So that the ket for the $360^\circ$ rotated state differs from the original ket by a minus sign. We would require a $720^\circ$ or $\phi=4 \pi$ rotation to get back to the same state with plus sign. Notice that this minus sign disappears from the expectation of $\vec{S}$ because $\vec{S}$ is sandwiched between $\bra{\alpha}$ and $\ket{\alpha}$, both of which change sign. (Will this minus sign be observable ? see Sakurai)



\subsection{Matrix Representation}
%% page 32
The rotation operator in spin space is given by Eq. (\ref{chapter18.eqn13-spin}). For a spin-$1/2$ particle we can use the eigenkets $\ket{\pm}$ of $S_z$ as basis. Then $\vec{S}$ and $U^{(s)}_{\hat{n}}\qty(\phi)$ are expressed as $2\times 2$ matrices. We then have
\begin{equation}
\vec{S} \doteq \frac{1}{2} \hbar \vec{\sigma}
\end{equation}
Therefore
\begin{equation}
e^{-\frac{\iu}{\hbar} \vec{S}\cdot \hat{n} \phi} \doteq e^{-\iu \frac{\vec{\sigma}\cdot \hat{n}}{2} \phi}
\end{equation}

The $2\times 2$ matrices $\exp(-\iu \frac{\vec{\sigma}\cdot \hat{n}}{2} \phi)$ act on the two component spinor $\chi$ where
\begin{equation}
\chi = \mqty[\braket{+}{\alpha} \\ \braket{-}{\alpha}]
\end{equation}

Now
\begin{align*}
U_n^{(s)}\qty(\phi) 
&\doteq e^{-\iu \frac{\phi}{2} \vec{\sigma}\cdot\hat{n}} \\
&= 1 + \qty(-\frac{\iu\phi}{2}) \vec{\sigma}\cdot\hat{n} 
+ \frac{1}{2 !} \qty(-\frac{\iu\phi}{2})^2 \qty(\vec{\sigma}\cdot\hat{n})^2 + \ldots 
+ \frac{1}{m !} \qty(-\frac{\iu\phi}{2})^m \qty(\vec{\sigma}\cdot\hat{n})^m + \ldots
\end{align*}
Applying the identity
\begin{equation}
\qty(\vec{\sigma}\cdot\hat{n})^2 = \hat{n}\cdot\hat{n} 1_{2\times 2} + \iu \vec{\sigma} \cdot \qty(\hat{n} \times \hat{n}) = 1_{2\times 2}
\end{equation}
Which leads to
\begin{equation}
\qty(\vec{\sigma}\cdot\hat{n})^m = 
\begin{cases}
1 \qq{if $m$ is even}\\
0 \qq{if $m$ is odd}
\end{cases}
\end{equation}

We get
\begin{align}
U_n^{(s)}\qty(\phi) 
&= \qty[1 - \frac{1}{2!} \qty(\frac{\phi}{2})^2 + \frac{1}{4 !} \qty(\frac{\phi}{2})^4 - \ldots] 1_{2\times 2}
- \iu \vec{\sigma} \cdot \hat{n} \qty[\phi - \frac{1}{3 !} \qty(\frac{\phi}{2})^3 + \frac{1}{5 !} \qty(\frac{\phi}{2})^5 - \ldots] \nonumber\\
&= \cos(\frac{\phi}{2}) 1_{2\times 2} - \iu \vec{\sigma} \cdot \hat{n} \sin(\frac{\phi}{2}) \qq{spin 1/2 particles}
\end{align}
Next
\begin{align*}
\vec{\sigma} \cdot \hat{n} 
&= \sigma_x n_x + \sigma_y n_y + \sigma_z n_z \\
&= \mqty[0 & 1 \\ 1 & 0] n_x + \mqty[0 & -\iu \\ \iu & 0] n_y + \mqty[1 & 0 \\ 0 & -1] n_z \\
&= \mqty[n_z & n_x - \iu n_y \\ n_x + \iu n_y & -n_z]
\end{align*}
Therefore
\begin{equation}
e^{-\iu \vec{\sigma} \cdot \hat{n} \phi/2} = \mqty[\cos(\phi/2) - \iu n_z \sin(\phi/2) & \qty(- \iu n_x - n_y) \sin(\phi/2) \\
\qty(-\iu n_x + n_y) \sin(\phi/2) & \cos(\phi/2) + \iu n_z \sin(\phi/2)]
\end{equation}
Note that
\begin{equation}
\eval{e^{-\iu \vec{\sigma}\cdot \hat{n}\phi/2}}_{\phi=2 \pi} = -1 \qq{for eny} \hat{n}
\end{equation}
Thus
\begin{align}
\chi \quad \substack{\longrightarrow \\ 2\pi \qq{rotation}} \quad - \chi
\end{align}



\subsection{Eigen Spinors of $\vec{\sigma} \cdot \hat{n}$}
%% page 36
As an instructive application of the rotation matrix, let us see how we can construct eigenspinors of $\vec{\sigma}\cdot \hat{n}$ with eigenvalues $\pm 1$ where $\hat{n}$ is some unit vector along some specified direction. Out objective is to construct $\chi_{\pm}$ satisfying $\vec{\sigma} \cdot \hat{n} \chi_{\pm} = \pm \chi_{\pm}$
Actually, $\chi_{\pm}$ can be obtained as a  straight eigenvalue problem, but here we present an alternative method using the rotation matrix.\\


Let the polar and azimuthal angle of $\hat{n}$ be $\theta$ and $\phi$, respectively. Let us start with $\mqty[1 \\ 0]$, the two-component spinor that represents the spin up state. Given this, we first rotate about the $y$-axis by an angle $\theta$, then rotate by an angle $\phi$ about the $z$-axis, as shown in Fig. (\ref{chapter18.fig5})


\begin{figure}
	%%%%%%% TODO
	\centering
	\includegraphics[width=0.5\linewidth]{Pictures/not-found.jpg}
	\caption{.}
	\label{chapter18.fig5}
\end{figure}

The described spin state is then obtained.
\begin{align*}
\chi_{+} 
&= e^{-\iu \sigma_z \phi/2} e^{-\iu \sigma_y \theta/2} \mqty[1 \\ 0] \\
&=  \qty[\cos(\frac{\phi}{2}) - \iu \sigma_z \sin(\frac{\phi}{2})] \qty[\cos(\frac{\theta}{2}) - \iu \sigma_y \sin(\frac{\theta}{2})] \mqty[1 \\ 0] \\
&= \mqty[\cos(\phi/2) - \iu \sin(\phi/2) & 0 \\ 0 & \cos(\phi/2) + \iu \sin(\phi/2)] \mqty[\cos(\phi/2) & -\sin(\phi/2) \\ \sin(\phi/2) & \cos(\phi/2)] \mqty[1 \\ 0] \\
&= \mqty[e^{-\iu \phi/2} & 0 \\ 0 & e^{\iu \phi/2}]
\mqty[\cos(\theta/2) \\ \sin(\theta/2)] \\
&= \mqty[e^{-\iu \phi/2}\cos(\theta/2) \\ e^{\iu \phi/2}\sin(\theta/2)]
\end{align*}
To get $\chi_{-}$ we could apply the same sequence of relations to $\mqty[0 \\1]$, or, we could get $\chi_{-}$ from  $\chi_{+}$ simply by noting that $-\hat{n}$ has polar coordinates $\qty(\pi - \theta, \pi + \phi)$, so that

\begin{equation}
\chi_{-} = \mqty[- \iu e^{-\iu \phi/2}\sin(\theta/2) \\ \iu e^{\iu \phi/2}\cos(\theta/2)]
\end{equation}

Disregarding the overall phase factor $\iu$, we cann write
\begin{equation}
\chi_{-} = \mqty[- e^{-\iu \phi/2}\sin(\theta/2) \\ e^{\iu \phi/2}\cos(\theta/2)]
\end{equation}



\subsection{Rotation of two component spinors}
%% page 39
We are now prepared to study the global behaviour of spin $1/2$ particle under rotation. That is, we shall now take into account both the internal and external degrees of freedom of the particle.


Consider a spin $1/2$ particle whose state is represented by $\ket{\psi}$ in the state space (Hilbert space) $\mathcal{H} = \mathcal{H}_{r}\otimes \mathcal{H}_{s}$. The ket can be represented by the spinors $\qty[\psi]\qty(\vec{r})$ having the components
\begin{equation}
\psi_\epsilon\qty(\vec{r}) = \braket{\vec{r}, \epsilon}{\psi}
\end{equation} 

where $\epsilon = \pm$ represents the two degrees of freedom in the spin space. Then
\begin{equation}
\ket{\psi} \doteq \qty[\psi] \qty(\vec{r}) = \mqty[\psi_{+}\qty(\vec{r}) \\ \psi_{-}\qty(\vec{r})]
\end{equation}
If we perform an arbitrary rotation on the particle, its state vector changes to $\ket{\psi^\prime}$ where
\begin{equation}
\ket{\psi^\prime} = U \ket{\psi}
\end{equation}

with
\begin{align*}
 U 
 &= e^{-\frac{\iu}{\hbar} \vec{J}\cdot \hat{n} \phi} = e^{-\frac{\iu}{\hbar} \qty(\vec{L} + \vec{S})\cdot \hat{n} \phi} \\
 &= e^{-\frac{\iu}{\hbar} \vec{L}\cdot \hat{n} \phi} e^{-\frac{\iu}{\hbar} \vec{S}\cdot \hat{n} \phi} \\
 &= U^{(r)}\qty(\phi) U^{(s)}\qty(\phi)
\end{align*}
Where
\begin{align}
U^{(r)}\qty(\phi) &= e^{-\frac{\iu}{\hbar} \vec{L}\cdot \hat{n} \phi} \\
U^{(s)}\qty(\phi) &= e^{-\frac{\iu}{\hbar} \vec{S}\cdot \hat{n} \phi} 
\end{align}
Note that $\comm{\vec{L}}{\vec{S}} = 0$.\\

The rotation operator $U^{(r)}\qty(\phi)$ acts in the space $\mathcal{H}_{r}$ with basis $\{\ket{\vec{r}}\}$, i.e., the space of external variables, and the operator $U^{(s)}\qty(\phi)$ acts on the spin space $\mathcal{H}_s$ with basis $\{\ket{\pm}\}$.


%%% page 41
We write the spinor corresponding to the transformed state as
\begin{equation}
\qty[\psi^\prime] \qty(\vec{r}) = \mqty[\psi_{+}^\prime\qty(\vec{r}) \\ \psi_{-}^\prime\qty(\vec{r})]
\end{equation}
We will now derive a formula which connects the spinor $\qty[\psi^\prime] \qty(\vec{r})$ to the spinor $\qty[\psi] \qty(\vec{r})$.
First, let us write the components of $\psi_\epsilon^\prime\qty(\vec{r})$ of the spinor $\qty[\psi^\prime] \qty(\vec{r})$ as
\begin{equation}
 \psi_\epsilon^\prime\qty(\vec{r}) = \braket{\vec{r}\epsilon}{\psi^\prime} = \mel{\vec{r}\epsilon}{U}{\psi}
\end{equation}
Using the closure (i.e., completeness) relation
\begin{equation}
\sum_{\epsilon^\prime = \pm} \int \dd[3]{r^\prime} \ketbra{\vec{r}^\prime\epsilon^\prime}{\vec{r}^\prime\epsilon^\prime} = \hat{1}
\end{equation}
We obtain

\begin{align*}
\psi_\epsilon^\prime\qty(\vec{r}) =\sum_{\epsilon^\prime = \pm} \int \dd[3]{r^\prime} \mel{\vec{r}\epsilon}{U}{\vec{r^\prime}\epsilon^\prime} \braket{\vec{r^\prime}\epsilon^\prime}{\psi}
\end{align*}

Now, since the basis vectors $\{\ket{\vec{r}\epsilon} = \ket{\vec{r}} \otimes \ket{\epsilon}\}$ are tensor products, the matrix elements of the operator $U$ in this basis can be decomposed in the following manner:
\begin{equation}
\mel{\vec{r}\epsilon}{U}{\vec{r^\prime}\epsilon^\prime} = 
\mel{\vec{r}}{U^{(r)}}{\vec{r^\prime}}
\mel{\epsilon}{U^{(s)}}{\epsilon^\prime}
\end{equation}

Now, 
\begin{align*}
U^{(r)}\ket{\vec{r}} &= \ket{R\vec{r}} \\
\therefore \bra{\vec{r}} U^{(r)^\dagger} &= \bra{R \vec{r}} \\
\qq{or,} \bra{\vec{r}} = \bra{R r} U^{(r)} \qq{since $U^{(r)}$ is unitary}
\end{align*}
Since $\vec{r}$ is arbitrary
\begin{equation}
\bra{R^{-1} \vec{r}} = \bra{\vec{r}}  U^{(r)}
\end{equation}
Therefore we have
\begin{equation}
\mel{\vec{r}}{U^{(r)}}{\vec{r^\prime}} = \braket{R^{-1}\vec{r}}{\vec{r^\prime}} = \delta\qty(\vec{r^\prime} - R^{-1}\vec{r})
\end{equation}

Next, let us call
\begin{equation}
\mel{\epsilon}{U^{(s)}}{\epsilon^\prime} = U^{(1/2)}_{\epsilon\epsilon^\prime}
\end{equation}
Then
\begin{equation}
\mel{\vec{r}\epsilon}{U}{\vec{r^\prime}\epsilon^\prime} = \delta\qty(\vec{r^\prime} - R^{-1}\vec{r}) U^{(1/2)}_{\epsilon\epsilon^\prime}
\end{equation}
and the transformed spinor is
\begin{equation}
\psi^\prime_{\epsilon}\qty(\vec{r}) = \sum_{\epsilon^\prime = \pm} U^{(1/2)}_{\epsilon\epsilon^\prime} \psi_{\epsilon^\prime}\qty(R^{-1} \vec{r})
\end{equation}
Explicitly
\begin{equation}
\mqty[\psi^\prime_{+}\qty(\vec{r}) \\ \psi^\prime_{-}\qty(\vec{r})] = \mqty[U^{(1/2)}_{++} & U^{(1/2)}_{+-} \\ U^{(1/2)}_{-+} & U^{(1/2)}_{--}] \mqty[\psi_{+}\qty(R^{-1} \vec{r}) \\ \psi_{-}\qty(R^{-1} \vec{r}) ]
\end{equation}
Thus we obtain the following result : each component of the new spinor $\qty[\psi^\prime]$ at the point $\vec{r}$ is a linear combination of the two components of the original spinor $\qty[\psi]$ at the point $R^{-1}\vec{r}$. The coefficients of these linear combinations are the elements of the $2\times 2$ matrix which represents $U^{(s)}$ in the $\{\ket{\pm}\}$ basis.




\subsection{spin Precession}
Sakurai\\

%% page 44
Consider a spin $1/2$ particle with its space degrees of freedom suppressed. If the particle is subjected to an external magnetic field, its Hamiltonian can be written as
\begin{equation}
\label{chapter18.eqn1-spin-precession}
H = -\vec{\mu} \cdot \vec{B}
\end{equation}
where $\vec{\mu}$ is the magnetic moment operator of the spin $1/2$ particle, say electron. Since the electron has only spin degrees of freedom, we can write
\begin{equation}
\label{chapter18.eqn2-spin-precession}
\vec{\mu} = g_s \frac{q_e}{2 m_e} \vec{S}
\end{equation}
where $q=-e$ is the charge of the electron and $g_s=2$ is the spin gyromagnetic ratio of the electron. With $g_s=2$, the Hamiltonian can be written as
\begin{equation}
\label{chapter18.eqn3-spin-precession}
H = + \frac{e}{m_e} \vec{S}\cdot\vec{B}
\end{equation}

If the direction of $\vec{B}$ is taken as the $\hat{z}$ axis, i.e., if $\vec{B} = \hat{z} B$, the Hamiltonian becomes
\begin{equation}
\label{chapter18.eqn5-spin-precession}
H =  \frac{e B}{m_e} S_z = \omega S_z
\end{equation}

by defining
\begin{equation}
\label{chapter18.eqn4-spin-precession}
\omega = \frac{e B}{m_e}
\end{equation}
Now suppose that the electron is in an initial spin-state $\ket{\alpha, t=0}$. We take what is the state of the system at a later time $t$. We can find this state by applying the time evolution operator to the initial state
\begin{equation}
\label{chapter18.eqn6-spin-precession}
\ket{\alpha, t} = T(t, 0)\ket{\alpha, t=0}
\end{equation}
where $T(t,0)$ is the time evolution operator given by
\begin{equation}
\label{chapter18.eqn7-spin-precession}
T(t, 0) = e^{-\iu H t/\hbar} \qq{assuming $H$ is time independent}
\end{equation}
In our case $H$ is given by Eq. (\ref{chapter18.eqn5-spin-precession}) so that
\begin{equation}
\label{chapter18.eqn8-spin-precession}
T(t, 0) = e^{-\iu \omega t S_z / \hbar}
\end{equation}
Applying Eq. (\ref{chapter18.eqn8-spin-precession}) in Eq. (\ref{chapter18.eqn6-spin-precession}) we obtain
\begin{equation}
\label{chapter18.eqn9-spin-precession}
\ket{\alpha,t} = e^{-\iu \omega t S_z / \hbar} \ket{\alpha,t=0}
\end{equation}

We notice that the time-evolution operator (Eq. (\ref{chapter18.eqn8-spin-precession})) is nothing but the rotation operator in the Hilbert space corresponding to the rotation of the physical system by an angle $\omega t$ about $z$-axis, i.e., about the direction of the applied magnetic field. This means that the spin precesses around the magnetic field as an axis. To see this let us calculate the expectation value of $S_x, S_y$ and $S_z$ at time $t$.

\begin{align*}
\expval{S_x}_t &= \mel{\alpha t}{S_x}{\alpha t} = \mel{\alpha t=0}{e^{\iu\omega t S_z / \hbar} S_x e^{-\iu\omega t S_z / \hbar}}{\alpha t=0} \\
\expval{S_y}_t &= \mel{\alpha t}{S_y}{\alpha t} = \mel{\alpha t=0}{e^{\iu\omega t S_z / \hbar} S_y e^{-\iu\omega t S_z / \hbar}}{\alpha t=0} \\
\expval{S_z}_t &= \mel{\alpha t}{S_z}{\alpha t} = \mel{\alpha t=0}{e^{\iu\omega t S_z / \hbar} S_z e^{-\iu\omega t S_z / \hbar}}{\alpha t=0}
\end{align*}
Now, we can easily show the following relations (from Section (\ref{chapter28.sect.rotation-in-spin-space}))
\begin{align*}
e^{\iu\omega t S_z / \hbar} S_x e^{-\iu\omega t S_z / \hbar} &= S_x\cos(\omega t) - S_y\sin(\omega t)\\
e^{\iu\omega t S_z / \hbar} S_y e^{-\iu\omega t S_z / \hbar} &= S_x\sin(\omega t) + S_y\cos(\omega t) \\
e^{\iu\omega t S_z / \hbar} S_z e^{-\iu\omega t S_z / \hbar} &= S_z
\end{align*}
Thus
\begin{align}
\label{chapter18.eqn10-spin-precession}
\begin{split}
\expval{S_x}_t &= \expval{S_x}_{t=0}\cos(\omega t) - \expval{S_y}_{t=0}\sin(\omega t) \\
\expval{S_y}_t &= \expval{S_x}_{t=0}\sin(\omega t) + \expval{S_y}_{t=0}\cos(\omega t) \\
\expval{S_z}_t &= \expval{S_z}_{t=0}
\end{split}
\end{align}
The set of equations (\ref{chapter18.eqn10-spin-precession}) shows clearly that the spin of the electron precesses around the $z$-axis, i.e., around the direction of $\vec{B}$ with an angular frequency $\omega$, i.e., witha time period $\tau = \frac{2\pi}{\omega}$ see Fig. (\ref{chapter18.fig6})
\begin{figure}
	%%%%%%% TODO
	\centering
	\includegraphics[width=0.5\linewidth]{Pictures/not-found.jpg}
	\caption{Precession of electron  around the $z$-axis, i.e., around the direction of $\vec{B}$ with an angular frequency $\omega$}
	\label{chapter18.fig6}
\end{figure}
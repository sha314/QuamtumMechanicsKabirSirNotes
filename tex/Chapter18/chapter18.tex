%% writing by shahnoor

\chapter{sheet-18 : Rotations and Angular Momentum}

\ifpdf
\graphicspath{{Chapter23/figs/}}
\else
\graphicspath{{Chapter23/figs/}}
\fi

reference : Sakurai, Cohen-Tanondgi\\

A rotation of a physical system is specified by the angle of rotation and the axis of rotation. The rotation can be either positive or negative. If a right-handed screw turned in the direction of rotation proceeds along the positive direction of the axis, the rotation is said to be positive. Thus, for examp, $\phi\hat{z}$ denotes a positive rotation by an angle $\phi$ about the z-axis Fig. (\ref{chapter18.fig1})

\begin{figure}
	%%%%%%% TODO
	\centering
	\includegraphics[width=0.5\linewidth]{Pictures/not-found.jpg}
	\caption{A positive rotation of the physical system by an angle $\phi$ about the $z$-axis.}
	\label{chapter18.fig1}
\end{figure}
In our subsequent discussions we will consider active rotation\index{active and passive rotation}, i.e., rotation of the physical system rather than the rotation of the coordinates.\\

Now, finite rotations about different axes do not commute, i.e., the change in the coordinate of a point in the physical system depends on the order the rotations are performed. To work out quantitatively the extent in which rotations about different axes fail to commute, we have to construct the matrices corresponding to rotations in the three-dimensional real space $(x,y,z)$.\\

In each rotation a vector $\vec{r}$ with coordinates $(x, y, z)$ changes to a new vector $\vec{r^\prime}$ with coordinate $(x^\prime,y^\prime,z^\prime)$. The matrix connecting $(x^\prime,y^\prime,z^\prime)$ with $(x,y,z)$ is the matrix corresponding to the rotation.

%% page 3
Thus
\begin{equation}
\label{chapter18.eqn1}
\mqty[x^\prime\\ y^\prime \\ z^\prime] = \mqty[R]  \mqty[x\\ y \\ z]
\end{equation}
Where $R$ is the $3\times3$ square matrix corresponding to the rotation. In a rotation, the length of the vector $\vec{r}$ remains unchanged, i.e.,
\begin{equation}
\norm{\vec{r^\prime}} = \sqrt{{x^{\prime}}^2 + {y^{\prime}}^2 + {z^{\prime}}^2} = \sqrt{x^2 + y^2 + z^2} =\norm{\vec{r}}
\end{equation}
Therefore, the matrix $R$ must be orthogonal, i.e.,
\begin{equation}
R^T R = R R^T = \mathcal{1}
\end{equation}
Next, we will construct explicitly the rotation matrices $R$ in the three-dimensional space corresponding to positive rotations about the $x$-axis, $y$-axis and $z$-axis.

First, consider a finite rotation by an angle $\phi$ in a positive sense about the $z$-axis (Fig. \ref{chapter18.fig2})
\begin{figure}
	%%%%%%% TODO
	\centering
	\includegraphics[width=0.5\linewidth]{Pictures/not-found.jpg}
	\caption{A positive rotation by an angle $\phi$ about the $z$-axis.}
	\label{chapter18.fig2}
\end{figure}
We have
\begin{align*}
x^\prime = \cos\phi x - \sin\phi y \\
y^\prime = \sin\phi x + \cos\phi y
z^\prime = z
\end{align*}
In matrix form
\begin{equation*}
	\mqty[x^\prime\\ y^\prime \\ z^\prime] = \mqty[\cos\phi & - \sin\phi& 0 \\
		\sin\phi & \cos\phi & 0 \\
		0& 0& 1]  
	\mqty[x\\ y \\ z]
\end{equation*}
Thus, the rotation matrix corresponding to a positive rotation by an angle $\phi$ about the $z$-axis is 
\begin{equation}
\label{chapter18.eqn4}
R_z\qty(\phi) = \mqty[\cos\phi & - \sin\phi& 0 \\
\sin\phi & \cos\phi& 0 \\
0& 0& 1]
\end{equation}
Next, consider a rotation about the $x$-axis. The corresponding matrix is
\begin{equation}
\label{chapter18.eqn5}
R_x\qty(\phi) = \mqty[1& 0& 0 \\
0& \cos\phi & - \sin\phi\\
0& \sin\phi & \cos\phi]
\end{equation}

Similarly
\begin{equation}
\label{chapter18.eqn6}
R_y\qty(\phi) = \mqty[\cos\phi& 0& \sin\phi \\
0& 1& 0 \\
-\sin\phi& 0 &\cos\phi]
\end{equation}

%% page 6
For infinitesimal rotation, i.e., $\phi=\epsilon$, the rotation matrices, up to second order in $\epsilon$ are
\begin{align}
\label{chapter18.eqn7}
\begin{split}
R_x\qty(\epsilon) = \mqty[1 & 0 & 0 \\ 0 & 1-\frac{\epsilon^2}{2} &- \epsilon \\ 0 & \epsilon & 1 - \frac{\epsilon^2}{2}] \\
R_y\qty(\epsilon) = \mqty[1 - \frac{\epsilon^2}{2} & 0 & \epsilon \\ 0 & 1 & 0 \\ -\epsilon & 0 & 1 - \frac{\epsilon^2}{2}] \\
R_z\qty(\epsilon) = \mqty[1 - \frac{\epsilon^2}{2} & -\epsilon & 0 \\ \epsilon & 1 - \frac{\epsilon^2}{2} & 0 \\ 0 & 0 & 1]
\end{split}
\end{align}
Now the multiplication leads to (up to $\order{\epsilon^2}$)
\begin{equation}
\label{chapter18.eqn8a}
R_x\qty(\epsilon) R_y\qty(\epsilon) = \mqty[1-\frac{\epsilon^2}{2} & 0 & \epsilon\\ \epsilon^2 & 1- \frac{\epsilon^2}{2} & -\epsilon\\ -\epsilon & \epsilon & 1-\epsilon^2] + \order{\epsilon^3}
\end{equation}
and
\begin{equation}
\label{chapter18.eqn8b}
R_y\qty(\epsilon) R_x\qty(\epsilon) = \mqty[1-\frac{\epsilon^2}{2} & \epsilon^2 & \epsilon\\ 0 & 1- \frac{\epsilon^2}{2} & -\epsilon\\ -\epsilon & \epsilon & 1-\epsilon^2] + \order{\epsilon^3}
\end{equation}
From (\ref{chapter18.eqn8a}) and (\ref{chapter18.eqn8b}) we have
\begin{equation}
\label{chapter18.eqn9}
R_x(\epsilon) R_y(\epsilon) - R_y(\epsilon)  R_x(\epsilon) = \mqty[0 & -\epsilon^2 & 0 \\ \epsilon^2 & 0 & 0 \\ 0 & 0 & 0] = R_z(\epsilon^2) - 1
\end{equation}
In these calculations all terms higher than $\epsilon^2$ have been ignored. Eq. (\ref{chapter18.eqn9}) leads to the import result that infinitesimal rotations about different axes do commute up to first order. Now, we have
\begin{equation}
\label{chapter18.eqn10}
1 = R_{any}(0)
\end{equation}
where $any$ stands for any rotation axis. Thus Eq. (\ref{chapter18.eqn9}) can be written as
\begin{equation}
\label{chapter18.eqn11}
\comm{R_x(\epsilon)}{R_y(\epsilon)} = R_z(\epsilon^2) - R_{any}(0)
\end{equation}

\begin{align}
\label{chapter18.eqn12}
\comm{R_y(\epsilon)}{R_z(\epsilon)} = R_x(\epsilon^2) - R_{any}(0)\\
\label{chapter18.eqn13}
\comm{R_z(\epsilon)}{R_x(\epsilon)} = R_y(\epsilon^2) - R_{any}(0)
\end{align}

Eqs. (\ref{chapter18.eqn11}) to (\ref{chapter18.eqn13}) are examples of the commutation relations between rotational matrices in three dimensional real space. These commutation relations will be used later to derived the angular momentum commutation relations.



\section{Rotations in Hilbert Space}
%% page 9

Consider a physical system with state vector $\ket{\psi}$ in Hilbert space.
\begin{figure}
	%%%%%%% TODO
	\centering
	\includegraphics[width=0.5\linewidth]{Pictures/not-found.jpg}
	\caption{rotation in Hilbert space}
	\label{chapter18.fig3}
\end{figure}
If the system is now rotated by a certain angle about a certain axis, the state vector changes to $\ket{\psi}_R$. Thus there exists an operator $U(R)$ in Hilbert space which carries the state $\ket{\psi}$ to $\ket{\psi}_R$, i.e.,
\begin{equation}
\label{chapter18.eqn14}
\ket{\psi}_R = U(R) \ket{\psi}
\end{equation}
The operator $U(R)$ is unitary so that normalization of the states remain unaltered.


What we have done is to established a correspondence between a rotation in real three-dimensional space and a unitary operator $U(R)$ in the Hilbert space,
\begin{align*}
R &\longleftrightarrow U(R) \\
\qq{Rotation in 3-space}& \longleftrightarrow \qq{Transformation in Hilbert space}
\end{align*}
Note that $R$ is a $3\times 3$ orthogonal matrix acting on the components of a classical vector in $3$-space while $U(R)$ is a unitary operator acting on the \textit{vectors} of a Hilbert space (ket space). We could also find a matrix representation of the operator $U(R)$ in the Hilbert space by choosing an appropriate set of basis kets. If the number of kets in the basis set is $N$, then the matrix representation of $U(R)$ would be $N\times N$ dimensional.

%% page 11
For example, if we consider a spin-$1/2$ particle with no other degrees of freedom, then $N=2$ and $U(R)$ would be a $2\times 2$ unitary matrix, for a spin-$3/2$ particle with no other degrees of freedom, $N=4$, and $U(R)$ would be a $4\times 4$ unitary matrix.\\


Now, we will construct the unitary operator $U(R)$. To do so, it is advantageous to consider infinitesimal rotations of the physical system. To be specific, suppose the physical system is rotated by an infinitesimal angle $\dd{\phi}$ about the $z$-axis. Therefore
\begin{equation}
R = \hat{z} \dd{\phi}
\end{equation}
We can write $U(R)$ as
\begin{equation}
\label{chapter18.eqn15}
U(\dd{\phi} \hat{z}) = 1 - \frac{J_z}{\hbar} \dd{\phi}
\end{equation}
where $J_z$ is a hermitian operator with dimensions of action (i.e., dimensions of $\hbar$ : $\qq{Energy}\times \qq{Time}$ or $\qq{Position}\times \qq{Momentum}$). At this state $J_z$ is \textbf{not} yet identified with the $z$-component of the total angular momentum operator. This identification will be made later after we derive the commutations of $J_z$ with other generators.\\

In Eq. (\ref{chapter18.eqn15}), $J_z$ is the generator of the unitary operator $U(\dd{\phi} \hat{z})$. The operator $U(\phi \hat{z})$ corresponding to a finite positive rotation $\phi$ about the $z$-axis can be obtained by successively compounding infinitesimal rotations about the same axis. Thus
\begin{align}
U_z(\phi) 
&= \lim\limits_{N \rightarrow \infty} \qty[1 - \frac{\iu J_z}{\hbar} \frac{\phi}{N}]^N \nonumber\\
\label{chapter18.eqn16a}
&= e^{-\iu J_z \phi/\hbar} \\
\label{chapter18.eqn16b}
&= 1 - \frac{\iu J_z \phi}{\hbar} - \frac{J_z^2 \phi^2}{2 \hbar^2}
\end{align}
Similarly we can write
\begin{align}
\label{chapter18.eqn17}
U_x(\phi) &= e^{-\iu J_x \phi/\hbar} \\
\label{chapter18.eqn18}
U_y(\phi) &= e^{-\iu J_y \phi/\hbar}
\end{align}
In general, for a positive rotation by an angle $\phi$ about an axis $\hat{n}$, we have
\begin{align}
\label{chapter18.eqn19}
U_{\hat{n}}(\phi) &= e^{-\iu \vec{J}\cdot\hat{n} \phi/\hbar}
\end{align}
From Eqs. (\ref{chapter18.eqn16a}), (\ref{chapter18.eqn17}) and (\ref{chapter18.eqn18}) we note that the hermitian operators $J_x,\ J_y$ and $J_z$ are the generators of the unitary transformation operaors in the Hilbert space if the system is rotated about the $x$, $y$ and $z$-axis, respectively. We will now show that the three generators $J_x,\ J_y$ and $J_z$ obey the commutation relations of angular momentum operators.



\section{Commutation relations of $J_x,J_y,J_z$}
%% page 14

To obtain the commutation relations between $J_x,\ J_y$ and $J_z$, we need the concept of a group (see Appendix (\ref{appendix4.group})). Now, the rotations form a group. The group multiplication is the application of two rotations successively. To see that the set of all rotations of a physical system form a group, we note that two successive rotations is equivalent to one single rotation. The inverse of a rotation $\phi \hat{n}$ is $-\phi\hat{n}$. The unit element is no rotation at all.\\


To every rotation of the physical system there corresponds a $3\times 3$ orthogonal matrix $R$ acting on the coordinates of a classical vectors, and a unitary operator $U(R)$ acting on the state kets in the Hilbert space. We say that the $3\times 3$ orthogonal matrices $R$ is a representation of the rotation group in the ordinary $3$-space. The set of unitary operator $U(R)$ is also a representation of the rotation group but in the Hilbert space of state vectors. Thus we may postulate $U(R)$ has the same group properties as $R$
\begin{table}
	\begin{tabular}{c|c|c}
		identity & $1\cdot R = R\cdot 1 = R$ & $\mathbb{1} U(R) = U(R) \mathbb{1} = U(R)$ \\ \hline
		closure & $R_1 R_2 = R_3$ & $U(R_1) U(R_2) = U(R_3)$ \\ \hline
		inverse & $R \cdot R^{-1} = 1 = R^{-1}\cdot R$ & 
			\begin{minipage}{5cm}
				\begin{align*}
				U(R) U(R^{-1}) = \mathbb{1} =  U(R^{-1})  U(R)\\
				\therefore  U(R^{-1}) =  U^{-1}(R) 
				\end{align*}
			\end{minipage}\\ \hline
		associativity & 
		\begin{minipage}{5cm}
			\begin{align*}
			R_1 \cdot (R_2 \cdot R_3) \\
			= (R_1 \cdot R_2) \cdot R_3 \\
			= R_1 \cdot R_2 \cdot R_3
			\end{align*}
		\end{minipage}
			 & 
		\begin{minipage}{5cm}
			\begin{align*}
			U(R_1)\qty(U(R_2) U(R_3)) \\
			= \qty(U(R_1)U(R_2)) U(R_3)\\
			= U(R_1)U(R_2) U(R_3)
			\end{align*}
		\end{minipage}
	\end{tabular}
\end{table}

Therefore, to any equation involving the unitary operator $U(R)$. The analogue of Eq. (\ref{chapter18.eqn9}) in Hilbert space is
\begin{equation}
\label{chapter18.eqn20}
U_x(\epsilon) U_y(\epsilon) - U_y(\epsilon) U_x(\epsilon) = U_z(\epsilon^2) - 1
\end{equation}

Eqs. (\ref{chapter18.eqn9}) and (\ref{chapter18.eqn20}) are valid up to second order in $\epsilon$. We therefore expand Eq. (\ref{chapter18.eqn20}) up to second order obtaining
\begin{equation}
\label{chapter18.eqn21}
\begin{split}
\qty(1 - \frac{\iu J_x \epsilon}{\hbar} - \frac{J_x^2 \epsilon^2}{2 \hbar^2}) 
\qty(1 - \frac{\iu J_y \epsilon}{\hbar} - \frac{J_y^2 \epsilon^2}{2 \hbar^2})
-
\qty(1 - \frac{\iu J_y \epsilon}{\hbar} - \frac{J_y^2 \epsilon^2}{2 \hbar^2})
\qty(1 - \frac{\iu J_x \epsilon}{\hbar} - \frac{J_x^2 \epsilon^2}{2 \hbar^2})\\
= \qty(1 - \frac{\iu J_z \epsilon^2}{\hbar}) - 1
\end{split}
\end{equation}
Terms of the order $\epsilon$ automatically drop out. Equating terms of order  $\epsilon^2$ on both sides of Eq. (\ref{chapter18.eqn21}) we obtain

\begin{equation}
\label{chapter18.eqn22}
\comm{J_x}{J_y} = \iu \hbar J_z
\end{equation}

Repeating this kind of arguments with rotations about other axes, we obtain
\begin{align}
\label{chapter18.eqn23}
\comm{J_y}{J_z} = \iu \hbar J_x \\
\label{chapter18.eqn24}
\comm{J_z}{J_x} = \iu \hbar J_y
\end{align}
Equations (\ref{chapter18.eqn22}) to (\ref{chapter18.eqn24}) are the fundamental commutation relation of angular momentum operators.
We can combine Eqs. (\ref{chapter18.eqn22}) to (\ref{chapter18.eqn24}) as
\begin{equation}
	\comm{J_i}{J_j} = \iu \hbar \varepsilon_{i j k} J_k
\end{equation}
Where $\varepsilon_{i j k}$ is the levi-civita symbol.


 We thus conclude that the generators of the unitary transformations of state vectors in Hilbert space corresponding to rotations of the physical system are nothing but the angular momentum operators.



\section{Rotation operator applied to a spinless particle}
%%% page 21





\section{Rotation operator in spin space}
%% page 27
\subsection{Matrix Representation}
%% page 32

\subsection{Rotation of two component spinors}
%% page 39




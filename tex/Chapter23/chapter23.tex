%% writing by shahnoor

\chapter{sheet-23 : Time Dependent Perturbation}
\ifpdf
\graphicspath{{Chapter23/figs/}}
\else
\graphicspath{{Chapter23/figs/}}
\fi


Suppose that a quantum mechanical system is described by a Hamiltonian $H_0$. Then we proceed to act on the system by a time-dependent external force described by a potential $V(t)$ added to $H_0$. The new Hamiltonian is
\begin{equation}
\label{chapter23.eqn1}
H = H_0 + V(t)
\end{equation}
The problem with $V(t) = 0$ is assumed to be solved exactly. In other words, the energy eigenvalues $E_n$ and eigenkets $\ket{n}$ defined by Eq. (\ref{chapter23.eqn2}) are known exactly.
\begin{equation}
\label{chapter23.eqn2}
H_0 \ket{n} = E_n \ket{n}
\end{equation}

We are interested in situations where the system is initially in an eigenstate of $H_0$, say $\ket{i}$. The time-dependent potential $V(t)$ can cause transitions to states other than $\ket{i}$. The basis question we ask is : what is the probability at some later time $t$, for the system to be found in the state $\ket{n}$ with $n\neq i$. As an example we might shine light on an atom and ask what are the chances that light ionizes the atom. \\

To formulate the problem, we have to solve the time-dependent Schr\"{o}dinger equation (\ref{chapter23.eqn3}) with the initial condition (\ref{chapter23.eqn4}).
\begin{align}
\label{chapter23.eqn3}
\iu \hbar \pdv{}{t}\ket{\psi(t)} &= \qty[H_0 + V(t)] \ket{\psi(t)} \\
\label{chapter23.eqn4}
\ket{\psi(t_0)} &= \ket{i} \qq{(initial condition)}
\end{align}

where $t_0$ is some earlier time. Then the probability that at some later time $t\ (t > t_0)$ the system makes a transition to the state $\ket{n}$ is given by Eq. (\ref{chapter23.eqn5}).
\begin{equation}
\label{chapter23.eqn5}
P_{i \rightarrow n}(t) = \norm{\braket{n}{\psi(t)}}^2
\end{equation}
%% page 3

To solve the problem, it is convenient to work in the interaction picture defined by Eqs. (\ref{chapter23.eqn6})
\begin{align}
\label{chapter23.eqn6}
\begin{split}
\ket{\psi_I(t)} &= e^{\iu H_0 (t-t_0)/\hbar} \ket{\psi(t)} \\
\hat{A}_I(t) &= e^{\iu H_0 (t-t_0)/\hbar} A e^{-\iu H_0 (t-t_0)/\hbar}
\end{split}
\end{align}
In the interaction picture we have
\begin{equation}
\label{chapter23.eqn7}
\iu \hbar \pdv{}{t} \ket{\psi_I(t)} = V_I(t) \ket{\psi_I(t)}
\end{equation}
i.e., the time evolution of the state vector in the interaction picture is determined by the time-dependent potential $V_I(t)$ expressed in the interaction picture
\begin{equation}
\label{chapter23.eqn8}
	V_I(t) = e^{\iu H_0 (t-t_0)/\hbar} V(t) e^{-\iu H_0 (t-t_0)/\hbar}
\end{equation}
Note that the state vector in the interaction picture coincides with the state vector in the Schr\"{o}dinger picture at the initial time $t_0$, i.e.,
\begin{equation}
\label{chapter23.eqn9}
\ket{\psi_I(t_0)} = \ket{\psi(t_0)} = \ket{i}
\end{equation}

Now, from Eq. (\ref{chapter23.eqn5}), the probability for transition from state $\ket{i}$ to state $\ket{n}$ can be written as
\begin{align}
P_{i\rightarrow n}(t) 
&= \norm{\mel{n}{e^{-\iu H_0(t - t_0)/\hbar}}{\psi_I(t)}}^2 \nonumber \\
&= \norm{\mel{n}{e^{-\iu E_n(t - t_0)/\hbar}}{\psi_I(t)}}^2 \nonumber \\
&= \norm{\braket{n}{\psi_I(t)}}^2
\label{chapter23.eqn10}
\end{align}
We can now solve Eq. (\ref{chapter23.eqn7}) for $\ket{\psi_I(t)}$ with the initial condition (\ref{chapter23.eqn9}) and the final transition probability using Eq. (\ref{chapter23.eqn10}).

%% page 5

In the interaction picture we can continue using $\{\ket{n}\}$ as our base kets and expand
\begin{equation}
\label{chapter23.eqn11}
	\ket{\psi_I(t)} = \sum_{n} a_n(t) \ket{n}
\end{equation}
where $a_n(t) = \braket{n}{\psi_I(t)}$. The expansion coefficient satisfy the initial conditions
\begin{equation}
\label{chapter23.eqn12}
a_n(t_0) = \begin{cases}
0 \quad if \ n\neq i \\
1 \quad if \ n = i 
\end{cases}
\end{equation}
Therefore the transition probability in Eq. () can be written as
\begin{equation}
\label{chapter23.eqn12-transition-proba}
P_{i\rightarrow n}(t) = \abs{a_n(t)}^2
\end{equation}


In our subsequent discussion we will take $t_0 = 0$ without any loss of generality. Now, taking the scalar product of Eq. (\ref{chapter23.eqn7}) with $\ket{n}$ we obtain
\begin{equation}
\label{chapter23.eqn13}
\iu \hbar \pdv{}{t} \braket{n}{\psi_I(t)} = \sum_{m} \mel{n}{V_I(t)}{m} \braket{m}{\psi_I(t)}
\end{equation}
where we have used the completeness relation $\sum_{m} \ketbra{m}{m} = 1$ on the right side of Eq. (\ref{chapter23.eqn13}). Next, we express the matrix element $\mel{n}{V_I(t)}{m}$ as

\begin{align}
\mel{n}{V_I(t)}{m} 
&= \mel{n}{e^{\iu H_0 (t-t_0)/\hbar} V(t) e^{-\iu H_0 (t-t_0)/\hbar}}{m} \nonumber\\
&= e^{\iu (E_n - E_m) t/\hbar} \mel{n}{V(t)}{m} \ (\text{with } t_0 = 0) \nonumber\\
&= e^{\iu \omega_{n m} t} V_{n m}(t)
\label{chapter23.eqn14}
\end{align}

Where we have defined
\begin{align}
\label{chapter23.eqn15}
\omega_{n m} &= \frac{E_n - E_m}{\hbar} \\
\label{chapter23.eqn16}
V_{n m}(t) &= \mel{n}{V(t)}{m}
\end{align}

Thus, Eq. (\ref{chapter23.eqn3}) can be written as
\begin{equation}
\label{chapter23.eqn17}
\iu \hbar \dv{a_n(t)}{t} = \sum_{m} V_{n m}(t) e^{\iu \omega_{n m} t} a_m(t)
\end{equation}
Explicitly this equation is
\begin{equation}
\label{chapter23.eqn18}
\iu \hbar \pdv{}{t}
\mqty[a_1(t) \\ a_2(t) \\ \vdots] = 
\mqty[V_{11}(t) & V_{12}(t) e^{\iu\omega_{12}t} & \ldots\\
V_{21}(t)e^{\omega_{21}t} & V_22(t) & \ldots \\
\vdots & \vdots & \vdots
]
\mqty[a_1(t) \\ a_2(t) \\ \vdots]
\end{equation}

%% page 7
This is the basic coupled differential equation we must solve, with the boundary condition (\ref{chapter23.eqn12}). So far no approximation has been made.


\section{Perturbation Scheme for Solving Eq. (\ref{chapter23.eqn17})}
The exact solutions of the coupled differential Eqs. (\ref{chapter23.eqn17}) is very difficult. We can derive a perturbation scheme by writing
\begin{equation}
V_{n m}(t) = \lambda V_{n m}(t)
\end{equation}
and then expanding $a_n(t)$ in a power series in $\lambda$
\begin{equation}
\label{chapter23.eqn19}
a_n(t) = a_n^{(0)}(t) + \lambda a^{(1)}_n (t) + \lambda^2 a^{(2)}_n (t) + \ldots
\end{equation}
The parameter $\lambda$ is introduced just to count the order of perturbation and $\lambda$ will be set equal to $1$ at the end. Substituting Eq. (\ref{chapter23.eqn19}) in Eq. (\ref{chapter23.eqn17}) and equating the coefficient of equal powers of $\lambda$ we find

\begin{align}
\label{chapter23.eqn20}
\begin{split}
\dot{a}_n^{(0)}(t) &= 0 \\
\dot{a}_n^{(1)}(t) &= \frac{1}{\iu \hbar} \sum_{m} V_{n m}(t) e^{\iu \omega_{n m} t} a_{m}^{(0)} \\
\dot{a}_n^{(2)}(t) &= \frac{1}{\iu \hbar} \sum_{m} V_{n m}(t) e^{\iu \omega_{n m} t} a_{m}^{(1)}(t) \\
\vdots & \quad \vdots \\
\dot{a}_n^{(s+1)}(t) &= \frac{1}{\iu \hbar} \sum_{m} V_{n m}(t) e^{\iu \omega_{n m} t} a_{m}^{(s)}(t) \\
\end{split}
\end{align}
These equations can now, in principle, be integrated successively to any given order in perturbation. The first Eq. (\ref{chapter23.eqn20}) simply confirms that $a_n^{(0)}$ is independent of time. We take
\begin{equation}
\label{chapter23.eqn21}
a_n^{(0)} = 
\begin{cases}
1 \qq{if} n = i \\
0 \qq{if} n \neq i
\end{cases}
\end{equation}
In order to satisfy the initial condition (\ref{chapter23.eqn12}). The higher order corrections $a_n^{(1)}(t), a_n^{(2)}(t), \ldots$ have to be evaluated by solving Eq. (\ref{chapter23.eqn20}) with the initial conditions
\begin{equation}
\label{chapter23.eqn22}
	a_n^{(1)}(t_0), a_n^{(2)}(t_0) = \ldots = 0
\end{equation}
%%% page 10
Now, substituting Eq. (\ref{chapter23.eqn21})  in the second of Eqs. (\ref{chapter23.eqn20}), we obtain first order
\begin{equation}
\label{chapter23.eqn23}
\dot{a}_n^{(1)}(t) = \frac{1}{\iu \hbar} V_{n i}(t) e^{\iu \omega_{n i} t} \qq{for all n}
\end{equation}
Integrating this we get
\begin{equation}
\label{chapter23.eqn24}
a_n^{(1)}(t) = \frac{1}{\iu \hbar} \int_{t_0}^{t} V_{n i}(t^\prime) e^{\iu \omega_{n i} t^\prime} \dd{t^\prime}
\end{equation}
The initial condition $a_n^{(1)}(t_0) = 0$ is automatically satisfied in the above equation.

%% page 11

Next, substituting Eq. (\ref{chapter23.eqn24}) into the third of Eqs. (\ref{chapter23.eqn20}), we obtain in second order
\begin{align*}
	\dot{a}_n^{(2)}(t) &= \frac{1}{\iu \hbar} \sum_{m} V_{n m}(t) e^{\iu \omega_{n m} t} a_m^{(1)}(t) \\
	or,\ a_n^{(2)}(t) &= \frac{1}{\iu \hbar} \sum_{m} \int_{t_0}^{t} \dd{t^\prime} V_{n m}(t^\prime) e^{\iu \omega_{n m} t^\prime} a_m^{(1)}(t^\prime) \\
	&=\frac{1}{\iu \hbar} \sum_{m} \int_{t_0}^{t} \dd{t^\prime} V_{n m}(t^\prime) e^{\iu \omega_{n m} t^\prime} 
	\frac{1}{\iu \hbar} \int_{t_0}^{t^\prime} \dd{t^{\prime\prime}} V_{m i}(t^{\prime\prime}) e^{\iu \omega_{m i} t^{\prime\prime}}
\end{align*}
Therefore
\begin{equation}
\label{chapter23.eqn25}
a_n^{(2)}(t) = \frac{1}{(\iu \hbar)^2} \sum_{m} \int_{t_0}^{t} \dd{t^\prime} \int_{t_0}^{t^\prime} \dd{t^{\prime\prime}} V_{n m}(t^\prime) e^{\iu \omega_{n m} t^\prime} V_{m i}(t^{\prime\prime}) e^{\iu \omega_{m i} t^{\prime\prime}}
\end{equation}
Continuing in this fashion we can obtain the higher order correction to $a_n(t)$. \\

We have now completed the formalism for time-dependent perturbation theory. We will now apply the formalism to some specific problems.



\section{Examples}
\subsection{Constant perturbation}
%% page 12
A constant perturbation switched on at $t=0$. As an application of the time-dependent perturbation theory, let us consider a constant perturbation suddenly turned on at $t=0$
\begin{equation}
\label{chapter23.eqn26}
V(t) = \begin{cases}
0 \qq{for} t<0 \\
V \qq{for} t \geq 0
\end{cases}
\end{equation}
\begin{figure}
	%%%%%%% TODO
	\centering
	\includegraphics[width=0.5\linewidth]{Pictures/not-found.jpg}
	\caption{Step potential}
	\label{chapter23.fig1}
\end{figure}
we have
\begin{equation}
a_n^{(0)} = \delta_{n i}
\end{equation}
and with $t_0 = 0$
\begin{align*}
a_n^{(1)}(t) 
&= \frac{1}{\iu \hbar} V_{n i} \int_{0}^{t} e^{\iu \omega_{n i} t^\prime} \dd{t^\prime}\\
&= \frac{1}{\iu \hbar} V_{n i} \eval{\frac{e^{\iu \omega_{n i} t^\prime}}{\iu \omega_{n i}}}{0}{t} \\
&= \frac{1}{\iu \hbar} V_{n i} \left(\frac{e^{\iu \omega_{n i} t} - 1}{\iu \omega_{n i}}\right)
\end{align*}
Therefore
\begin{equation}
\label{chapter23.eqn27}
	a_n^{(1)}(t) = \frac{1}{\iu \hbar} V_{n i} e^{\iu \omega_{n i} t/2}\frac{\sin \omega_{n i} t/2}{\omega_{n i} t/2}
\end{equation}
Therefore, in first order, the probability of transition from an initial state $\ket{i}$ to a final state $\ket{n}$ s.t. $(n \neq i)$
\begin{align}
\label{chapter23.eqn28}
P_{i \rightarrow n}(t) = \abs{a_n^{(1)}(t)}^2 = \frac{1}{\hbar^2} \abs{V_{n i}}^2 \frac{\sin[2]{\omega_{n i}t/2}}{\qty(\omega_{n i}/2)^2}
\end{align}
The probability of transition of the state $n$ depends not only on $\abs{V_{n i}}^2$ but also on $\omega_{n i}$, i.e., on the energy difference $E_n - E_i$. Note that if $V_{n i} = 0$, there would be no transition to the state $\ket{n}$. In other words, to have a transition to the final state $\ket{n}$, the potential $V$ should have a spatial dependence such that $V_{n i} = \mel{n}{V}{i} \neq 0$. The transition probability is shown in Fig. (\ref{chapter23.fig2}).

\begin{figure}
	%%%%%%% TODO
	\centering
	\includegraphics[width=0.5\linewidth]{Pictures/not-found.jpg}
	\caption{Transition probability $P_{i\rightarrow n}$ as a function of $E_n$.}
	\label{chapter23.fig2}
\end{figure}
We see that $P_{i \rightarrow n}(t)$ exhibits a sharp peak about $E_n = E_i$. The height of the peak is proportional to $t^2$ while its width is approximately $2 \pi \hbar / t$. Thus the probability of transition to a state $n$ is large when its energy lies under the bump around $E_i$. The final energy will lie under the bump if $(E_n - E_i) < \frac{2 \pi \hbar}{t}$. This means that transition $i \rightarrow n$ will occur mainly towards those final states whose energy is located in a band of width $\delta E \approxeq \frac{2 \pi \hbar}{t}$ about the initial energy $E_i$, so that energy of the system is conserved within $\frac{2 \pi \hbar}{t}$. This result can be related to the time-energy uncertainly relation 
\begin{equation}
\Delta E \cdot \Delta t \sim \hbar
\end{equation}
where $\Delta t (=t)$ is the length of time the perturbation has acted and $\Delta E \sim \delta E$. If $\Delta t$ is small we have a broader peak, and as a result we tolerate a fair amount of energy non-conservation. On the other hand, if the perturbation has been on for very long time, we have a narrow peak, and approximate energy conservation is required for a transition with appreciable probability.

%% page 17
In practice, we are interested to find the transition probability to a group of final states $[n]$ whose energy is roughly degenerate with the initial state energy and lies within the range $E_n - \varepsilon/2, E_n + \varepsilon/2)$ centered about the value $E_n$. This is the case, for example, when one studies transitions to states belonging to the continuous spectra.


In such a case we are interested in the total probability, that is transition probabilities summed over final states with $E_n \simeq E_i$. 
\begin{equation}
\label{chapter23.eqn29}
P_{i \rightarrow [n]} = \sum_{\substack{n \\ E_n\simeq E_i}} \abs{a_n^{(1)}(t)}^2
\end{equation}

Let us now define by $\rho(E_n)$ the density of levels on the energy scale, so that $\rho(E_n) \dd{E_n}$  is the number of final states within the energy interval $(E_n, E_n + \dd{E_n})$, thus Eq. (\ref{chapter23.eqn29}) can be written as
\begin{equation}
\label{chapter23.eqn30}
	P_{i \rightarrow [n]}(t) = \int \dd{E_n} \rho(E_n) \abs{a_n^{(1)}(t)}^2
\end{equation}
where the spread in the final state energy is $\varepsilon$. Using Eq. (\ref{chapter23.eqn28}) in (\ref{chapter23.eqn30}) we obtain
\begin{equation}
\label{chapter23.eqn31}
P_{i \rightarrow [n]}(t) = 4 \int \dd{E_n} \rho(E_n) \sin[2]{\frac{(E_n - E_i)t}{2 \hbar}} \frac{\abs{V_{n i}}^2}{(E_n - E_i)^2}
\end{equation}
Now, as $t$ becomes large, we take advantage of the fact that
\begin{equation}
\lim\limits_{t \rightarrow \infty} \frac{1}{(E_n - E_i)^2} \sin[2]{\frac{(E_n - E_i)t}{2 \hbar}} = \frac{\pi t}{2 \hbar} \delta(E_n - E_i)
\end{equation}
which follows from
\begin{equation}
\lim\limits_{t \rightarrow \infty} \frac{1}{(\pi} \frac{\sin[2]{\alpha x}}{\alpha x^2} =  \delta(x)
\end{equation}
Thus, for large times the contribution to the integral in Eq. (\ref{chapter23.eqn31}) comes from a small band of energy around $E_i$. It is now possible to take $\abs{V_{n i}}^2$ outside the integral and perform the integration with the $\delta$-function. Thus
\begin{align}
P_{i \rightarrow [n]}(t) &= \abs{V_{n i}}^2 \frac{4\pi t}{2 \hbar} \int \dd{E_n} \delta(E_n - E_i)\rho(E_n)  \nonumber\\
P_{i \rightarrow [n]}(t) &= \abs{V_{n i}}^2 \frac{2\pi t}{\hbar} \eval{\rho(E_n)}_{E_n=E_i}
\label{chapter23.eqn32}
\end{align}
Thus the total probability is proportional to $t$ for large values of $t$. Notice that linearity in $t$ is a consequence of the fact that the total transition probability to the area under the peak in Fig. (\ref{chapter23.fig1}) where the height varies as $t^2$ and the width as $1/t$.

It is convenient to consider the transition rate - that is the transition probability per unit time. The transition rate to a group of final state is
\begin{align}
W_{i \rightarrow [n]} &= \dv{}{t} P_{i \rightarrow [n]}(t) \nonumber\\
&= \abs{V_{n i}}^2 \frac{2\pi}{\hbar} \eval{\rho(E_n)}_{E_n=E_i}
\label{chapter23.eqn33}
\end{align}
This formula, which is of great practical importance, is called \textbf{Fermi's golden rule}\index{Fermi's golden rule}. We sometimes write Eq. (\ref{chapter23.eqn33}) as
\begin{equation}
\label{chapter23.eqn34}
W_{i \rightarrow [n]} \frac{2 \pi}{\hbar} \abs{V_{n i}}^2 \delta(E_n - E_i)
\end{equation}
where it must be understood that the expression is integrated with $\int \dd{E_n} \rho(E_n)$.
We should also understand what is meant by $\abs{V_{n i}}^2$. There may be several different groups of final states $n_1, n_2, \ldots$ all of which have about the same energy $E_i$ but for which the perturbation matrix elements $\abs{V_{n i}}^2$ and the density of states $\rho(E_n)$ although nearly constant within each group, differs from one group to another. In such cases we must treat each group separately even though they are degenerate in energy.


\subsection{Oscillating Electric Field}
%Hydrogen atoms in between the plates of a parallel plate capacitor
%% page 22
A system of hydrogen atoms in the ground state is contained between the plates of a parallel plate capacitor. A voltage pulse is applied to the capacitor so as to produce a homogeneous electric field
\begin{equation}
\varepsilon = 
\begin{cases}
0 \qq{for} t < 0 \\
\varepsilon_0 e^{-t / \tau} \qq{for} t > 0
\end{cases}
\end{equation}
\begin{enumerate}
	\item Show that after a long time, the fraction of atoms in the $2 p (m=0)$ state $(\ket{n l m} = \ket{2 1 0})$ is, to first-order $\frac{2^15}{3^10} \frac{a_0 ^2 e^2 \varepsilon^2}{\hbar^2 \qty(\omega^2 + \frac{1}{\tau^2})}$. Where $a_0$ is the Bohr radius and $\hbar \omega$ is the energy difference between the $2p$ and the ground state.
	
	\item What is the fraction of atoms in the $2s$ state?
\end{enumerate}
Given:
\begin{align*}
\psi_{1 0 0}(\vec{r}) &= R_{1 0}(r) Y_{0 0} = \frac{2}{\sqrt{4 \pi}} \frac{1}{a^{3/2}} e^{-r / a_0} \\
\psi_{2 1 0}(\vec{r}) &= R_{2 1}(r) Y_{1 0}(\theta,\phi) = \frac{1}{4\pi} \frac{1}{(2 a_0)^{3/2}} \qty(\frac{r}{r_0}) e^{-r / 2 a_0} \cos\theta \\
\int_{0}^{\infty} r^n e^{-\beta r} \dd{r} &= \frac{n!}{\beta^{n + 1}} \qq{($\beta > 0$, $n$=positive integers)}
\end{align*}
Solution:\\

\begin{figure}
	%%%%%%% TODO
	\centering
	\includegraphics[width=0.4\linewidth]{Pictures/not-found.jpg}
	\includegraphics[width=0.4\linewidth]{Pictures/not-found.jpg}
	\caption{Parallel electric field}
	\label{chapter23.fig4}
\end{figure}

The electric filed is homogeneous but time varying. Suppose $\vec{\varepsilon}$ points along the $z$-axis. Therefore
\begin{equation}
\vec{\varepsilon} = (0, 0, \varepsilon_z)
\end{equation}
with
\begin{equation}
\varepsilon_z = \varepsilon_0 e^{-t / \tau} \qq{for} t>\tau
\end{equation}
The force on the electron has only $z$-component. We have
\begin{equation}
F_z = q_e \varepsilon_z = - e \varepsilon_0 e^{-t/\tau}
\end{equation}
where $q_e = -e = \qq{charge of electron}$
Now $F_z = - \pdv{V}{z}$ gives
\begin{equation}
\label{chapter23.eqn1-oscillating}
V = e \varepsilon_0 z e^{-t/\tau} \quad (t > 0)
\end{equation}
The hydrogen atom is in the ground state $\ket{100}$ at $t=0$. The perturbation is switched on at $t=0$. In the first-order perturbation theory, the probability of transition from state $i$ to state $n$ is 
\begin{equation}
P_{i \rightarrow n}(t) = \abs{a_n^{(1)}(t)}^2
\end{equation}
where
\begin{equation}
a_n^{(1)}(t) = \frac{1}{\iu \hbar} \int_{0}^{t} \mel{n}{V(t^\prime)}{i} e^{\iu \omega_{n i} t^\prime} \dd{t^\prime}
\end{equation}
where $\omega_{n i} = \frac{E_n - E_i}{\hbar}$. Writing $\omega \equiv \omega_{n i}$, we find in the present example
\begin{equation}
a_n^{(1)}(t) = \frac{1}{\iu \hbar} e \varepsilon_0 \mel{n}{z}{i} \int_{0}^{t} e^{(\iu \omega - 1/\tau) t^\prime} \dd{t^\prime}
\end{equation}
For $t (t >> t^\prime)$ we have
\begin{align}
a_n^{(1)}(t) 
&= \frac{e \varepsilon_0}{\iu \hbar}  \mel{n}{z}{i} \left[- \frac{1}{(\iu \omega - 1/\tau)}\right] \nonumber\\
&= \frac{e \varepsilon_0}{\hbar(\omega - \iu/\tau)}  \mel{n}{z}{i}
\label{chapter23.eqn2-oscillating}
\end{align}
Now, states of the hydrogen atoms (disregarding spin) are denoted by $\ket{n l m}$. The initial state is the ground state, i.e., $\ket{i} = \ket{1 0 0}$ while the final state $\ket{n}$ is $2 p (m=0)$, i.e., $\ket{n} = \ket{2 1 0}$
Also
\begin{align*}
\braket{\vec{r}}{1 0 0} &= \psi_{1 0 0}(\vec{r})
\braket{\vec{r}}{2 1 0} &= \psi_{2 1 0}(\vec{r})
\end{align*}
Therefore
\begin{align*}
\mel{2 1 0}{z}{1 0 0}
&= \int \psi^*_{2 1 0}(\vec{r}) r\cos\theta \psi_{1 0 0}(\vec{r}) \\
&= \frac{2}{4\pi} \frac{1}{2^{3/2} a_0^4} \int e^{- r / 2 a_0} \cos\theta r^2 \cos\theta e^{-r/a_0} r^2 \dd{r}\dd{\Omega} \\
&= \frac{2}{4\pi} \frac{1}{2^{3/2} a_0^4} \int_0^\infty r^4 e^{-3 r/2 a_0} \int_{0}^{\pi/2} \cos[2]{\theta} \sin\theta\dd{\theta} \int_{0}^{2 \pi} \dd{\phi} \\
&= \frac{2}{4\pi} \frac{1}{2^{3/2} a_0^4} \cdot 2\pi \cdot \frac{2}{3} \int_{0}^{\infty} r^4 e^{-3 r/2 a_0} \dd{r} \\
&= \frac{2}{2^{3/2} 3 a_0} \frac{4!}{\qty(\frac{2}{2 a_0})^5} \\
&= \frac{4!}{2^{3/2}} \qty(\frac{2}{3})^6 a_0
\end{align*}
Therefore the probability of transition from the state $\ket{1 0 0}$ to the state $\ket{2 1 0}$ is 
\begin{align*}
P_{i \rightarrow n}(t) 
&= \abs{a_n^{(1)}(t)}^2 \\
&= \frac{e^2 \varepsilon_0^2}{\hbar^2 (\omega^2 + \frac{1}{\tau^2})} \cdot \frac{(4!)^2 2^{12}}{2^3 3^12} a_0 ^2 \\
&= \qty(\frac{2^15}{3^10}) \frac{e^2 \varepsilon_0^2 a_0^2}{\hbar^2 (\omega^2 + \frac{1}{\tau^2})} 
\end{align*}
This is the fraction of atoms in the state $\ket{2 1 0}$ for large times $(t >> \tau)$.\\

(b) Probability of transition to $2 s = \ket{2 0 0}$ state is zero. Because the matrix  element of $z$ between the initial and final states is $\mel{2 0 0}{z}{1 0 0} = 0$. Because the integrand is odd. So there is no transition to the $\ket{2 0 0}$ state from the ground state.




\subsection{Harmonic Perturbation}
%% page 29
We now consider a sinusoidally varying time-dependent potential, commonly referred to as harmonic perturbation:
\begin{equation}
\label{chapter23.eqn1-harmonic}
V(t) = V e^{i \omega t} + V^\dagger e^{-i \omega t}
\end{equation}
where $V$ may still depend on $\hat{\vec{r}}, \hat{\vec{p}}$ and $\hat{\vec{s}}$. 

We assume that only one of the eigenstates of $H_0$ is populated initially. The perturbation is turned on at $t=0$. So, in first order, the transition amplitude from state $i$ to state $n$ is
\begin{align}
	a_n^{(1)} 
	&= \frac{1}{\iu\hbar} \int_{0}^{t} \qty(V_{n i} e^{\iu \omega t^\prime} + V_{n i}^\dagger e^{-\iu \omega t^\prime}) e^{\iu \omega_{n i} t^\prime} \dd{t^\prime} \nonumber\\
	\label{chapter23.eqn2-harmonic}
	&= \frac{1}{\hbar} \left[\frac{1 - e^{\iu(\omega - \omega_{n i})t}}{\omega + \omega_{n i}} V_{n i} 
	+ \frac{1 - e^{\iu(\omega_{n i} - \omega )t}}{-\omega + \omega_{n i}} V_{n i}^\dagger\right]
\end{align}

It is clear from the above equation that if $t$ is large enough, the probability of finding the system will be appreciable if the denominator of the one or the other of the two terms on the right of Eq. (\ref{chapter23.eqn2-harmonic}) is close to zero. Moreover, assuming that $E_n \neq E_i$ (so that the levels $E_n$ and $E_i$ are not degenerate), both the denominator cannot simultaneously close to zero. A good approximation is therefore to neglect the interference between the two terms in calculating the transition probabilities.
\begin{enumerate} [start=0,label={(case \arabic*):}]
	\item suppose $\omega_{n i} + \omega \simeq 0$ which means $E_n \simeq E_i - \hbar \omega$.
	Then
\begin{align*}
	P_{i \rightarrow n}(t) 
	&= \abs{a_n^{(1)}(t)}^2 \\
	&\simeq \frac{\abs{V_{n i}}^2}{\hbar^2} \abs{\frac{1 - e^{\iu (\omega + \omega_{n i})t}}{\omega + \omega_{n i}}}^2 \\
	&= \frac{\abs{V_{n i}}^2}{\hbar^2} \frac{\sin[2]{(\omega + \omega_{n i})t/2}}{[(\omega + \omega_{n i})t/2]^2} \\
	&= \frac{\abs{V_{n i}}^2}{\hbar^2} \cdot 2 \pi t \delta(\omega + \omega_{n i})
\end{align*}
	The transition rate is than
	\begin{align*}
	W_{i \rightarrow n}(t) 
	&= \dv{}{t} P_{i \rightarrow n}(t) \\
	&= \frac{2 \pi}{\hbar^2}  \abs{V_{n i}}^2 \delta(\omega + \omega_{n i}) \\
	&= \frac{2 \pi}{\hbar^2}  \abs{V_{n i}}^2 \delta(\omega + \frac{E_n - E_i}{\hbar}) \\
	&= \frac{2 \pi}{\hbar}  \abs{V_{n i}}^2 \delta(E_n - E_i + \hbar\omega)
	\end{align*}
	Therefore the transition rate to a group of final state is
	\begin{align}
		W_{i \rightarrow [n]}(t)
		&= \frac{2 \pi}{\hbar}  \abs{V_{n i}}^2 \int \delta(E_n - E_i + \hbar\omega)\rho(E_n) \dd{E_n} \nonumber\\
		\label{chapter23.eqn5-harmonic}
		&= \frac{2 \pi}{\hbar}  \abs{V_{n i}}^2 \eval{\rho(E_n)}_{}^{E_n = E_i - \hbar\omega}
	\end{align}
	
	\item Next suppose that the denominator of the second term of Eq. (\ref{chapter23.eqn2-harmonic}) is close to zero. Therefore $-\omega + \omega_{n i} = 0$, i.e., $E_n = E_i + \hbar\omega$. Then proceeding exactly as in case $1$, we have
	\begin{equation}
	\label{chapter23.eqn6-harmonic}
	W_{i \rightarrow [n]}(t)
	= \frac{2 \pi}{\hbar}  \abs{V_{n i}}^2 \eval{\rho(E_n)}_{}^{E_n = E_i + \hbar\omega}
	\end{equation}
\end{enumerate}
We see from Eqs.(\ref{chapter23.eqn5-harmonic}) and (\ref{chapter23.eqn6-harmonic}) that, in case of harmonic perturbation, we do not have energy conservation satisfied by the quantum-mechanical system above. Rather, the apparent lack of energy conservation is compensated by the energy given out to or energy taken from the external potential $V(t)$.

In case of Eq. (\ref{chapter23.eqn5-harmonic}) we have stimulated emission\index{stimulated emission, absorption}. The quantum mechanical system gives up energy $\hbar\omega$ to $V$ Fig. (\ref{chapter23.fig3}). Clearly, Stimulated emission is possible if the quantum mechanical system is in an excited state.

\begin{figure}
	%%%%%%% TODO
	\centering
	\includegraphics[width=0.4\linewidth]{Pictures/not-found.jpg}
	\includegraphics[width=0.4\linewidth]{Pictures/not-found.jpg}
	\caption{Stimulated emission: system gives up energy $\hbar\omega$ to $V$ and the initial state $i$ is an excited state.}
	\label{chapter23.fig3}
\end{figure}


Next, 
in case of Eq. (\ref{chapter23.eqn5-harmonic}) we have absorption\index{stimulated emission, absorption}. The quantum mechanical system receives energy $\hbar\omega$ from $V$ and ends up in an excited state Fig. (\ref{chapter23.fig3}). Thus a time-dependent perturbation can be regarded as an inexhaustible source or sink of energy.


Now, note that
\begin{equation}
V_{n i} = \mel{n}{V}{i} = \mel{i}{V^\dagger}{n}^* = V_{i n}^{\dagger *}
\end{equation}
Therefore
\begin{equation}
\label{chapter23.eqn7-harmonic}
	\abs{V_{i n}^\dagger}^2 = \abs{V_{i n}}^2
\end{equation}

Combining Eq. (\ref{chapter23.eqn7-harmonic}) with (\ref{chapter23.eqn5-harmonic}) and (\ref{chapter23.eqn6-harmonic}) we have
\begin{equation}
\label{chapter23.eqn8-harmonic}
\frac{\qq{emission rate for} i \rightarrow [n]}{\qq{density of final states for} [n]} = \frac{2 \pi}{\hbar} \abs{V_{n i}}^2 = \frac{2 \pi}{\hbar} \abs{V_{n i}^\dagger}^2 = \frac{\qq{Absorption rate for} n \rightarrow [i]}{\qq{density of final states for} [i]}
\end{equation}
where in the absorption case we let $i$ stand for the final states. Eq. (\ref{chapter23.eqn8-harmonic}) which expresses symmetry between absorption and emission is known as detailed balancing\index{detailed balancing}.

To summarize:\\
	For constant perturbation, we obtain appreciable transition probability for $\ket{i} \rightarrow \ket{n}$ only if $E_n \simeq E_i$. In contrast, for harmonic perturbation we have appreciable transition probability only if $E \simeq E_i \pm \hbar\omega$ for stimulated emission or absorption \index{stimulated emission, absorption}.
%\chapterimage{head2.png} % Chapter heading image
%\appendix{Gamma and Beta function}


\chapter{Group and Their Properties}
\label{appendix4.group}

 A group $G$ is a collection of objects $g_1,g_2,\ldots$, satisfying the following properties:
 \begin{enumerate}
 	\item[closure]: If $g_1 \in G$ and $g_2 \in G$, then
 	\begin{equation}
	 	g_1 \cdot g_2 \in G
 	\end{equation}
 	where the symbol $'\cdot'$ denotes the \textit{group multiplication}. In other words, a binary operation between the group elements is defined (the binary operation called 'multiplication') such that the product of two group elements leads to another group element. This property of group multiplication is called the closure property \index{group property}.
 	
 	
 	
 	
 	\item[identity] There exists an element in the group, called the unit element \index{group property}, and denoted by $1$ or $e$ such that
 	\begin{equation}
	 	1 \cdot g = g \cdot 1 = g
 	\end{equation}
 	
 	
 	
 	
 	\item[inverse] For every $g\in G$, there exists another element in the group, called the inverse \index{group property} of $g$ and denoted by $g^{-1}$ such that
 	\begin{equation}
	 	g\cdot g^{-1} = g^{-1}\cdot g = 1
 	\end{equation}
 	
 	
 	\item[associativity] Group multiplication is associative\index{group property}:
 	\begin{equation}
	 	g_1 \cdot (g_2 \cdot g_3) = (g_1 \cdot g_2) \cdot g_3
 	\end{equation}
 \end{enumerate}
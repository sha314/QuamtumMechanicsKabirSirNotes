\chapter{sheet-12 : The Path Integral Formulation of Quantum Theory}
\section{Background Materials}
\begin{enumerate}
	\item Basis states \\
	\begin{align}
		\hat{Q}_s \ket{q} &= q\ket{q} \\
		\hat{P}_s \ket{p} &= p\ket{p} 
	\end{align}
	The states $\{\ket{q}\}$ and $\{\ket{p}\}$ are basis states, i.e., $\mathbbm{1}$
	\begin{align}
		\int \dd{q} \ket{q}\bra{q} &= \mathbb{1} \\
		\int \dd{p} \ket{p}\bra{p} &= \mathbb{\hat{1}} \\
	\end{align}
	where the normalization is chosen as
	\begin{align}
		\braket{q}{q^\prime} &= \delta(q-q^\prime) \\
		\braket{p}{p^\prime} &= \delta(p-p^\prime) 
	\end{align}
	The operators $\hat{Q}_s$ and $\hat{P}_s$ can be expressed in coordinate representation as follows
	\begin{align}
		\bra{q}\hat{Q}_s &= q \bra{q} \\
		\bra{q}\hat{P}_s &= -\iu \hbar \pdv{q} \bra{q} 
	\end{align}
	In momentum representaion
	\begin{align}
		\bra{p}\hat{Q}_s &= \iu \hbar \pdv{p} \bra{p} \\
		\bra{p}\hat{P}_s &=  \bra{p} 
	\end{align}
	The fundamental commutation relation between $\hat{Q}$ and $\hat{P}$ is 
	\begin{equation}
		\left[\hat{Q}_s, \hat{P}_s\right] = \iu
		 \hbar \mathbb{\hat{1}}
	\end{equation}
	For later purposes we will need the momentum eigenstates in coordinate representation, i.e., $\braket{q}{p}$. To find $\braket{q}{p}$ we proceed as follows
	\begin{align}
		\hat{P}_s \ket{p} &= p\ket{p} \\
		\bra{q}\hat{P}_s \ket{p} &= p\bra{q}\ket{p} \\
		-\iu \hbar \pdv{q} \braket{q}{p} &= p \braket{q}{p}
	\end{align}
	This equation is easy to solve for $\braket{q}{p}$. We find
	\begin{equation}
		\braket{q}{p} = C e^{\iu p q / \hbar}
	\end{equation}
	The constant $C$ is chosen such that we have the normalization $\braket{p}{p^\prime } \delta(p-p^\prime)$. Now
	\begin{align*}
		\braket{p}{p^\prime} 
		&= \int \dd{q} \braket{p}{q}\braket{q}{p^\prime} \\
		&= \int \dd{q} C^* e^{-ipq/\hbar} C^{\iu p^\prime q /\hbar} \\
		&= \left|C\right|^2 \int_{-\infty}^{\infty} \dd{q} e^{-\iu(p-p^\prime)q/\hbar} \\
		&= \left|C\right|^2  2\pi \delta(\frac{p-p^\prime}{\hbar}) \\
		&= \left|C\right|^2  2\pi \hbar \delta(p-p^\prime)
	\end{align*}
	Choosing $C$ to be real and positive, we must have
	\begin{equation}
		C = \frac{1}{\sqrt{2\pi \hbar}}
	\end{equation}
	Thus
	\begin{equation}
		\braket{q}{p} = \frac{1}{\sqrt{2\pi \hbar}} e^{\iu p q / \hbar}
	\end{equation}
	
	
	\item Quantum Mechanics in Heisenberg picture//
	The Heisenberg picture of quantum dynamics is obtained from the Schr\"{o}dinger picture by the following transformation of all kets and all operators
	\begin{align}
		\ket{}_H &= e^{\iu \hat{H}t/\hbar}\ket{}_s \label{chapter12.eqn-schrodinger-to-heisengerg-ket}\\
		\hat{\Omega}_H(t) &= e^{\iu \hat{H}t/\hbar} \hat{\Omega}_S e^{-\iu \hat{H}t/\hbar}
		\label{chapter12.eqn-schrodinger-to-heisengerg-operator}
	\end{align}
	where we have assumed the system is conservative, i.e., $\hat{H}$ is independent of time. \\
	
	In the Heisenberg picture, the base kets, for example, the eigenkets of $\hat{Q}_H(t)$ and $\hat{P}_H(t)$ are time dependent. We have
	\begin{align}
		\hat{Q}_H(t) \ket{q,t}_H &= q \ket{q,t}_H \\
		\hat{P}_H(t) \ket{p,t}_H &= p \ket{p,t}_H
	\end{align}
	where
	\begin{align}
		\ket{q,t}_H &= e^{\iu \hat{H}t/\hbar}\ket{q}\\
		\ket{p,t}_H &= e^{\iu \hat{H}t/\hbar}\ket{p}
	\end{align}
	and
	\begin{align}
		\hat{Q}_H(t) &= e^{\iu \hat{H}t/\hbar} \hat{Q}_S e^{-\iu \hat{H}t/\hbar}\\
		\hat{P}_H(t) &= e^{\iu \hat{H}t/\hbar} \hat{P}_S e^{-\iu \hat{H}t/\hbar}
	\end{align}
	The orthogonality and completeness of the Heisenberg picture base kets are
	\begin{align}
		_H\braket{q,t}{q^\prime,t}_H = \braket{q}{q^\prime} &=  \delta(q-q^\prime) \\
		_H\braket{p,t}{p^\prime,t}_H = \braket{p}{p^\prime} &=  \delta(p-p^\prime) 
	\end{align}
	Note that these are equal time relations. And
	\begin{align}
		\hat{\mathbb{1}} &= \int \dd{q} \ket{q,t}_H \  {_H}\bra{q,t}\label{chapter12.eqn1-completeness-q}\\
		\hat{\mathbb{1}} &= \int \dd{p} \ket{p,t}_H\ {_H}\bra{p,t}\label{chapter12.eqn1-completeness-p}
	\end{align}
	To show the validity of equation (\ref{chapter12.eqn1-completeness-q}), for example, we use the transformation of kets and bras from the schr\"{o}dinger picture to the Heisenberg picture (\ref{chapter12.eqn-schrodinger-to-heisengerg-ket}), i.e., 
	\begin{equation}
		{_H}\bra{} = {_S}\bra{} e^{-\iu \hat{H}t/\hbar} 
	\end{equation}
	Thus the right hand side of equation (\ref{chapter12.eqn1-completeness-q}) can be written as
	\begin{align}
		\int \dd{q}  \ket{q,t}_H {_H}\bra{q,t} &= \int \dd{q} \quad e^{\iu \hat{H}t/\hbar}\ket{q} \bra{q}e^{-\iu \hat{H}t/\hbar} \\
		&= e^{\iu \hat{H}t/\hbar}\left(\int \dd{q}  \ket{q} \bra{q}\right)e^{-\iu \hat{H}t/\hbar} \\
		&= e^{\iu \hat{H}t/\hbar}\quad\mathbb{1}\quad e^{-\iu \hat{H}t/\hbar} \\
		&= \mathbb{1}
	\end{align}
	We note that the \underline{state vector} in the Heisenberg picture is independent of time, while the state vector in the Schr\"{o}dinger picture is time-dependent. This is very simply shown as follows:
	\begin{align}
		\ket{\psi}_H &= e^{\iu \hat{H}t/\hbar} \ket{\psi(t)}_S \\
		&= e^{\iu \hat{H}t/\hbar} e^{-\iu \hat{H}t/\hbar} \ket{\psi(0)}_S\\
		&= \ket{\psi(0)}_S
	\end{align}
	Thus the state ket in the Heisenberg picture is independent of time and is same as the initial state ket in the Schr\"{o}dinger picture.
	
	\end{enumerate}


	\section{Propagator}
	The dynamics of a quantum system is completely specified by the "Feynman Kernel", or the propagator or the transition amplitude defined as
	%%%%%%%%%%%%%
	%%%%%%%%%% TODO
	%%%%%%%%%%%%%
	\begin{equation}
		U(q_2, t_2; q_1, t_1) = {\_H}\braket{q_2, t_2}{q_1, t_1}_H
		\label{chapter12.eqn1-propagator}
	\end{equation}
	Transforming to the Schr\"{o}dinger picture base kets, we can write equation (\ref{chapter12.eqn1-propagator}) as
	\begin{align}
		U(q_2, t_2; q_1, t_2) 
		&= \bra{q_2}e^{-\iu \hat{H}t_2/\hbar} e^{\iu \hat{H}t_1/\hbar} \ket{q_1} \nonumber \\
		&= \bra{q_2}e^{-\iu \hat{H}(t_2-t_1)/\hbar} \ket{q_1} \label{chapter12.eqn2-propagator}
	\end{align}
	We see that the propagator is the matrix element in the coordinate basis of the time-evolution operator in the Schr\"{o}dinger picture. The physical meaning of the propagator is that it is the probability amplitude of finding the particle at $q_2$ at time $t_2$ if the particle was at $q_1$ at an earlier time $t_1$. Knowing the propagator is equivalent to solving the Schr\"{o}dinger equation, for it allows us to calculate the Schr\"{o}dinger picture wave function at any moment of time if the wave function is known at an earlier moment. This is shown below:
	\begin{align}
		\psi_S(q,t )
		&= \braket{q}{\psi_S(t)} \\
		&= \bra{q}e^{-\iu \hat{H}t/\hbar} \ket{\psi_S(0)} \\
		&= {\_H}\braket{q,t}{\psi}_H \\
		&= \int \dd{q^\prime} {\_H}\braket{q,t}{q^\prime, t^\prime}_H {\_H}\braket{q^\prime, t^\prime}{\psi}_H \\
		&= \int \dd{q^\prime} U(q,t;q^\prime, t^\prime) \psi_S(q^\prime,t^\prime)
	\end{align}
	The path integral formalism of quantum dynamics provides a means to construct the transition amplitude ${\_H}\braket{q,t}{q^\prime, t^\prime}_H$ from the classical Hamiltonian or Lagrangian alone, without any reference to non commuting operators or Hilbert space vectors.\\
	
	\section{Path Integral for the Propagator}
		We will now calculate
		\begin{equation}
			U(x,t;x_0, t_0) = {\_H}\braket{x,t}{x_0, t_0}
			\label{chapter12.eqn1-propagator-path-integral}
		\end{equation}
		where $t>t_0$. For this purpose let us devide the time integral $(t,t_0)$ into $N$ equal segments each of duration $\epsilon$. Namely, let
		\begin{equation}
			\epsilon = \frac{t-t_0}{N}
			\label{chapter12.eqn2-propagator-path-integral}
		\end{equation}
		In other words, we are discretizing the time interval, and in the end we will take the continuum limit $\epsilon\rightarrow 0$ and $N\rightarrow \infty$. We label the end times $t_0$ and $t$ and the intermediate times as $t_1, t_2, \ldots, t_{N-1}, t_N=t$. Further, we will let $x_N=x$. The intermediate times are 
		\begin{equation}
			t_i=t_0+i \epsilon, \quad i=1,2,\ldots,N-1
			\label{chapter12.eqn3-propagator-path-integral}
		\end{equation}
		
		At each intermediate time a complete set of basis states $\ket{x_i,t_i}$ may be inserted:
		\begin{equation}
			\braket{x t}{x_0 t_0} =\int_{-\infty}^{\infty}\dd{x_1} \ldots \int_{-\infty}^{\infty} \dd{x_{N-1}} \braket{x t}{x_{N-1} t_{N-1}}\braket{x_{N-1} t_{N-1}}{x_{N-2} t_{N-2}} \ldots \braket{x_2 t_2}{x_{1} t_{1}} \braket{x_1 t_1}{x_{0} t_{0}}
			\label{chapter12.eqn4-propagator-path-integral}
		\end{equation}
		
		Here we have omitted the subscript $H$ in the Heisenberg picture basis vectors since there is no scope for confusion. Note that while there are $N$ scalar products in euquation (\ref{chapter12.eqn4-propagator-path-integral}), there are only $N-1$ intermediate points so that the number of integrations is $N-1$. Since $x_N = x$ and $t_N= t$, we can write equation (\ref{chapter12.eqn4-propagator-path-integral}) as
		\begin{equation}
			\braket{x t}{x_0 t_0} = \int \prod_{i=1}^{N-1} \dd{q_i} \prod_{i=0}^{N-1} \braket{x_{i+1} t_{i+1}}{x_i t_i}
			\label{chapter12.eqn5-propagator-path-integral}
		\end{equation}
		Equation (\ref{chapter12.eqn5-propagator-path-integral}) can be interpreted as follows: A particle that propagates from $x_0$ at time $t_0$ to $x$ at time $t$ can take an arbitrary intermediate trajectory, shown in figure (\ref{chapter12.fig1-propagator-path-integral-trajectory})
		\begin{figure}
			\centering
			\caption{text}
			\label{chapter12.fig1-propagator-path-integral-trajectory}
		\end{figure}
	Such a path is characterized by the coordinate values $x_i$ at intermediate grid @@@@ in the time interval $(t_0,t)$. One such path is shown in the figure as a zigzag curve. Since each intermediate coordinates $x_i\quad (i=1,2,\ldots,N-1)$ can vary form $-\infty$ to $\infty$, it is essential that all conceivable paths connecting the end points are taken into account. According tot the representation principle of Quantum Mechanics they all contribute to the transition amplitude (\ref{chapter12.eqn5-propagator-path-integral}). Of course, some trajectories may turn out to be more important than others.\\
	
	We will now calculate the intermediate scalar products which themselves are propagators but over infinitesimal time intervals. An intermediate scalar product has the form $\braket{x_{i+1} t_{i+1}}{x_i t_i}$. We can calculate this inner product up to first order in $\epsilon$ from equation (\ref{chapter12.eqn2-propagator-path-integral}) as follows
	\begin{align}
		\braket{x_{i+1} t_{i+1}}{x_i t_i} &= \bra{x_{i+1}} e^{-\iu \hat{H}t_{i+1}/\hbar}e^{\iu \hat{H}t_{i}/\hbar}\ket{x_i}\\
		&= \bra{x_{i+1}} e^{-\iu \hat{H}(t_{i+1} - t_i)/\hbar}\ket{x_i}\\
		&= \bra{x_{i+1}} e^{-\iu \hat{H}\epsilon/\hbar}\ket{x_i}\\
		&= \bra{x_{i+1}} \left(\mathbb{1} - \frac{\iu \epsilon}{\hbar}\hat{H} + \order{\epsilon^2}\right)\ket{x_i}
	\end{align}
	We will take $\hat{H}$ to be of the form
	\begin{equation}
		\hat{H} = \frac{\hat{P}}{2 m} + V(\hat{X})
	\end{equation}
	Therefore
	\begin{align}
		\braket{x_{i+1} t_{i+1}}{x_i t_i} 
		&= \bra{x_{i+1}}\left[\mathbb{1} - \frac{\iu \epsilon}{\hbar}\left(\frac{\hat{P}^2}{2 m} + V(\hat{X}) + \right) + \order{\epsilon^2}\right]\ket{x_i}\\
		&= \int_{-\infty}^{\infty} \dd{p} \braket{x_{i+1}}{p} \bra{p} \left[\mathbb{1} - \frac{\iu \epsilon}{\hbar}\left(\frac{\hat{P}^2}{2 m} + V(\hat{X})\right) + \order{\epsilon^2}\right] \ket{x_i} \\
		&= \int_{-\infty}^{\infty} \dd{p} \braket{x_{i+1}}{p} \braket{p}{x_i} \left[\mathbb{1} - \frac{\iu \epsilon}{\hbar}\left(\frac{p^2}{2 m} + V(x_i)\right) + \order{\epsilon^2}\right]  \\
		&= \int \dd{p} \frac{1}{\sqrt{2\pi \hbar}}e^{\iu p x_{i+1}/\hbar}\frac{1}{\sqrt{2\pi \hbar}}e^{-\iu p x_{i}/\hbar} e^{-\iu \epsilon/\hbar \left( \frac{p^2}{2 m} + V(x_i) \right)} \\
		&= \frac{1}{2\pi\hbar} \int_{-\infty}^{\infty} \dd{p} e^{\iu p (x_{i+1} - x_i)/\hbar} e^{-\iu \epsilon/\hbar \left( \frac{p^2}{2 m} + V(x_i) \right)} \\
		&= \frac{1}{2\pi\hbar} e^{-\iu \epsilon V(x_i)/\hbar} \int_{-\infty}^{\infty} e^{\iu p \epsilon\frac{(x_{i+1} - x_i)}{\epsilon}\frac{1}{\hbar}} e^{-\iu \epsilon/\hbar \left( \frac{p^2}{2 m}\right)} \dd{p}\\
		&= \frac{1}{2\pi\hbar} e^{-\iu \epsilon V(x_i)/\hbar} \int_{-\infty}^{\infty} e^{\iu p \epsilon \dot{x}_i / \hbar} e^{-\iu \epsilon \frac{p^2}{2 m \hbar}} \dd{p} \\
		&= \frac{1}{2\pi\hbar} e^{-\iu \epsilon V(x_i)/\hbar} \int_{-\infty}^{\infty} e^{-\frac{\iu \epsilon}{2 m \hbar} \left(p^2 - 2 m p \dot{x}_i \right)} \dd{p}
		\label{chapter12.eqn6-path-integral-propagator}
	\end{align}
	In the above we have defined
	\begin{equation}
		\dot{x}_i = \frac{x_{i+1} x_i}{\epsilon}
	\end{equation}
	Now
	\begin{equation}
		p^2 - 2m p \dot{x}_i = (p- m \dot{x}_i)^2 - m^2 \dot{x}_i^2
	\end{equation}
	We make the change of variable
	\begin{equation}
		p^\prime = p -m \dot{x}_i
	\end{equation}
	Therefore equation (\ref{chapter12.eqn6-path-integral-propagator}) can be written as
	
	\begin{align}
		\braket{x_{i+1} t_{i+1}}{x_i t_i} 
		&= \frac{1}{2 \pi \hbar} e^{-\iu \epsilon V(x_i)/\hbar} e^{-\iu \epsilon (-m^2 \dot{x}_i^2) / 2 m \hbar}  \int_{-\infty}^{\infty} \dd{p^\prime} e^{-\iu \epsilon {p^\prime}^2 / 2 m \hbar} \\
		&= \frac{1}{2 \pi \hbar} e^{\frac{\iu \epsilon}{\hbar} \left(\frac{1}{2} m \dot{x}_i^2 - V(x_i)\right)} \int_{-\infty}^{\infty} \dd{p^\prime} e^{-\iu \epsilon {p^\prime}^2 / 2 m \hbar}
	\end{align}
	Now we use the standard integral
	\begin{equation}
		\int_{-\infty}^{\infty} e^{-\alpha x^2} \dd{x} = \sqrt{\frac{\pi}{\alpha}}
	\end{equation}
	to get
	\begin{equation}
		\int_{-\infty}^{\infty} \dd{p^\prime} e^{-\iu \epsilon p^\prime / 2 m \hbar} = \left(\frac{\pi}{\iu \epsilon/2 m \hbar}\right)^{1/2} = \left(\frac{2 \pi \hbar m}{\iu \epsilon}\right)^{1/2}
	\end{equation}
	Therefore
	\begin{align}
		\braket{x_{i+1} t_{i+1}}{x_i t_i} 
		&= \frac{1}{2 \pi \hbar} \left(\frac{2 \pi \hbar m}{\iu \epsilon}\right)^{1/2} e^{\frac{\iu \epsilon}{\hbar} \left(\frac{1}{2} m \dot{x}_i^2 - V(x_i)\right)} \\
		&= \left(\frac{m}{2\pi\hbar\iu \epsilon}\right)^{1/2} e^{\frac{\iu \epsilon}{\hbar} \left(\frac{1}{2} m \dot{x}_i^2 - V(x_i)\right)}
		\label{chapter12.eqn7-path-integral-propagator}
	\end{align}
	We now substitute equation (\ref{chapter12.eqn7-path-integral-propagator}) into equation (\ref{chapter12.eqn5-propagator-path-integral}) to get
	\begin{align}
		\braket{x t}{x_0 t_0} 
		&= \int \prod_{i=1}^{N-1} \dd{x_i} \prod_{i=0}^{N-1} \braket{x_{i+1} t_{i+1}}{x_i t_i}\\
		&= \int \prod_{i=1}^{N-1} \dd{x_i} \prod_{i=0}^{N-1} \left(\frac{m}{2 \pi \hbar \iu \epsilon}\right)^{1/2} e^{\frac{\iu \epsilon}{\hbar} \left(\frac{1}{2} m \dot{x}_i^2 - V(x_i)\right)} \\
		&= \left(\frac{m}{2 \pi \hbar \iu \epsilon}\right)^{N/2} \int \prod_{i=1}^{N-1} \dd{x_i}e^{\frac{\iu}{\hbar} \sum_{i=0}^{N-1} \epsilon \left(\frac{1}{2} m \dot{x}_i^2 - V(x_i)\right)}
		\label{chapter12.eqn8-path-integral-propagator}
	\end{align}
	We now consider a path $x(t^\prime)$ connecting the initial and the final space-time point such that the value of $x(t^\prime)$ at the intermediate times $t_1, t_2, \ldots, t_{N-1}$ are $x(t_i^\prime)=x_i$. Therefore we can write
	\begin{align}
		\sum_{i=0}^{N-1} \epsilon \left(\frac{1}{2} m \dot{x}_i^2 - V(x_i)\right) 
		&= \int_{t_0}^{t} \left[\frac{1}{2} m \dot{x}^2(t^\prime) - V(x(t^\prime))\right] \dd{t^\prime} \\
		&= \int_{t_0}^{t} L\left(x(t^\prime), \dot{x}(t^\prime)\right) \dd{t^\prime} \\
		&= S[x(t^\prime)]
	\end{align}
	Where $S\left[x(t^\prime)\right]$ is the action calculated along the particular path. Since we are integrating over $x_i\quad (i=1,\ldots,N-1)$, we are effectively summing the exponential in equation (\ref{chapter12.eqn8-path-integral-propagator}) over all conceivable paths connecting $(x_0, t_0)$ to $(x, t)$. We define the path integral as
	\begin{equation}
		\mathcal{D}[x(t^\prime)] = \lim\limits_{N\rightarrow \infty} \left(\frac{m}{2 \pi \hbar \iu \epsilon}\right)^{N/2} \int \prod_{i=1}^{N-1} \dd{x_i}
		\label{chapter12.eqn9-path-integral-propagator}
	\end{equation}
	Therefore, we can write equation (\ref{chapter12.eqn8-path-integral-propagator}) as
	\begin{equation}
		\braket{x, t}{x_0, t_0} = \int \mathcal{D}\left[x(t^\prime)\right] e^{\frac{i}{\hbar} S\left[x(t^\prime)\right]}
		\label{chapter12.eqn10-path-integral-propagator}
	\end{equation}
	This is the path integral formula for the propagator. We can think of equation (\ref{chapter12.eqn10-path-integral-propagator}) as a sysbolic way of writing equation (\ref{chapter12.eqn8-path-integral-propagator}) with $N \rightarrow \infty$. In calculating path integrals we use equation (\ref{chapter12.eqn8-path-integral-propagator}) and set $N \rightarrow \infty$.
	
	\section{Path Integral for a Free Particle}
	For a free particle $V=0$. Therefore the lagrangian is
	\begin{equation}
		L = T - V = T = \frac{1}{2} m \dot{x}^2(t)
		\label{chapter12.eqn11-path-integral-free-particle}
	\end{equation}
	the path integral formula for the propagator of a free particle is
	\begin{align}
		\braket{x, t}{x_0, t_0} 
		&= \int \mathcal{D}\left[x(t^\prime)\right] \exp\left[\frac{\iu}{\hbar} S\left[x(t^\prime)\right]\right] \\
		&= \lim\limits_{N \rightarrow \infty} \left(\frac{m}{2 \pi \hbar \iu \epsilon}\right)^{N/2} \int \prod_{i=1}^{N-1} \dd{x_i} \exp\left[\frac{\iu \epsilon}{\hbar} \sum_{i=0}^{N-1} \frac{1}{2} m \dot{x}_i^2\right]
		\label{chapter12.eqn12-path-integral-free-particle}
	\end{align}
	In equation (\ref{chapter12.eqn12-path-integral-free-particle})
	\begin{equation}
		\epsilon = \frac{t - t_0}{N}
	\end{equation}
	Also $\dot{x}_i$ can be written as
	\begin{equation}
		\dot{x}_i = \frac{x_{i+1} - x_i}{\epsilon}
		\label{chapter12.eqn13-path-integral-free-particle}
	\end{equation}
	
	\begin{figure}
		\centering
		\caption{text}
		\label{chapter12.fig2}
	\end{figure}
	For notational convenience we let $x_N = x$ where $x$ is the final position. We only integrate over the position the particle may have at intermediate times $t_1, t_2, \ldots, t_{N-1}$.\\
	
	Using equation (\ref{chapter12.eqn13-path-integral-free-particle}), equation (\ref{chapter12.eqn12-path-integral-free-particle}) can be written as
	\begin{align}
		\braket{x t}{x_0 t_0} 
		&= \lim\limits_{N \rightarrow \infty} \left(\frac{m}{2\pi \hbar \iu \epsilon}\right)^{N/2} \int \prod_{i=1}^{N-1} \dd{x_i} e^{\frac{\iu \epsilon}{\hbar} \sum_{i=0}^{N-1} \frac{m}{2} \left(\frac{x_{i+1} - x_i}{\epsilon}\right)^2} \\
		&= \lim\limits_{N \rightarrow \infty} \left(\frac{m}{2\pi \hbar \iu \epsilon}\right)^{N/2} \int \prod_{i=1}^{N-1} \dd{x_i} e^{\frac{\iu m}{2\hbar \epsilon} \sum_{i=0}^{N-1} (x_{i+1} - x_i)^2}
		\label{chapter12.eqn14-path-integral-free-particle}
	\end{align}
	At this stage it is convenient to make a change of variable
	\begin{equation}
		y_i = \left(\frac{m}{2\hbar \epsilon}\right)^{1/2} x_i
	\end{equation}
	In terms of new variables equation (\ref{chapter12.eqn14-path-integral-free-particle}) is written as
	\begin{equation}
		\braket{x t}{x_0 t_0} 
		= \lim\limits_{N \rightarrow \infty} \left(\frac{m}{2\pi \hbar \iu \epsilon}\right)^{N/2} \left(\frac{2\hbar \epsilon}{m}\right)^{(N-1)/2} \int \prod_{i=1}^{N-1} \dd{y_i} e^{-\sum_{i=0}^{N-1} \frac{(y_{i+1} - y_i)^2}{i}}
		\label{chapter12.eqn15-path-integral-free-particle}
	\end{equation}
	We now have to do the gaussian integral over the variables $y_1, y_2, \ldots, y_{N-1}$.\\
	
	\textbf{$y_1$ integral}\\
	\begin{equation}
		I_1 = \int_{-\infty}^{\infty} \dd{y_1} \exp\left[-\frac{1}{\iu} \left[(y_1 - y_0)^2 + (y_2 - y_1)^2\right]\right]
	\end{equation}
	consider the exponent
	\begin{equation}
		(y_1 - y_0)^2 + (y_2 - y_1)^2 = 2 y_1^2 - 2 (y_0 + y_2) y_1 + (y_0^2+y_2^2)
	\end{equation}
	therefore,
	\begin{equation}
		I_1 = \exp \left[-\frac{1}{\iu} \left(y_0^2+ y_2^2\right)\right] \int_{-\infty}^{\infty} \dd{y_1} \exp\left[-\frac{1}{\iu} \left(2y_1^2 - 2(y_0 + y_2) y_1\right)
		\right]
	\end{equation}
	Now we use the standard integral
	\begin{equation}
		\int_{-\infty}^{\infty} e^{-\alpha x^2 + \beta x} \dd{x} = \left(\frac{\pi}{\alpha}\right)^{1/2} \exp \left(\frac{\beta^2}{4\alpha}\right)
	\end{equation}
	choosing $\alpha = \frac{2}{\iu}$ and $\beta = \frac{2(y_0 + y_2)}{\iu}$
	\begin{align*}
		I_1 
		&= \exp \left[-\frac{1}{\iu} (y_0^2 + y_2^2)\right] \left(\frac{\iu \pi}{2}\right)^{1/2} \exp\left[\frac{-4 \left(y_0 +y_2\right)^2}{4 (2/\iu)}\right]\\
		&= \exp \left[-\frac{1}{\iu} (y_0^2 + y_2^2)\right] \left(\frac{\iu \pi}{2}\right)^{1/2} \exp\left[\frac{\left(y_0 +y_2\right)^2}{2\iu}\right] \\
		&= \left(\frac{\iu \pi}{2}\right)^{1/2} \exp \left[-\frac{1}{2\iu} \left( 2(y_0^2 + y_2^2) - \left(y_0 +y_2\right)^2 \right)\right]
	\end{align*}
	Thus
	\begin{equation}
		I_1 = \left(\frac{\iu \pi}{2}\right)^{1/2} \exp \left[-\frac{1}{2\iu} \left(y_2 - y_0\right)^2\right]
		\label{chapter12.eqn16-path-integral-free-particle}
	\end{equation}
	Next we do the integral over $y_2$. THe variable $y_2$ occurs in the $i=2$ term in equation (\ref{chapter12.eqn15-path-integral-free-particle}) and also in $I_1$ in equation (\ref{chapter12.eqn16-path-integral-free-particle}). Therefore, the $y_2$ integral is
	\begin{align*}
		I_2 
		&= \int \dd{y_2} \exp \left[-\frac{1}{\iu} \left(y_3 - y_2\right)^2\right] \left(\frac{\iu \pi}{2}\right)^{1/2} \exp \left[-\frac{1}{2 \iu} \left(y_2 - y_0\right)^2\right] \\
		&= \left(\frac{\iu \pi}{2}\right)^{1/2} \int \dd{y_2} \exp \left[-\frac{1}{\iu} \left(y_3 - y_2\right)^2 -\frac{1}{2 \iu} \left(y_2 - y_0\right)^2\right] \\
		&= \left(\frac{\iu \pi}{2}\right)^{1/2} \int \dd{y_2} \exp \left[-\frac{1}{\iu} \{\left(y_3 - y_2\right)^2 + \left(y_2 - y_0\right)^2\}\right] \\
	\end{align*}
	Consider the term in the curly brackets
	\begin{align*}
		2\left(y_3 - y_2\right)^2 + \left(y_2 - y_0\right)^2 &= 2\left(y_3^2 + y_2^2 - 2y_2 y_3\right) + \left(y_2^2 + y_0^2 - 2 y_0 y_2\right) \\
		&= 3 y_2^2 - 2y_2 \left(2y_3 + y_0\right) + \left(2 y_3^2 + y_0^2\right)
	\end{align*}
	where the first term is quadratic in $y_2$, second term is linear in $y_2$ and the last term is independent of $y_2$. We have
	\begin{equation}
		I_2 = \left(\frac{\iu \pi}{2}\right)^{1/2} \exp\left[-\frac{1}{2\iu}\left(2 y_3^2 + y_0^2\right)\right] \int \dd{y} \exp\left[-\frac{1}{2\iu} \left(3 y_2^2 - 2y_2 \left(2 y_3 + y_0\right)\right) \right]
	\end{equation}
	
	We use standard integral (\ref{appendix1.eqn2}) from appendix (\ref{appendix1.Integrals}) and choose
	\begin{align}
		\alpha &= \frac{3}{2\iu} \\
		\beta &= \frac{\left(y_0 2 y_3\right)}{\iu}
	\end{align}
	
	\begin{align*}
		I_2 
		&= \left(\frac{\iu \pi}{2}\right)^{1/2} \exp\left[-\frac{1}{2\iu} \left(2 y_3^2 + y_0^2\right)\right] \left(\frac{2\pi \iu}{3}\right)^{1/2} \exp\left[-\frac{\left(y_0 + 2 y_3\right)^2}{6/\iu}\right] \\
		&= \left(\frac{\iu \pi}{2}\right)^{1/2} \left(\frac{2\pi \iu}{3}\right)^{1/2}  \exp\left[-\frac{1}{2\iu} \left(2 y_3^2 + y_0^2\right)\right]  \exp\left[\frac{\left(y_0 + 2 y_3\right)^2}{6\iu}\right] \\
		&= \qty(\frac{\iu^2 \pi^2}{3})^{1/2} \exp\left[-\frac{1}{\iu}\qty{\frac{1}{3} y_3^2 + \frac{1}{3} y_0^2 -\frac{2}{3} y_0 y_3}\right]
	\end{align*}
	\begin{equation}
		I_2 = \qty(\frac{\iu^2 \pi^2}{3})^{1/2} \exp\left[-\frac{(y_3-y_0)^2}{3\iu}\right]
		\label{chapter12.eqn17-path-integral-free-particle}
	\end{equation}
	Now the trend is clear. Finally, integrating $(N-1)$ times we get
	\begin{equation}
		I_{N-1} = \frac{\left(\iu \pi\right)^{(N-1)/2}}{N^{1/2}} \exp\left[-\left(y_N - y_0\right)^2/N\iu
		\right]
	\end{equation}
	where $y_N = y$.\\

	Therefore, the path integral formula for the propagator of a free particle is (using the above formula in equation (\ref{chapter12.eqn15-path-integral-free-particle}))
	\begin{align}
		\braket{x t}{x_0 t_0}_{free} = U_{free}\left(x, t; x_0, t_0\right) = \lim\limits_{N\rightarrow \infty} \left(\frac{m}{2\pi \hbar \iu \epsilon}\right)^{N/2} \left(\frac{2\hbar\epsilon}{m}\right)^{(N-1)/2}  \frac{(\iu \pi)^{(N-1)/2}}{N^{1/2}} \exp\left[-\left(y_N - y_0\right)^2 / N\iu
		\right]
	\end{align}
	previously we defined
	\begin{equation}
		y = \left(\frac{m}{2\hbar \epsilon}\right)^{1/2} x
	\end{equation}
	Therefore
	\begin{align*}
		U(x, t; x_0, t_0) 
		&= \lim\limits_{N\rightarrow \infty} \left(\frac{m}{2\pi\hbar\iu\epsilon}\right)^{N/2} \left(\frac{2\hbar \epsilon}{m}\right)^{(N-1)/2} \frac{(\iu \pi)^{(N-1)/2}}{N^{1/2}} \exp \left[-\frac{m(x_N - x_0)^2}{2\hbar \epsilon N \iu}\right] \\
		&= \lim\limits_{N\rightarrow \infty} \left(\frac{m}{2\pi\hbar\iu N \epsilon}\right)^{N/2} \left(\frac{2\pi\hbar N \iu \epsilon}{m}\right)^{(N-1)/2} \exp \left[-\frac{m(x_N - x_0)^2}{2\hbar \epsilon N \iu}\right]
	\end{align*}
	Now $\lim\limits_{N\rightarrow \infty} N\epsilon = t-t_0 \equiv \Delta t$,
	also $x_N = x$.\\
	
	Hence
	\begin{equation}
		U(x,t; x_0 t_0) = \left(\frac{m}{2\pi \hbar \iu (t-t_0)}\right)^{1/2} \exp\left[-\frac{m(x-x_0)^2}{2\hbar\iu (t-t_0)}\right]
		\label{chapter12.eqn18-path-integral-free-particle}
	\end{equation}
	This is the propagator for a free particle obtained using the path integral formula.
	
	\subsection{check of calculation}
	We have
	\begin{equation}
		\lim\limits_{ t \rightarrow t_0} \braket{x t}{x_0 t_0} = \delta(x-x_0)
	\end{equation}
	Therefore equation (\ref{chapter12.eqn18-path-integral-free-particle}) must reduce to the delta function $\delta(x-x_0)$ when $t=t_0$. Taking $\Delta = \sqrt{\frac{2\hbar \iu (t-t_0)}{m}}$ in equation (\ref{chapter12.eqn18-path-integral-free-particle}) we can write
	\begin{equation}
		U_{free}\qty(x, t; x_0 t_0) = \frac{1}{\pi^{1/2} \Delta} \exp\left[-(x-x_0)^2/\Delta^2\right]
	\end{equation}
	In the limit $t\rightarrow t_0$, $\Delta \rightarrow 0$. Therefore
	\begin{align}
		\lim\limits_{t\rightarrow t_0} U_{free}(x, t; x_0 t_0) &= \lim\limits_{\Delta \rightarrow 0} \frac{1}{\pi^{1/2} \Delta} \exp\left[-(x-x_0)^2 / \Delta^2\right] \\
		&= \delta\qty(x-x_0)
	\end{align}
	Thus, the free particle propagator (equation (\ref{chapter12.eqn18-path-integral-free-particle})) has the correct limiting behavior in the limit $t\rightarrow t_0$.
	
	\section{Derivation of the propagator for a free particle without using the path integral formula}
	Since for a free particle, the Hamiltonian is simple and its eigenvalues and eigenvectors are known, we can find the propagator $U\qty(x, t; x_0, t_0)$ without using the path integral formula. We now calculate the propagator for a free particle directly without using the path integral formula.\\
	
	
	The propagator is 
	\begin{align}
		\braket{x t}{x_0 t_0} &\equiv U\qty(x, t; x_0, t_0) \\
		&= \bra{x} e^{-\iu \hat{H}(t-t_0)/\hbar}\ket{x_0} \\
		&= \int \dd{p} \bra{x} e^{-\iu \hat{H}(t-t_0)/\hbar}\ket{p}\braket{p}{x_0} \\
		&= \int \dd{p} \bra{x} e^{-\iu \frac{\hat{p}^2\qty(t-t_0)}{2 m \hbar}}\ket{p}\braket{p}{x_0} \\
		&= \int \dd{p} e^{-\iu \frac{p^2\qty(t-t_0)}{2 m \hbar}}\braket{x}{p}\braket{p}{x_0} \\
		&= \int \dd{p} e^{-\iu p^2 \Delta t /2 m \hbar} \frac{1}{\sqrt{2\pi\hbar}} e^{\iu p x/\hbar} \frac{1}{\sqrt{2\pi\hbar}} e^{\iu p x_0/\hbar} \\
		&= \frac{1}{2\pi\hbar} \int_{-\infty}^{\infty} \dd{p} \exp\qty[-\iu p^2 \Delta t/2 m \hbar + \iu p \qty(x-x_0)/\hbar]
	\end{align}
	Using standard integral from appendix (equation \ref{appendix1.eqn2}) and 
	\begin{align}
		\alpha &= \frac{\iu \Delta t}{2 m \hbar} \\
		\beta &= \frac{\iu \qty(x-x_0)}{\hbar}
	\end{align}
	\therefore
	\begin{align*}
		U\qty(x,t'x_0,t_0) 
		&= \qty(\frac{1}{2\pi \hbar}) \qty(\frac{\pi}{\iu \Delta t / 2 m \hbar})^{1/2} e^{-\frac{\qty(x-x_0)^2/\hbar^2}{4\qty(\iu \Delta t /2 m \hbar)}} \\
		&= \frac{1}{2\pi \hbar} \qty(\frac{2\pi \hbar m}{\iu \Delta t})^{1/2} e^{-m\qty(x-x_0)^2/2 \hbar \iu \Delta t} \\
		&= \qty(\frac{m}{2\pi \hbar \iu \Delta t})^{1/2} e^{-m\qty(x-x_0)^2/2 \hbar \iu \Delta t} 
	\end{align*}
	That is,
	\begin{equation}
		U\qty(x,t;x_0,t_0) = \qty(\frac{m}{2\pi\hbar \iu \qty(t-t_0)})^{1/2} e^{-\frac{m\qty(x-x_0)^2}{2\hbar \iu \qty(t-t_0)}}
	\end{equation}
	Which is the same result we obtained earlier by using the path integral formula.
	
	
	\section{The Classical Action}
		Suppose a particle propagates from $x_0$ at time $t_0$ to $x$ at a later time $t$. Of all the conceivable paths from $\qty(x_0, t_0)$ to $\qty(x,t)$, there is one path for which the action is minimum. This path is called the classical path and the minimum value of the action along the classical is called the classical action.
		
		
		%%% figure
		\begin{figure}
			\caption{Classical action}
		\end{figure}
	
		For a free particle, the classical path is the straight line connecting the points $\qty(t_0, x_0)$ to $\qty(t, x)$ in the space-time diagram.
		
		The equation for the classical path is then
		\begin{equation}
			x_{cl}\qty(t^\prime) = x_0 + \frac{\qty(x-x_0)}{\qty(t-t_0)} \qty(t^\prime - t_0)
		\end{equation}
		Here $t$ and $t_0$ are fixed times and $t$'s is the running variable. From the above equation we have
		\begin{equation}
			\dot{x}_{cl}\qty(t^\prime) = \frac{x-x_0}{t-t_0} = constant
		\end{equation}
		Therefore, the classical action for a free particle is (for a free particle $V=0$ so we have $L=T-V=T=\frac{1}{2} m \dot{x}^2$)
		\begin{align}
			S_{cl} 
			&= S\qty[x_{cl}\qty(t^\prime)] \\
			&= \int_{t_0}^{t} L(x_{cl}, \dot{x}_{cl}) \dd{t^\prime} \\
			&= \int_{t_0}^{t} \frac{1}{2} m \dot{x}_{cl}\qty(t^\prime) \dd{t^\prime} \\
			&= \frac{1}{2} m \qty(\frac{x-x_0}{t-t_0})^2 \qty(t-t_0) \\
			&= \frac{m \qty(x-x_0)^2}{2 \qty(t-t_0)}
		\end{align}
		Now, the free propagator is
		\begin{equation}
			U(x,t;x_0, t_0) = \frac{m}{2\pi \hbar \iu \qty(t-t_0)} e^{-\frac{m\qty(x-x_0)^2}{2\hbar \iu \qty(t-t_0)}}
		\end{equation}
		In terms of classical action we can write
		\begin{equation}
			U(x,t;x_0,t_0) = \qty(\frac{m}{2\pi \hbar \iu\qty(t-t_0)})^{1/2} e^{\frac{i}{\hbar} S_{cl}}
		\end{equation}
		
		\section{Discussion on Path Integral}
		\textbf{Reference: Shankar}
		\subsection{Principle of Least Action}
		If a particle moves from $x_0$ at time $t_0$ to a different point $x$ at a later time $t$, then of all the paths between the points $\qty(x_0, t_0)$ to $\qty(x,t)$, a classical particle takes the path for which the action is minimum. This is called the principle of least action
		
		%%% FIGURE
		\begin{figure}
			\caption{least action}
		\end{figure}
		Suppose a particle follows the path $x(t^\prime)$. Then the action for this path is
		\begin{equation}
			S\qty[x(\qty(t^\prime))] = \int_{t_0}^{t} L(x(t^\prime), \dot{x}(t^\prime)) \dd{t^\prime}
		\end{equation}
		Next consider a slightly varied path
		
		\begin{equation}
			x\qty(t^\prime) + \eta\qty(t^\prime)
		\end{equation}
		where $\eta$ is very small and 
		\begin{equation}
			\eta\qty(t_0) = \eta\qty(t) = 0
		\end{equation}
		Then the action for the varied path is 
		\begin{equation}
			S\qty[x(t^\prime) + \eta\qty(t^\prime)] = \int_{t_0}^{t} L\qty(x\qty(t^\prime) + \eta\qty(t^\prime), \dot{x}\qty(t^\prime) + \dot{\eta}\qty(t^\prime)) \dd{t^\prime}
		\end{equation}
		Then upto first order in $\eta\qty(t^\prime)$, the variation of the action is
		\begin{align*}
			\var{S\qty[x\qty(t^\prime)]} 
			&= S\qty[x\qty(t^\prime) + \eta\qty(t^\prime)] - S\qty[x\qty(t^\prime)] \\
			&= \int_{t_0}^{t} \qty[\pdv{L\qty(x, \dot{x})}{x} \eta\qty(t^\prime)  + \pdv{L\qty(x, \dot{x})}{\dot{x}} \dot{\eta}\qty(t^\prime)] \dd{t^\prime} \\
			&= \int_{t_0}^{t} \qty[\pdv{L}{x}\eta\qty(t^\prime) + \dv{t^\prime}\qty(\pdv{L}{\dot{x}} \eta\qty(t^\prime))  - \dv{t^\prime}\qty(\pdv{L}{\dot{x}}) \eta\qty(t^\prime)] \dd{t^\prime} \\
			&= \int_{t_0}^{t} \qty[\pdv{L}{x}  -
			 \dv{t^\prime}\qty(\pdv{L}{\dot{x}})] \eta\qty(t^\prime) \dd{t^\prime}
			 + \int_{t_0}^{t} \qty[
			 \dv{t^\prime}\qty(\pdv{L}{\dot{x}} \eta\qty(t^\prime))] \dd{t^\prime}
		\end{align*}
		The second term on the right hand side of the above equation is zero:
		\begin{equation}
			\int_{t_0}^{t} \dv{t^\prime}\qty(\pdv{L}{\dot{x}} \eta\qty(t^\prime)) \dd{t^\prime} = \eval{\pdv{L}{\dot{x}}\eta\qty(t^\prime)}_{t^\prime=t_0}^{t^\prime=t} = 0
		\end{equation}
		Since $\eta\qty(t) = \eta\qty(t^\prime) =0$. Thus we have
		\begin{equation}
			\var{S\qty[x\qty(t^\prime)]}  = \int_{t_0}^{t} \qty[\pdv{L}{x}  -
			\dv{t^\prime}\qty(\pdv{L}{\dot{x}})] \eta\qty(t^\prime) \dd{t^\prime} + \order{\eta^2}
		\end{equation}
		Now, if the path is the classical path, i.e., $x(t^\prime) = x_{cl}(t^\prime)$, Then $\var{S} = 0$ up to first order in $\eta$. Therefore we must have
		\begin{equation}
			\int_{t_0}^{t} \qty[\pdv{L}{x}  -
			\dv{t^\prime}\qty(\pdv{L}{\dot{x}})]_{cl} \eta\qty(t^\prime) \dd{t^\prime} = 0
		\end{equation}
		Since $\eta(t^\prime)$ is arbitrary except at the end times $t_0$ and $t$, we must have
		\begin{equation}
			\qty[\pdv{L}{x}  -
			\dv{t^\prime}\qty(\pdv{L}{\dot{x}})]_{cl} = 0
		\end{equation}
		Thus the variation $\var{S}$ from the classical path is 
		\begin{equation}
			\var{S} = \order{\eta^2}
		\end{equation}
		
		
		
		\section{Discussion on The Phase of the Path Integral}
		We derived previously
		\begin{equation}
			U\qty(x,t;x_0, t_0) = \int \mathcal{D}\qty[x\qty(t^\prime)] e^{\frac{\iu}{\hbar} S\qty[x\qty(t^\prime)]}
		\end{equation}
		Every path contributes a phase factor in the path integral, the phase being $\frac{\iu}{\hbar} S\qty[x\qty(t^\prime)]$ where $x\qty(t^\prime)$ is a particular path between the points $\qty(t_0, x_0)$ and $\qty(t,x)$ in the space-time diagram. Historically we can write
		\begin{equation}
			U = \sum_{all\ paths} e^{\frac{\iu}{\hbar} S\qty[x\qty(t^\prime)]}
		\end{equation}
		The most surprising this about the path integral is that every path, including the classical path $x_{cl}\qty(t^\prime)$ gets the same weight, that is to say a complex number of unit modules.\\
		
		
		Of all the paths, there is a special path, called the classical path for which $S$ is minimum or stationary. A slight change in path from the classical one does not change the action, more precisely the change in action is only of second order in the change of path.
		
		
		%%% figure 
		\begin{figure}
			\caption{paths}
		\end{figure}
	
	
	Consider a path $x_a\qty(t^\prime)$ far away from the classical path. Its contribution to the path integral is
	\begin{equation}
		Z_a = e^{\iu S\qty[x_a\qty(t^\prime)] / \hbar}
	\end{equation}
	While doing the path integral if we vary the path from $x_a\qty(t^\prime)$ to a neighboring one, there will be slightly change in the action. But, there will be a large change in the phase $S/\hbar$, since $\hbar$ is small. So, for paths well away from the classical path, contributions cancel because of the large change in phase from one path to the next. However, the situation is different for the classical path and the bundle of paths close to it. Here the action is stationary and so the phase of each of the paths near the classical path is about the same. In other words, the paths in the neighborhood of the classical path contribute constructively to the path integral.\\
	
	
	
	Thus the propagator $U$ is dominated by the paths near the classical path. The classical path is important not because it contributes a lot by itself, but because the paths in the vicinity of the classical path contribute coherently.\\
	
	
	How far from the classical path must we deviate before destructive interference sets in? One may say crudely that coherence would be lost once the phase differs from the stationary value $\frac{1}{\hbar} S\qty[x_{cl}(t^\prime)]$ by about $\pi$, i.e., if the action changes from the classical action by about $\pi \hbar$. For a macroscopic partivle this means a very tight constraint on its path since $S_{cl}$ is typically of the order of $1\ erg\ sec \approx 10^{27} \hbar$[\footnote{$\hbar = 1.0546 \times 10^-27 \ erg\ sec = 1.0546 \times 10^{-34} J s$}]. For a macroscopic particle, a slight change of the path from the classical would change the action by an amount much more than $\pi \hbar$. So, only the classical path contributes to the path integral. Therefore a macroscopic particle has a well defined path, namely the classical path. \\
	
	
	For a microscopic particle like an electron, the action is much smaller. Hence for a large variation of the path from the classical one, the change of action remains less than $\pi \hbar$. It follows that a large number of widely varying paths around the classical path contributes coherently to the propagator $U$. Therefore one cannot say that a microscopic particle follows a definite path as it propagates from one point to another. There is  a lot of --- in the path that a microscopic particle can choose as it propagates between two fixed space time points.\\
	
	
	Consider the following example. A free particle leaves the origin at $t=0$ and arrives at $x=1\ cm$ at $t= 1 \ sec$, the classical path is 
	\begin{equation}
		x_{cl}\qty(t) = a t
	\end{equation}
	where $a$ is a constant with value $a=1\ cm/sec$. Choose another path
	\begin{equation}
		x\qty(t) = b t^2
	\end{equation}
	where $b= 1\ cm/sec^2$.
	
	
	%% figure
	\begin{figure}
		\caption{alternative paths}
	\end{figure}

	We will now calculate the change in action for a macroscopic particle of mass $1 g$ between these two paths. The action for the classical path is
	\begin{align*}
		S\qty[x_{cl}] 
		&= \int_{0}^{1} \frac{1}{2} m \dot{x}_{cl}^2\qty(t) \dd{t} \\
		&= \frac{1}{2} m a^2 \times 1 sec \\
		&= \frac{1}{2} \times \qty(1 g) \times \qty(1 cm/sec)^2 \times (1 sec) = 0.5 erg\ sec
	\end{align*}
	While the action for the alternative path is 
	\begin{align*}
		S\qty[x\qty(t)] 
		&=\int_{0}^{1}\frac{1}{2} \qty(2 b t)^2 \dd{t} \\
		&= 2 b^2 m \int_{0}^{1} t^2 \dd{t}\\
		&= 2 b^2 m \qty(\frac{1}{3} sec^3) \\
		&= \frac{2 \qty(1 cm\ {sec}^{-1})^2 \qty(1 g)\qty(1\ sec^3)}{3} \\
		&= 0.67\ erg \ sec
	\end{align*}
	Therefore $\Delta S = 0.17 \ erg\ sec \approx 1.7 \times 10^{26} \hbar >> \pi \hbar$.\\
	We can therefore ignore non classical paths for the macroscopic particle. On the other hand, for an electron whose mass is $m \approx 10^{-27} g$, the change in action is $\Delta S \approx = \frac{1}{6} \hbar < \pi \hbar$ or the phase difference is $\Delta S/\hbar \approx \frac{1}{6} < \pi$. For the electron the classical path and a wide range of paths around the classical path would contribute to $U$. It is in such cases assuming that the particle moves in a well defined trajectory $x_{cl}\qty(t)$, leads to conflict with experiment.
	
		
		
	%%%% Schrodinger
	\section{Equivalence to the Schrodinger Equation}
	%%%% page 43
	
	
	\section{Potentials of the Form $V=a+b x+ c x^2 + d \dot{x} + e x \dot{x}$}
	
	\subsection{Special Cases}
	\begin{enumerate}
		\item Free Particle\\
		
		\item Harmonic Oscillator \\
		
		
	\end{enumerate}
		
		
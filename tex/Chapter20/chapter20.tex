%%%%%%%%%%%%%%%
% tensor package is used for index management
%
%
%

\chapter{sheet-20 : Time Independent Perturbation}

The major task in any practical application of quantum mechanics is to solve the eigenvalue equation of the Hamiltonian $H$ of the system. Considering the bound states, the eigenvalues of $H$ are discrete and corresponding to each eigenvalue there may be one or several linearly independent eigengectors. The eigenvalue equation, i.e., the time independent Schr\"{o}dinger equation is 
\begin{equation}
	H \ket{E_n} = E_n \ket{E_n}
	\label{chapter20.eqn1-intro}
\end{equation}

Except for few special cases the eigenvalue equation cannot be solved exactly. The eqaution then has to be solved numerically, or approximate methods have to be devised to solve the equation to any desired order of accuracy.

%%% page 2
Time dependent perturbation theory applies when $H$ is of the form
\begin{equation}
	H = H_0 + V
	\label{chapter20.eqn2-intro}
\end{equation}
where the eigenvalues and eigenvectors of $H_0$ are completely known and $V$ is an additional time-independent potential called the perturbation.\\


Let us denote the eigenvalues of $H_0$ as $E_n^{(0)}$ and the corresponding eigenvectors as $\ket{E_n^{(0)}}$  so that the eigenvalue equation for $H_0$ is written as
\begin{equation}
	H_0 \ket{E_n^{(0)}} = E_n^{(0)} \ket{E_n^{(0)}}
	\label{chapter20.eqn3-intro}
\end{equation}

We assume that all the eigenvalues and the eigenvectors of $H_0$ are already calculated.

\section{Non-degenerate Perturbation Theory}
We assume that the eigenvalues $E_n^{(0)}$ of $H_0$ are non-degenerate, i.e., there is only one linearly independent eigenvector $\ket{E_n^{(0)}}$ corresponding to $E_n^{(0)}$. Since the eigenvectors $\ket{E_n^{(0)}}$ form a complete set of vectors, we can express any vector in the Hilbert space as a linear combination of the eigenvectors of $H_0$. Further, eigenvectors belonging to different eigenvalues are orthogonal. We also normalize each of the eigenvectors of $H_0$. So these eigenvectors form a complete orthonormal set, i.e.,
\begin{equation}
	\braket{E_n^{(0)}}{E_m^{(0)}} = \delta_{n m}
	\label{chapter20.eqn4-non-degenerate}
\end{equation}
and
\begin{equation}
	\hat{1} = \sum_{k} \ketbra{E_k^{(0)}}{E_k^{(0)}}
	\label{chapter20.eqn5-non-degenerate}
\end{equation}
Now we modify the eigenvalue equation for the full Hamiltonian $H$ (equation (\ref{chapter20.eqn1-intro})) as
\begin{equation}
	(H_0 + \lambda V) \ket{E_n}_\lambda = E_{n\lambda} \ket{E_n}_\lambda
	\label{chapter20.eqn6-non-degenerate}
\end{equation}
Where we have introduced a real parameter $\lambda$ whose value lies in the range $(0,1)$. The eigenvalue $E_{n \lambda}$ and the eigenvector $\ket{E_n}_\lambda$  in equation (\ref{chapter20.eqn6-intro}) are not quite the same as the corresponding quantities in equation (\ref{chapter20.eqn1-intro}). It is only in the limit the limit $\lambda\rightarrow 1$ would $E_{n\lambda}$ and  $\ket{E_n}_\lambda$ in equation (\ref{chapter20.eqn6-non-degenerate}) coincide with actual values. Furthermore,
\begin{align}
	\lim\limits_{\lambda \rightarrow 0} E_{n \lambda} &= E_n^{(0)} \\
	\lim\limits_{\lambda \rightarrow 0} \ket{E_n}_\lambda &= \ket{E_{n}^{(0)}}
\end{align}
Thus, as $\lambda \rightarrow 0$, the perturbation is switched off and as $\lambda \rightarrow 1$, the full perturbation $V$ is operative.\\

We will now set up a perturbative scheme for solving $E_{n,\lambda}$ and $\ket{E_n}_\lambda$ and at the end set $\lambda = 1$. First, we write $E_{n,\lambda}$ and $\ket{E_n}_\lambda$ as power series in $\lambda$
\begin{align}
\label{chapter20.eqn7-non-degenerate}
E_{n\lambda} &= E_n^{(0)} + \lambda E_n^{(1)} + \lambda^2 E_n^{(2)} + \ldots \\
\label{chapter20.eqn8-non-degenerate}
\ket{E_n}_\lambda &= \ket{E_n^{(0)}} + \lambda \ket{E_n^{(1)}} + \lambda^2 E_n^{(2)} + \ldots
\end{align}

Substituting equation (\ref{chapter20.eqn7-non-degenerate}) and (\ref{chapter20.eqn8-non-degenerate}) in equation (\ref{chapter20.eqn6-non-degenerate}) we have
\begin{align*}
\qty(H_0 + \lambda V)\qty(\ket{E_n^{(0)}} + \lambda \ket{E_n^{(1)}} + \lambda^2 E_n^{(2)} + \ldots)& \\
= \qty(E_n^{(0)} + \lambda E_n^{(1)} + \lambda^2 E_n^{(2)} + \ldots)
&\qty(\ket{E_n^{(0)}} + \lambda \ket{E_n^{(1)}}
+ \lambda^2 E_n^{(2)} + \ldots)
\end{align*}
or
\begin{align*}
H_0 \ket{E_n^{(0)}} + &\lambda \qty(H_0 \ket{E_n^{(1)}}  + V\ket{E_n^{(0)}})  + \lambda^2(H_0 \ket{E_n^{(2)}} + V\ket{E_n^{(1)}}) + \ldots  \\
&= E_n^{(0)} \ket{E_n^{(0)}} + \lambda\qty(E_n^{(1)}\ket{E_n^{(0)}} + E_n^{(0)}\ket{E_n^{(1)}})  \\
&+ \lambda^2 \qty(E_n^{(2)} \ket{E_n^{(0)}}  + E_n^{(1)}\ket{E_n^{(1)}}  +  E_n^{(0)}\ket{E_n^{(2)}}) + \ldots
\end{align*}
We will solve this equation order by order in $\lambda$. So we equate the coefficient of equal power of $\lambda$ on both sides of the above equation. We  have up to order $\lambda^2$
\begin{align}
\label{chapter20.eqn8b-non-degenerate}
\lambda^0 : \quad &  H_0 \ket{E_{n}^{(0)}} = E_n^{(0)}\ket{E_{n}^{(0)}} \\
\label{chapter20.eqn9-non-degenerate}
\lambda^1 : \quad & H_0\ket{E_{n}^{(1)}} + V\ket{E_{n}^{(0)}} = E_n^{(0)}\ket{E_{n}^{(1)}} + E_n^{(1)}\ket{E_{n}^{(0)}} \\
\label{chapter20.eqn10-non-degenerate}
\lambda^2 : \quad & H_0\ket{E_n^{(2)}} + V\ket{E_{n}^{(1)}} = E_n^{(0)}\ket{E_{n}^{(2)}} + E_{n}^{(1)}\ket{E_{n}^{(1)}}  + E_{n}^{(2)}\ket{E_{n}^{(0)}}
\end{align}
Equation (\ref{chapter20.eqn8b-non-degenerate}) is considered solved because we have assumed that we know fully the eigenvalues and eigenvectors of $H_0$.

\subsection{First-Order correction to energy: $E_n^{(1)}$}
The first order correction to the unperturbed energy of the $n$-th level is $E_n^{(1)}$. This can be found from equation (\ref{chapter20.eqn9-non-degenerate}). We start by taking the product of equation (\ref{chapter20.eqn9-non-degenerate}) with $\bra{E_n^{(0)}}$. We get

\begin{equation}
\matrixel{E_n^{(0)}}{H_0}{E_n^{(1)}}
 + \matrixel{E_n^{(0)}}{V}{E_{n}^{(0)}} = E_n^{(0)} \braket{E_n^{(0)}}{E_n^{(1)}} + E_n^{(1)} \braket{E_n^{(0)}}{E_n^{(0)}}
\label{chapter20.eqn11-non-degenerate}
\end{equation}
Now 
\begin{equation}
H_0 \ket{E_{n}^{(0)}} = E_n^{(0)} \ket{E_{n}^{(0)}}
\end{equation}
Since $H_0$ is hermitian
\begin{equation}
	\bra{E_n^{(0)}} H_0 = E_n^{(0)} = E_n^{(0)} \bra{E_n^{(0)}}
\end{equation}
Also
\begin{equation}
\braket{E_n^{(0)}}{E_n^{(0)}} = 1
\end{equation}
Therefore equation (\ref{chapter20.eqn11-non-degenerate}) becomes
\begin{align}
E_n^{(0)} \braket{E_n^{(0)}}{E_n^{(1)}} + \matrixel{E_n^{(0)}}{ V}{E_{n}^{(0)}} 
&= E_n^{(0)} \braket{E_n^{(0)}}{E_n^{(1)}} + E_n^{(1)} \nonumber \\
or, \quad E_n^{(1)} &= \matrixel{E_n^{(0)}}{V}{E_{n}^{(0)}} \equiv V_{n n}
\end{align}
This is a fundamental result of time independent perturbation theory of non-degenerate levels. The first order correction to the $n$-th energy level is the expectation value of the perturbation potential in the unperturbed state.



\subsection{First-Order correction to eigenstate}
The ket $\ket{E_{n}^{(1)}}$ is the first-order correction to the zeroth-order eigenket $\ket{E_{n}^{(0)}}$. The ket $\ket{E_{n}^{(1)}}$ is also found from equation (\ref{chapter20.eqn9-non-degenerate}). First we write $\ket{E_{n}^{(1)}}$ as a linear combination of $\ket{E_{n}^{(0)}}$
\begin{align}
\label{chapter20.eqn13-non-degenerate}
\ket{E_{n}^{(1)}} 
&= \sum_{m} \ket{E_{m}^{(0)}} \braket{E_m^{(0)}}{E_n^{(1)}} \\
\label{chapter20.eqn14-non-degenerate}
&= \sum_{m} \ket{E_{m}^{(0)}} C^(1)_{m n}
\end{align}
Where we have defined
\begin{equation}
C^(1)_{m n} = \braket{E_m^{(0)}}{E_n^{(1)}}
\label{chapter20.eqn15-non-degenerate}
\end{equation}
Using equation (\ref{chapter20.eqn14-non-degenerate}) we write equation (\ref{chapter20.eqn9-non-degenerate}) as
\begin{align*}
	\sum_{m} H_0 \ket{E_{m}^{(0)}} C^{(1)}_{m n} + V \ket{E_{n}^{(0)}} 
	&= \sum_{m} E_n \ket{E_{m}^{(0)}} C^{(1)}_{m n} + E_{n}^{(1)} \ket{E_{n}^{(0)}} \\
	or, \quad 
	\sum_{m} \qty(E_{n}^{(0)}  -  E_{m}^{(0)}) \ket{E_{m}^{(0)}} C^{(1)}_{m n} 
	&= V \ket{E_n^{(0)}} - E_{n}^{(1)} \ket{E_{n}^{(0)}}
\end{align*}
Taking the scalar product with $\bra{E_k^{(0)}}$ we have
\begin{equation}
\qty(E_n^{(0)} - E_k^{(0)}) C^{(1)}_{k n} = \matrixel{E_k^{(0)}} {V} {E_n^{(0)}} - E_n^{(1)} \delta_{k n}
\label{chapter20.eqn16-non-degenerate}
\end{equation}
If $k=n$, the left side is zero and we recover the result
\begin{equation}
E_n^{(1)}  = \matrixel{E_n^{(0)}} {V} {E_n^{(0)}} , \quad k = n
\end{equation}
Thus, we cannot determine $C^{(1)}_{n n}$ from equation (\ref{chapter20.eqn16-non-degenerate}). This coefficient has to be determined from siderations of normalization of the eigenvectors as discussed later.\\

Next, if $k \neq n$, then equation (\ref{chapter20.eqn16-non-degenerate})ecomes
\begin{equation}
\qty(E_n^{(0)} -  E_{k}^{(0)}) C_{k n}^{(1)} = \matrixel{E_k^{(0)}} {V} {E_n^{(0)}} , \quad k \neq n
\end{equation}
or
\begin{equation}
	C_{k n}^{(1)} =  \frac{\matrixel{E_k^{(0)}} {V}{E_n^{(0)}}}{E_n^{(0)} -  E_{k}^{(0)}} , \quad k \neq n
	\label{chapter20.eqn17-non-degenerate}
\end{equation}
Using equation (\ref{chapter20.eqn14-non-degenerate}) and (\ref{chapter20.eqn17-non-degenerate}) the first order correction to the state is
\begin{align}
	\ket{E_{n}^{(1)}} 
	&= \sum_{k} \ket{E_k^{(0)}} C^{(1)}_{k n} \\
	\label{chapter20.eqn18-non-degenerate}
	&= C^{(1)}_{n n} \ket{E_{n}^{(0)}}  + \sum_{\substack{k \\ k \neq n}} \frac{\matrixel{E_k^{(0)}} {V}{E_n^{(0)}}}{E_n^{(0)} -  E_{k}^{(0)}} \ket{E_k^{(0)}} \\
	\label{chapter20.eqn19-non-degenerate}
	&= C^{(1)}_{n n} \ket{E_{n}^{(0)}}  + \sum_{\substack{k \\ k \neq n}} \frac{V_{k n}}{E_n^{(0)} -  E_{k}^{(0)}} \ket{E_k^{(0)}}
\end{align}
	Where
	\begin{equation}
		V_{k n} \equiv \matrixel{E_k^{(0)}} {V} {E_n^{(0)}}
	\end{equation}
	Therefo, upto first order in $\lambda$, the eigenstate $\ket{E_{n}}_\lambda$ is (see equation (\ref{chapter20.eqn8-non-degenerate}))
	\begin{align}
		\ket{E_n}_\lambda 
		&= \ket{E_n^{(0)}} + \lambda \ket{E_n^{(1)}} + \order{\lambda^2} \\
		\label{chapter20.eqn20-non-degenerate}
		&= \ket{E_n^{(0)}} + \lambda C^{(1)}_{n n} \ket{E_n^{(0)}} + \lambda \sum_{\substack{k \\ k \neq n}} \frac{V_{k n}}{E_n^{(0)} -  E_{k}^{(0)}} \ket{E_n^{(0)}} + \order{\lambda^2}
	\end{align}
	We want to normalize $\ket{E_n}_\lambda$ up to first order 
%	\begin{equation}
%		{}_{\lambda}\braket{E_n}{E_n}_{\lambda} = 1 + \order{\lambda^2}
%	\end{equation}
	\begin{equation}
		\tensor*[_{\lambda}]{\braket{E_n}{E_n}}{_{\lambda}} = 1 + \order{\lambda^2}
	\end{equation}
	using equation (\ref{chapter20.eqn20-non-degenerate}), the normalization condition can be written as (noting $\lambda$ is real)
	\begin{align}
		1 + \lambda C^{(1)}_{n n} + \lambda C^{(1)*}_{n n} + \order{\lambda^2}
		&= 1 + \order{\lambda^2} \\
		or, \quad C^{(1)}_{n n} +  C^{(1)*}_{n n} &= 0  \\
	i.e., \quad		\Re{C^{(1)}_{n n}} &= 0
	\end{align}
	Thus $C^{(1)}_{n n}$ is a purely imaginary number. We write
	\begin{equation}
		C^{(1)}_{n n} = \iu \alpha \quad (\alpha \in \real)
	\end{equation}

%%%% page 13
	Hence equation (\ref{chapter20.eqn20-non-degenerate}) can be written as
	\begin{align}
		\ket{E_n}_{\lambda} 
		&= \qty(1 + \iu \lambda \alpha) \ket{E_n^{(0)}} + \lambda \sum_{\substack{k \\ k \neq n}} \frac{V_{k n}}{E_n^{(0)} -  E_{k}^{(0)}} \ket{E_n^{(0)}} + \order{\lambda^2} \\
		&= e^{\iu \lambda \alpha} \ket{E_n^{(0)}} + \lambda \sum_{\substack{k \\ k \neq n}} \frac{V_{k n}}{E_n^{(0)} -  E_{k}^{(0)}} \ket{E_n^{(0)}} + \order{\lambda^2} \\
		e^{-\iu \lambda \alpha} \ket{E_n}_{\lambda}  &= \ket{E_n^{(0)}} + \lambda \sum_{\substack{k \\ k \neq n}} \frac{V_{k n}}{E_n^{(0)} -  E_{k}^{(0)}} \ket{E_n^{(0)}} + \order{\lambda^2}
	\end{align}
	Now $e^{-\iu \lambda \alpha}$ is an overall phase factor which does not affect the normalization of $\ket{E_n}_\lambda$ up to first order. This factor can be set equal to $1$ without loss of generality. So we take $\alpha = 0$, i.e.,
	\begin{equation}
		\braket{E^{(0)}_n}{E^{(1)}_n} \equiv C^{(1)}_{n n} = \iu \alpha = 0
	\end{equation}
	i.e., we can choose $\ket{E_n^{(1)}}$ to be orthonormal to $\ket{E_n^{(0)}}$.\\
	
	Thus, up to first order
	\begin{equation}
		\ket{E_n}_{\lambda}  = \ket{E_n^{(0)}} + \lambda \sum_{\substack{k \\ k \neq n}} \frac{V_{k n}}{E_n^{(0)} -  E_{k}^{(0)}} \ket{E_n^{(0)}} + \order{\lambda^2}
	\end{equation}
	Setting $\lambda = 1$ we get the desired eigenket of  the full Hamiltonian $H$ up to first order in the perturbing potential, i.e.,
	\begin{align}
		\ket{E_n}  
		&= \ket{E_n^{(0)}} + \ket{E_n^{(1)}} \\
		&= \ket{E_n^{(0)}} +  \sum_{\substack{k \\ k \neq n}} \frac{V_{k n}}{E_n^{(0)} -  E_{k}^{(0)}} \ket{E_n^{(0)}}
		\label{chapter20.eqn21-non-degenerate}
	\end{align}
	
	
	\subsection{Second-Order Correction to Energy: $E_n^{(2)}$}
	
	We can find the second order correction to the energy, i.e., $E^{(2)}_n$ from equation (\ref{chapter20.eqn10-non-degenerate}). First, multiply equation (\ref{chapter20.eqn10-non-degenerate}) by $\bra{E_n^{(0)}}$ 
	\begin{equation}
		\matrixel{E_n^{(0)}} {H_0} {E_n^{(2)}} + \matrixel{E_n^{(0)}} {V} {E_n^{(1)}} = E_n^{(0)} \braket{E_n^{(0)}}{E_n^{(2)}} + E_n^{(1)} \braket{E_n^{(0)}}{E_n^{(1)}} + e_n^{(2)}
		\label{chapter20.eqn22-non-degenerate}
	\end{equation}
	Since
	\begin{equation}
		\bra{E_n^{(0)}} H_0 = E_n^{(0)} \bra{E_n^{(0)}}
	\end{equation}
	The first term on the left hand side of equation (\ref{chapter20.eqn22-non-degenerate}) cancels the first term on the right. Therefore, we have
	\begin{align}
		\matrixel{E_n^{(0)}} {V} {E_n^{(1)}} 
		&=  E_n^{(1)} \braket{E_n^{(0)}}{E_n^{(1)}} + E_n^{(2)}\\
		i.e., \quad E_n^{(2)} &=  \matrixel{E_n^{(0)}} {V} {E_n^{(1)}}  - E_n^{(1)} \braket{E_n^{(0)}}{E_n^{(1)}}
		\label{chapter20.eqn23-non-degenerate}
	\end{align}
	Writing
	\begin{equation}
		\ket{E_n^{(1)}} = \sum_{m} \ket{E_m^{(0)}} \braket{E_m^{(0)}}{E_n^{(1)}}
	\end{equation}
	we have
	\begin{equation}
		E_n^{(2)} = \sum_{m} \matrixel{E_n^{(0)}} {V} {E_m^{(0)}} \braket{E_m^{(0)}}{E_n^{(1)}}  -  E_n^{(1)} \braket{E_m^{(0)}}{E_n^{(1)}}
	\end{equation}
	We now isolate the term with $m=n$ in the summation.
	
	Therefore we have
	\begin{equation}
		E_n^{(2)} = \matrixel{E_n^{(0)}} {V} {E_m^{(0)}} \braket{E_m^{(0)}}{E_n^{(1)}} + \sum_{\substack{m\\ m \neq n}} \matrixel{E_n^{(0)}} {V} {E_m^{(0)}} \braket{E_m^{(0)}}{E_n^{(1)}}  -  E_n^{(1)} \braket{E_m^{(0)}}{E_n^{(1)}}
		\label{chapter20.eqn24-non-degenerate}
	\end{equation}
	But
	\begin{align}
		\matrixel{E_n^{(0)}} {V} {E_m^{(0)}} &= E_n^{(1)}
	\end{align}
	So the first term cancels the third term in equation (\ref{chapter20.eqn24-non-degenerate}). We then have
	\begin{equation}
		E_n^{(2)} = \sum_{\substack{m\\ m \neq n}} \matrixel{E_n^{(0)}} {V} {E_m^{(0)}} \braket{E_m^{(0)}}{E_n^{(1)}}
		\label{chapter20.eqn25-non-degenerate}
	\end{equation}
	Where the 
	%prime on the summation symbol means that the 
	term $m=n$ is excluded from the sum.

	Now, we have found previously (\ref{chapter20.eqn17-non-degenerate}) 
	\begin{equation}
		\braket{E_m^{(0)}}{E_n^{(1)}} \equiv C^{(1)}_{m n} = \frac{\matrixel{E_m^{(0)}} {V} {E_n^{(0)}}}{E_n^{(0)} - E_m^{(0)}}
	\end{equation}
	substituting this in equation (\ref{chapter20.eqn25-non-degenerate}) we have
	\begin{equation}
		E_n^{(2)} = \sum_{\substack{m\\ m \neq n}} \frac{\matrixel{E_n^{(0)}} {V} {E_m^{(0)}} \matrixel{E_m^{(0)}} {V} {E_n^{(0)}}}{E_n^{(0)} - E_m^{(0)}}
		\label{chapter20.eqn26-non-degenerate} 
	\end{equation}
	This is the final expression for the second order correction $E_n^{(2)}$ for the $n-th$ level.
	
	Next, introducing the notation
	\begin{equation}
		V_{n m} \equiv \matrixel{E_n^{(0)}} {V} {E_m^{(0)}}
	\end{equation}
	we can write equation (\ref{chapter20.eqn26-non-degenerate}) as 
	\begin{equation}
		E_N^{(2)} = \sum_{\substack{m\\ m \neq n}} \frac{V_{n m} V_{m n}}{E_n^{(0)} - E_m^{(0)}}
	\end{equation}
	Since $V$ is a hermitian operator
	\begin{equation}
		V_{m n} = V_{n m}^*
	\end{equation}
	So,
	\begin{equation}
		E_n^{(2)} = \sum_{\substack{m\\ m \neq n}} \frac{\abs{V_{nm}}^2}{E_n^{(0)} - E_m^{(0)}}
		\label{chapter20.eqn27-non-degenerate} 
	\end{equation}
	Note that the second order correction to the ground state energy is negative. Also, in the second order, the effect of an energy level above the $n$-th level is to push down the energy of the $n$-th level. The effect of a level below the $n$-th level is to push up the energy of the $n$-th level. It is as if, the levels are repelling each other in the $2$nd order perturbation

\section{Degenerate Perturbation Theory}
	%% page 19
	In perturbation theory we seek a solution of the eigenvalue equation of the Hamiltonian $H$, where
	\begin{equation}
		H = H_0 + H^\prime
		\label{chapter20.eqn1-degenerate} 
	\end{equation}
	We assume that the eigenvalues and eigenfunctions of the unperturbed Hamiltonian $H_0$ are known. We then ask how the energy and the wave function of the $n$-th level of the $H_0$ are modified when the perturbation $H^\prime$ is turned on.
	
	Suppose that the $n^{th}$ level of $H_0$ is $g_n$-fold degenerate. Therefore
	\begin{equation}
		H_0 \psi^{(0)}_{n \alpha} = E_n^{(0)} \psi^{(0)}_{n \alpha} \quad ; \alpha = 1,2,\ldots, g_n
		\label{chapter20.eqn2-degenerate} 
	\end{equation}
	The $g_n$ wave functions $\{\psi^{(0)}_{n \alpha}\ ; \alpha=1,2,\ldots, g_n\}$ are the linearly independent of each other and they are all orthogonal to the unperturbed wave functions belonging to other energy levels.
	
	% page 20
	We note that any linear combination of the vectors $\qty{\psi^{(0)}_{n \alpha}; \ \alpha=1,\ldots,g_n}$ is also an eigenvector of $H_0$ with the same eigenvalue $E_n^{(0)}$. Thus if we construct a vector $\chi^{(0)}_{n \beta}$ as
	\begin{equation}
		\chi^{(0)}_{n \beta} = \sum_{\alpha = 1}^{g_n} C_{\alpha \beta} \psi^{(0)}_{n\alpha}
		\label{chapter20.eqn3-degenerate} 
	\end{equation}
	
	Then $\chi^{(0)}_{n \beta}$ is also an eigenvector of $H_0$ with eigenvalue $E_n^{(0)}$:
	\begin{equation}
		H_0 \chi^{(0)}_{n \beta} = E_n^{(0)} \chi^{(0)}_{n \beta}
		\label{chapter20.eqn4-degenerate} 
	\end{equation}
	Now the vectors $\qty{\psi^{(0)}_{n \alpha}; \ \alpha=1,\ldots,g_n}$ need not be orthogonal. However, by using the Schmidt procedure, we can make the degenerate eigenvectors orthonormal by taking suitable linear combinations if they are not orthogonal to start with. This procedure can be applied to all vectors belonging to every level.
	
	% page 20a
	Thus, we will assume that all vectors whether belonging to the same leel or not are normalized and orthogonal to each other, i.e.,
	\begin{equation}
		\braket{\psi^{(0)}_{n\alpha}}{\psi^{(0)}_{m\beta}} = \delta_{n m}\delta_{\alpha \beta}
		\label{chapter20.eqn5-degenerate} 
	\end{equation}
	Further, the eigenvectors of $H_0$ space the entire Hilbert space, i.e., they form a complete set of states. The completeness condition can be written as
	\begin{equation}
		\hat{1} = \sum_{k} \sum_{\alpha = 1}^{g_k} \ketbra{\psi^{(0)}_{k\alpha}}{\psi^{(0)}_{k\alpha}}
		\label{chapter20.eqn6-degenerate} 
	\end{equation}
	The "full" eigenvalue equation for the $n^{th}$ level is written as
	\begin{equation}
		H \psi_{n \alpha} = E_{n \alpha} \psi_{n \alpha} \quad ; \quad \alpha = 1,2,\ldots,g_n
		\label{chapter20.eqn7-degenerate} 
	\end{equation}
	In order to facilitate counting of different orders, we may write
	\begin{equation}
		H = H_0 + \lambda H^\prime
	\end{equation}
	Where $\lambda$ is a real parameter which we set equal to one at the end of our calculations.
	% page 20b
	The eigenvalues $E_{n \alpha}$ and the eigenvector $\psi_{n\alpha}$ are now functions of $\lambda$. In the limit $\lambda \rightarrow 0$ $E_{n\alpha}$ tends to $E_n^{(0)}$, i.e.,
	\begin{equation}
		\lim\limits_{\lambda \rightarrow 0} E_{n\alpha} = E_n^{(0)}
		\label{chapter20.eqn8-degenerate} 
	\end{equation}
	However, there is a difficulty in taking the corresponding limits for $\psi_{n\alpha}$. Since there are $g_n$ linearly independent unperturbed eigenfunctions corresponding to $E_n^{(0)}$, we do not know to which particular eigenfunction will $\psi_{n\alpha}$ tend to when $\lambda \rightarrow 0$. Suppose
	
	\begin{equation}
		\psi_{n\alpha} \quad \substack{\rightarrow \\ \lambda=0} \quad \chi_{n\alpha}^{(0)}
	\end{equation}
	Where $\chi_{n\alpha}^{(0)}$ is some linear combination of $\qty{\psi_{n \alpha}^{(0)}, \alpha=1,2,\ldots,g_n}$.
	
	Now we write
	\begin{equation}
		\psi_{n \alpha} = \chi_{n\alpha}^{(0)} + \lambda \psi_{n \alpha}^{(1)} + \lambda^2 \psi_{n \alpha}^{(2)}
		\label{chapter20.eqn9-degenerate} 
	\end{equation}
	where $\chi_{n\alpha}^{(0)}$ is as yet some undetermined linear combination of $\qty{\psi_{n \alpha}^{(0)}, \alpha=1,2,\ldots,g_n}$. We also write the perturbed energy $E_{n\alpha}$ as 
	\begin{equation}
		E_{n\alpha} = E_n^{(0)} + \lambda E_{n\alpha}^{(1)} + \lambda^2 E_{n\alpha}^{(2)}
		\label{chapter20.eqn10-degenerate}
	\end{equation}
	where we have used the fact that $E_{n\alpha}^{(0)} = E_n^{(0)}$ for all $\alpha$.\\
	
	Next, we substitute (\ref{chapter20.eqn9-degenerate}) and (\ref{chapter20.eqn10-degenerate}) in equation (\ref{chapter20.eqn7-degenerate}). We have

	\begin{equation}
		\qty(H_0 + \lambda H^\prime) \qty(\chi_{n\alpha}^{(0)} + \lambda \psi_{n \alpha}^{(1)} + \ldots) 
		= \qty(E_n^{(0)} + \lambda E_{n\alpha}^{(1)} + \ldots) \qty(\chi_{n\alpha}^{(0)} + \lambda \psi_{n \alpha}^{(1)} + \ldots) 
	\end{equation}
	Equating the coefficient of equal powers of $\lambda$ on both sides of this equation we obtain the zeroth order equation
	\begin{equation}
		H_0 \chi_{n\alpha}^{(0)} = E_n^{(0)} \chi_{n\alpha}^{(0)}
	\end{equation}
	which is equation (\ref{chapter20.eqn4-degenerate}) written earlier. In the first order we have
	
	%%	page 22
	
	
	\begin{equation}
		\qty(H_0 - E_n^{(0)}) \psi_{n \alpha}^{(1)} = \qty(E_{n\alpha}^{(1)} - H^\prime) \chi_{n\alpha}^{(0)}
		\label{chapter20.eqn11-degenerate}
	\end{equation}
	
	Now, we write
	\begin{equation}
		\psi_{n \alpha}^{(1)} = \sum_{k\beta} C_{k\beta, n\alpha} \psi_{k \beta}^{(0)}
		\label{chapter20.eqn12-degenerate}
	\end{equation}
	and
	\begin{equation}
		\chi_{n\alpha}^{(0)} = \sum_{\beta=1}^{g_n} a_{\beta\alpha} \psi_{n\beta}^{(0)}
		\label{chapter20.eqn13-degenerate}
	\end{equation}
	
	where the indices $\alpha$ and $\beta$ refer explicitly to degeneracy. Substituting (\ref{chapter20.eqn12-degenerate}) and(\ref{chapter20.eqn13-degenerate}) in equation (\ref{chapter20.eqn11-degenerate}) we find
	\begin{align}
		\qty(H_0 - E_n^{(0)}) \sum_{k,\beta} C_{k\beta, n\alpha} \psi_{k \beta}^{(0)} 
		&= \qty(E_{n\alpha}^{(1)} - H^\prime) \sum_{\beta=1}^{g_n} a_{\beta\alpha} \psi_{n\beta}^{(0)} \\
		or,\ \sum_{k,\beta} C_{k\beta, n\alpha} \qty(E_k^{(0)} - E_n^{(0)}) \psi_{k \beta}^{(0)} 
		&= \sum_{\beta=1}^{g_n} a_{\beta\alpha} \qty(E_{n\alpha}^{(1)} - H^\prime) \psi_{n\beta}^{(0)} 
	\end{align}
	Taking the scalar product with $\psi_{m\gamma}^{(0)}$ and using orthogonality $\braket{\psi_{m\gamma}^{(0)}}{\psi_{k\beta}^{(0)}} = \delta_{m k} \delta_{\gamma \beta}$, we have
	
	\begin{equation}
		C_{m\gamma, n\alpha} \qty(E_m^{(0)} - E_n^{(0)}) = \sum_{\beta=1}^{g_n} a_{\beta\alpha} \qty(E_{n\alpha}^{(1)}\delta_{mn}\delta_{\gamma\beta}  -  H^\prime_{m\gamma,n\beta})
		\label{chapter20.eqn14-degenerate}
	\end{equation}
	where we have written
	\begin{equation}
		H^\prime_{m\gamma,n\beta} = \matrixel{\psi_{m\gamma}^{(0)}} {H^\prime} {\psi_{n\beta}^{(0)}}
		\label{chapter20.eqn15-degenerate}
	\end{equation}
	
	\subsubsection{First Order Correction to Energy}
	%% page 25 (pdf) or 23(hand written)
	
	First, let us choose $m=n$ in equation (\ref{chapter20.eqn14-degenerate}). Then the left hand side of this equation is zero. We then have
	\begin{equation}
		\sum_{\beta=1}^{g_n} \qty(H^\prime_{n\gamma,n \beta}  -  E_{n\alpha}^{(1)}\delta_{\gamma \beta}) a_{\beta \alpha} = 0
		\label{chapter20.eqn16-degenerate}
	\end{equation}
	Simplifying the notation by writing
	\begin{equation}
		H^\prime_{n\gamma, n\beta} = H^{\prime (n)}_{\gamma\beta}
	\end{equation}
	We write equation (\ref{chapter20.eqn16-degenerate}) as
	% page 26
	
	\begin{equation}
		\sum_{\beta=1}^{g_n} \qty(H^{\prime (n)}_{\gamma \beta}  -  E_{n\alpha}^{(1)}\delta_{\gamma \beta}) a_{\beta \alpha} = 0
		\label{chapter20.eqn17-degenerate}
	\end{equation}
	Equation (\ref{chapter20.eqn16-degenerate}) is a set of $g_n$ linear equations for unknowns $\qty{a_{1\alpha}, a_{2\alpha}, \ldots, a_{g_n \alpha}}$ corresponding to $E_{n\alpha}^{(1)}$. The value of $E_{n\alpha}^{(1)}$ are not known a priori.\\
		
		However, we note that, for a solution of equation (\ref{chapter20.eqn17-degenerate}) to exist, the determinant formed by the coefficient of $a_{\beta\alpha}$ must vanish, i.e.,
		\begin{equation}
			\det \qty[H^{\prime (n)}_{\gamma \beta}  -  E^{(1)}_{n\alpha} \delta_{\gamma \beta}] = 0
			\label{chapter20.eqn18-degenerate}
		\end{equation}
		This is called the secular equation, which is a polynomial of degree $g_n$ in $E_{n\alpha}^{(1)}$. It has $g_n$ real roots $E_{n 1}^{(1)}, E_{n 2}^{(1)}, \ldots, E_{n, g_n}^{(1)}$. If all these roots are distinct, the degeneracy is completely removed to first order in the perturbation. On the other hand, if some or all roots of equation (\ref{chapter20.eqn18-degenerate}) are identical, the degeneracy is only partially (or not at all) removed. The residual degeneracy may then either be removed in higher order perturbation theory, or it may persist in all orders.\\
		
		Next, substituting each of the roots $E_{n\alpha}^{(1)}, \ \alpha=1,2,\ldots,g_n$ in equation (\ref{chapter20.eqn17-degenerate}) we can solve for coefficients $a_{1\alpha}, a_{2 \alpha}, \ldots, a_{g_n \alpha}$. In fact, one of the coefficients remain undetermined and the other coefficients are found in terms of the undetermined one. This is because the set of equation given by equation (\ref{chapter20.eqn17-degenerate}) are homogeneous. The undetermined coefficient is then obtained up to a phase by requiring that the eigenvector $a_{\beta\alpha}; \beta = 1,2,\ldots, g_n$ be normalized to unity.
		\begin{equation}
			a^{*}_{1\alpha} a{1\alpha}  +  a^{*}_{2\alpha} a{2\alpha} + \ldots + a^{*}_{g_n\alpha} a{g_n\alpha} = 1
		\end{equation}
		that is,
		\begin{equation}
			\sum_{\beta = 1}^{g_n}	a^{*}_{2\alpha} a{2\alpha} = 1 \quad ; \alpha=1,2,\ldots, g_n
		\end{equation}
		The correct zeroth order wave function is then found using equation (\ref{chapter20.eqn13-degenerate}), i.e.,
		\begin{equation}
			\chi_{n\alpha}^{(0)} = \sum_{\beta = 1}^{g_n} a_{\beta\alpha} \psi_{n\beta}^{(0)}
		\end{equation}
		The functions $\chi_{n\alpha}^{(0)}$ are eigenvectors of $H^\prime$ in the eigen subspace of $E_n^{(0)}$ with eigenvalue $E_{n\alpha}^{(1)}$, i.e.,
		\begin{equation}
			H^\prime \chi_{n\alpha}^{(0)}= E_{n\alpha}^{(1)} \chi_{n\alpha}^{(0)} \quad ; \alpha = 1,2,\ldots, g_n
		\end{equation}
		and the coefficients $a_{\beta \alpha}\ , \beta=1,2, \ldots, g_n$ form the $g_n$ component representation of the eigenvector $\chi_{n\alpha}^{(0)}$ using the basis $\qty{\psi_{n\beta}^{(0)}\ , \beta =1,2, \ldots,g_n}$.
		
		Thus
		\begin{align}
			\chi_{n\alpha}^{(0)} &= \mqty[a_{1\alpha} \\ a_{2\alpha} \\ \vdots \\ a_{g_n\alpha}] 
		\end{align}
		
		Thus, in summary, the first-order corrections to the $n^{th}$ degenerate level of $H_0$ with energy $E_n^{(0)}$ are obtained by diagonalizing $H^\prime$ in the eigen subspace of $E_n^{(0)}$. The eigenvalues of $H^\prime$ are the corrections to the energy and the corresponding eigenvectors of $H^\prime$ are the zeroth order approximation of the wavefunction.\\
		
		Once the correct zeroth-order wavefunctions $\chi_{n\alpha}^{(0)}\ , \alpha=1,2, \ldots, g_n$, have been determined, the first order correction $\psi_{n \alpha}^{(1)}$ to the wavefunction and second-order energy correction $E_{n\alpha}^{(2)}$ can be obtained in a way similar to non-degenerate perturbation theory.
		
		
		
		%% page 31 Ex
		\section{Examples of Non-Degenerate Perturbation Theory}
		\begin{enumerate}[label=Problem.\arabic*,start=1]
			\item 
			Calculate the first order energy shifts for the first three states of the infinite square well of width $a$ in one dimension due to the perturbation $V(x) = V_0 \frac{x}{a}$.\\
			
			\underline{Ans}\\
			\begin{equation*}
				H = H_0 + V
			\end{equation*}
			where
			\begin{equation*}
				H_0 = \frac{p^2}{2m} + V_0(x)
			\end{equation*}
			and
			\begin{equation*}
				V_0(x) = \begin{cases}
				0 \quad 0 \leq x \leq a \\
				\infty \quad \text{otherwise}
				\end{cases}
			\end{equation*}
			
			
			\begin{figure}
				%%%%%%% TODO
				\centering
				\includegraphics[width=0.5\linewidth]{Pictures/not-found.jpg}
				\caption{Infinite Square Well}
			\end{figure}
			
			\underline{Unperturbed states}
			\begin{equation*}
				H_0 \ket{E^{(0)}} = E^{(0)} \ket{E^{(0)}}
			\end{equation*}
			In the coordinate representation
			\begin{equation*}
				-\frac{\hbar^2}{2 m} \dv[2]{\psi^{(0)}\qty(x)}{x} + V\qty(x) \psi^{(0)}\qty(x) = E^{(0)} \psi^{(0)}\qty(x)
			\end{equation*}
			In the region $0 < x < a,\quad V(x) = 0$. Therefore
			\begin{align}
				-\frac{\hbar^2}{2 m} \dv[2]{\psi^{(0)}\qty(x)}{x} 
				&= E^{(0)} \psi^{(0)}\qty(x) \nonumber\\
		or\quad		\dv[2]{\psi^{(0)}\qty(x)}{x} + k^2 \psi^{(0)}\qty(x) &= 0 \quad (0 < x < a)
			\label{chapter20.ex1.eqn1}
			\end{align}
			where
			\begin{equation*}
				k = \sqrt{\frac{2 m E_0}{\hbar^2}}
			\end{equation*}
			The wave function must be zero at the boundaries of the potential and outside the potential.\\
			
			The general solution of the interior wave function $\psi^{(0)}\qty(x)$ is
			\begin{equation*}
				\psi^{(0)}\qty(x) = A \sin k x + B \cos k x
			\end{equation*}
			since $\psi^{(0)}\qty(x=0)=0$, we must have $B=0$. Thus
			\begin{equation*}
				\psi^{(0)}\qty(x) = A \sin k x 
			\end{equation*}
			And since $\psi^{(0)}\qty(x=a)$ is also zero, we must also have
			\begin{align}
				\sin k a &= 0 \nonumber\\
				or, \quad k a &= \pi, 2\pi, 3\pi, \ldots \nonumber\\
				i.e.,\quad k a &= n \pi, \quad n = 1,2,3,\ldots
				\label{chapter20.ex1.eqn2}
			\end{align}
			Equation (\ref{chapter20.ex1.eqn2}) is the quantization condition. The unperturbed energy levels are
			\begin{equation}
				E^{(0)}_n = \frac{\hbar^2 k^2}{2 m} = \frac{\hbar^2}{2 m \qty(\frac{n^2 \pi^2}{a^2})} = \frac{\pi^2 \hbar^2}{2 m a^2} n^2
				\label{chapter20.ex1.eqn3}
			\end{equation}
			The first three energy levels are
			\begin{align*}
				E_1^{(0)} &= \frac{\pi^2 \hbar^2}{2 m a^2} \\
				E_2^{(0)} &= 4 \frac{\pi^2 \hbar^2}{2 m a^2} \\
				E_3^{(0)} &= 9 \frac{\pi^2 \hbar^2}{2 m a^2}
			\end{align*}
			
			Now we will normalize the unperturbed wave functions. For an arbitrary level $n$,
			\begin{align*}
				\psi_n^{(0)}\qty(x) &= A_n \sin k_n x \\
				\therefore \ \int \psi_n^{(0)*}\qty(x) \psi_n^{(0)}\qty(x) \dd{x} &= 1 \\
				or,\ \abs{A_n}^2 \int \sin^2 k_n x \dd{x} &= 1 \\
				or,\ \abs{A_n}^2 \int_{0}^{a} \frac{1}{2} \qty(1 - \cos 2 k_n x) &= 1 \\
				or,\ \frac{\abs{A_n}^2}{2} \qty[a - \int_{0}^{a} \cos 2 k_n x \dd{x}] &= 1 \\
				or, \ \abs{A_n}^2 \frac{a}{2} &= 1
			\end{align*}
			Therefore we can choose
			\begin{equation}
				A_n = \sqrt{\frac{2}{a}}
			\end{equation}
			Therefore, normalized unperturbed wave functions for the first three levels are
			\begin{align*}
				\psi_1^{(0)} \qty(x) &=  \sqrt{\frac{2}{a}} \sin \frac{\pi x }{a} \\
				\psi_2^{(0)} \qty(x) &=  \sqrt{\frac{2}{a}} \sin \frac{2\pi x }{a} \\
				\psi_3^{(0)} \qty(x) &=  \sqrt{\frac{2}{a}} \sin \frac{3\pi x }{a} 
			\end{align*}
			
			\underline{First order correction to energy}\\
			The first-order energy corrections are then
			\begin{align*}
				E_1^{(1)} = \expval{V}{\psi_1^{(0)}} &= \frac{2}{a} \frac{V_0}{a} \int_{0}^{a} x \sin^2 \frac{\pi x}{a} \dd{x} = \frac{V_0}{2} \\
				E_2^{(1)} = \expval{V}{\psi_2^{(0)}} &= \frac{2}{a} \frac{V_0}{a} \int_{0}^{a} x \sin^2 \frac{2\pi x}{a} \dd{x} = \frac{V_0}{2} \\
				E_3^{(1)} = \expval{V}{\psi_3^{(0)}} &= \frac{2}{a} \frac{V_0}{a} \int_{0}^{a} x \sin^2 \frac{3\pi x}{a} \dd{x} = \frac{V_0}{2} \\
			\end{align*}
			Therefore, to first order, the perturbed energies are
			\begin{align}
				E_1 = E_1^{(0)} + E_1^{(1)} &= \frac{\pi^2 \hbar^2}{2 m a^2} + \frac{V_0}{2} \\
				E_2 = E_2^{(0)} + E_2^{(1)} &= 4\frac{\pi^2 \hbar^2}{2 m a^2} + \frac{V_0}{2} \\
				E_3 = E_3^{(0)} + E_3^{(1)} &= 9\frac{\pi^2 \hbar^2}{2 m a^2} + \frac{V_0}{2} \\
			\end{align}
			
			
			\item 
			A particle of mass $m$ moves in a $1$-dimensional oscillator potential
			\begin{equation}
				V(x) = \frac{1}{2} m \omega^2 x^2
			\end{equation}
			In the non-relativistic limit, where the kinetic energy and momentum are related by
			\begin{equation}
				T = \frac{p^2}{2 m}
			\end{equation}
			The ground state energy is well-known to be $E_0=\frac{1}{2} \hbar \omega$. Relativistically, the kinetic energy and the momentum are related by
			\begin{equation}
				T = E - m c^2 = \sqrt{m^2 c^4 + p^2 c^2} - m c^2
			\end{equation}
			(a) Determine the lowest order correction to the kinetic energy due to relativistic effects.\\
			(b) Considering the correction to the kinetic energy as a perturbation, compute the relativistic correction to the ground state energy.\\
			
			\underline{Ans}\\
			(a) We have
			\begin{align*}
				T = E - mc^2 &= \sqrt{m^2 c^4 + p^2 c^2} - m c^2 \\
				&= mc^2 \sqrt{1 + \frac{p^2c^2}{m^2 c^4}} - mc^2 \\
				&= mc^2 \qty(1 + \frac{p^2 c^2}{2 m^2 c^4} - \frac{p^4 c^4}{8 m^4 c^8} + \ldots) - mc^2 \\
				&= mc^2 \qty(1 + \frac{p^2}{2 m^2 c^2} - \frac{p^4 }{8 m^4 c^4} + \ldots) - mc^2 \\
				&\approx \frac{p^2}{2 m} - \frac{p^4}{8 m^3 c^2}
			\end{align*}
			where we have used the binomial expansion of $(1+x)^n$ (appendix (\ref{appendix2.series}))
			
			
			(b) The unperturbed Hamiltonian is
			\begin{equation}
				H_0 = \frac{p^2}{2 m} + \frac{1}{2} m \omega^2 x^2
			\end{equation}
			$H_0$ represents a one-dimensional harmonic oscillator. The eigenstates and the eigenvalues are 
			\begin{equation*}
				H_0 \ket{n^{(0)}} = E_n^{(0)} \ket{n^{(0)}}
			\end{equation*}
			where,
			\begin{equation*}
				E_n^{(0)} = \qty(n + \frac{1}{2}) \hbar \omega; \quad n=0,1,2,\ldots
			\end{equation*}
			We take the perturbation to be
			\begin{equation*}
				V = - \frac{p^4}{8 m^3 c^2}
			\end{equation*}
			The energy correction for the ground state is then
			\begin{align*}
				E_0^{(1)} 
				&= \expval{V}{0};\quad \ket{0} = \text{unperturbed ground state}\\
				&= - \expval{\frac{p^4}{8 m^3 c^2}}{0} \\
				&= - \frac{1}{8 m^3 c^2} \expval{p^4}{0}
			\end{align*}
			where
			\begin{equation*}
				p = \iu p_0 \qty(a^{\dagger} - a),\quad p_0 = \sqrt{\frac{\hbar m \omega}{2}}
			\end{equation*}
			Thus
			\begin{align*}
				E_0^{(1)} 
				&= - \frac{1}{8 m^3 c^2} \frac{\hbar^2 m^2 \omega^2}{4} \expval{\qty(a^{\dagger} - a)^4}{0} \\
				&= - \frac{\hbar^2 \omega^2}{32 m c^2} \expval{\qty(a^{\dagger} - a)^4}{0} \\
				&= - \frac{\hbar^2 \omega^2}{32 m c^2} \expval{\qty(a^{\dagger} - a)\qty(a^{\dagger} - a)\qty(a^{\dagger} - a)\qty(a^{\dagger} - a)}{0} \\
				&= + \frac{\hbar^2 \omega^2}{32 m c^2} \expval{a\qty(a^{\dagger} - a)\qty(a^{\dagger} - a)a^{\dagger}}{0} \\
				&= + \frac{\hbar^2 \omega^2}{32 m c^2} \expval{\qty(a^{\dagger} - a)\qty(a^{\dagger} - a)}{1} \\
			\end{align*}
			Two ways to solve from here
			\begin{align*}
				E_0^{(1)}  
				&= + \frac{\hbar^2 \omega^2}{32 m c^2} \expval{\qty(a^{\dagger} - a)\qty(a^{\dagger} - a)}{1} \\
				&= \frac{\hbar^2 \omega^2}{32 m c^2} \bra{1}\qty(a^\dagger - a) \qty(\sqrt{2}\ket{2} - \ket{0}) \\
				&= \frac{\hbar^2 \omega^2}{32 m c^2} \bra{1} \qty(\sqrt{6}\ket{3} -\ket{1} - 2 \ket{1}) \\
				&= -\frac{3\hbar^2 \omega^2}{32 m c^2}
			\end{align*}
			or,
			\begin{align*}
				E_0^{(1)}  
				&= + \frac{\hbar^2 \omega^2}{32 m c^2} \expval{\qty(a^{\dagger} - a)\qty(a^{\dagger} - a)}{1} \\
				&= \frac{\hbar^2 \omega^2}{32 m c^2}\expval{\qty(a^{\dagger}a^{\dagger} -a^{\dagger} a - a a^{\dagger} + a a)}{1} \\
				&= - \frac{\hbar^2 \omega^2}{32 m c^2} \expval{a^{\dagger} a + a a^{\dagger}}{1} \\
				&= - \frac{\hbar^2 \omega^2}{32 m c^2} \bra{1} \qty(a^\dagger \ket{0} + a \sqrt{2}\ket{2}) \\
				&= - \frac{\hbar^2 \omega^2}{32 m c^2} \bra{1} \qty(\ket{1} + 2\ket{1}) \\
				&= - \frac{\hbar^2 \omega^2}{32 m c^2} \bra{1} 3\ket{1}\\
				&= - \frac{3 \hbar^2 \omega^2}{32 m c^2}
			\end{align*}
			
			Where we have used the ladder operator for Harmonic oscillator, defined by
			\begin{align*}
				a\ket{n} = \sqrt{n} \ket{n - 1} \\
				a^\dagger\ket{n} = \sqrt{n+1} \ket{n+1} \\
			\end{align*}
			
			
			
			
			\item Find the first order correction to the $n^{th}$ level of a one-dimensional harmonic oscillator perturbed by the potential
			\begin{equation*}
				H^\prime \qty(x) = \epsilon_3 x^3 + \epsilon_4 x^4
			\end{equation*}
			\underline{Ans}\\
			For an unperturbed one-dimensional harmonic oscillator, the Hamiltonian is
			\begin{equation*}
				H_0 = - \frac{\hbar^2}{2 m} \dv[2]{}{x} + \frac{1}{2 } m \omega^2 x^2
			\end{equation*}
			The eigenvalue equation $H_0$ is completely solved. We have
			\begin{equation}
				H_0 \ket{n} = E_n \ket{n}
			\end{equation}
			where
			\begin{equation}
				E_n = \qty(n + \frac{1}{2}) \hbar \omega \ ;\quad n=0,1,2,\ldots
			\end{equation}
			Now the first-order correction to energy is 
			\begin{equation}
				E_n^{(1)} = \expval{H^\prime}{n} = \expval{\epsilon_3 x^3}{n} + \expval{\epsilon_4 x^4}{n}
			\end{equation}
			The expectation value of $x^3$ is any unperturbed state of the harmonic oscillator is zero, because the wave function corresponding to an unperturbed state is either even or odd. So
			\begin{equation}
				\expval{x^3}{n} = \int u_n^*\qty(x) x^3 u_m\qty(x) \dd{x} = 0
			\end{equation}
			because the integrand is odd. Therefore
			\begin{equation}
				E_n^{(1)} = \epsilon_4 \expval{x^4}{n}
				\label{chapter20.ex3.eqn1}
			\end{equation}
			At this state, we can work out the expectation value of $x^4$ in the $n^{th}$ unperturbed state, i.e., the right hand side of equation (\ref{chapter20.ex3.eqn1}), by using the wave function of the $n^{th}$ unperturbed state:
			\begin{equation}
				E_n^{(1)} = \epsilon_4 \expval{x^4}{n} = \int u_n^*\qty(x) x^4 u_m\qty(x) \dd{x}
			\end{equation}
			The wave function $u_n\qty(x)$ involve the Hermite polynomial $H_n\qty(x)$ and the integration in the above equation is non trivial. It is easier to find the expectation value $\expval{x^4}{n}$ using the creation ($a^\dagger$) and destruction ($a$) operator
			We define
			\begin{align}
				a &= \frac{1}{\sqrt{2 m \hbar \omega}} \qty(m \omega x + \iu p) \\
				a^\dagger &= \frac{1}{\sqrt{2 m \hbar \omega}} \qty(m \omega x + \iu p)
			\end{align}
			In terms of $a$ and $a^\dagger$, the unperturbed Hamiltonian is 
			\begin{equation}
				H_0 = \frac{\hat{p}^2}{2 m} + \frac{1}{2} m \omega \hat{x}^2 = \qty(a^\dagger a + \frac{1}{2}) \hbar \omega
			\end{equation}
			We can write the operators $\hat{x}$ and $\hat{p}$ in terms of $a$ and $a^\dagger$
			\begin{align}
				\hat{x} &= \qty(\frac{\hbar}{2 m \omega})^{1/2} \qty(a + a^\dagger)\\
				\hat{p} &= \frac{1}{\iu}\qty(\frac{\hbar}{2 m \omega})^{1/2} \qty(a - a^\dagger)
			\end{align}
			We also have
			\begin{align}
				a\ket{n} &= \sqrt{n} \ket{n - 1} \\
				a^\dagger \ket{n} &= \sqrt{n+1} \ket{n+1}
			\end{align}
			The perturbing Hamiltonian is
			\begin{align}
				H^\prime 
				&= \epsilon_3 x^3 + \epsilon_4 x^4 \\
				&= \epsilon_3 \qty(\frac{\hbar}{2 m \omega})^{3/2} \qty(a + a^\dagger)^3 + \epsilon_4 \qty(\frac{\hbar}{2 m \omega})^2 \qty(a + a^\dagger)^4
			\end{align}
			
			%% page 45
			The first order correction to energy is 
			\begin{equation}
				E_n^{(1)} = \epsilon_3 \expval{x^3}{n} + \epsilon_4 \expval{x^4}{n}
			\end{equation}
			The first term is zero, as we have argued previously. 
			%% <<< by shahnoor
			
			In terms of the operators $a$ and $a^\dagger$, we will have odd number of $a$ or $a^\dagger$ in each terms if we compute $(a+a^\dagger)^3$. Thus when sandwiched with $\ket{n}$ the orthogonality will yield zero.
			%% >>>
			So,
			\begin{align*}
				E_n^{(1)} &=  \epsilon_4 \expval{x^4}{n} \\
				&= \epsilon_4 \frac{\hbar^2}{4 m^2 \omega^2} \expval{\qty(a+a^\dagger)^4}{n}\\
				&= \epsilon_4 \frac{\hbar^2}{4 m^2 \omega^2} \expval{\qty(a+a^\dagger)\qty(a+a^\dagger)\qty(a+a^\dagger)\qty(a+a^\dagger)}{n}
			\end{align*}
			This expression on the right hand side has sixteen terms, each term having four factors $a$ or $a^\dagger$ in a variety of different orders. Only terms containing two $a$'a and two $a^\dagger$'s yield non-zero contributions. These terms are
						\begin{align*}
			E_n^{(1)} 
			&= \epsilon_4 \frac{\hbar^2}{4 m^2 \omega^2} \expval{a a a^\dagger a^\dagger + a a^\dagger a a^\dagger +  a^\dagger a a a^\dagger + a a^\dagger a^\dagger a + a^\dagger a a^\dagger a + a^\dagger a^\dagger a a}{n}
			\end{align*}
			The six expectation values in the above expression can be calculated in a straight forward manners.
			\begin{align*}
				\expval{a a a^\dagger a^\dagger}{n} = \matrixelement{n}{a a a^\dagger}{n+1} \sqrt{n + 1} \\
				&= \mel{n}{a a }{n+2} \sqrt{n+2} \sqrt{n+1} \\
				&= \mel{n}{a}{n+1} \sqrt{n+2} \sqrt{n+2} \sqrt{n+1} \\
				&= \braket{n}{n} \sqrt{n+1} \sqrt{n+2} \sqrt{n+2} \sqrt{n+1} \\
				&= (n+1) (n+2)
			\end{align*}
			\begin{align*}
				\expval{a a^\dagger a a^\dagger}{n} &= \mel{n}{a a^\dagger a}{n+1} \sqrt{n+1} \\
				&= \mel{n}{a a^\dagger}{n} \qty(n+1) \\
				&= \braket{n}{n} \qty(n+1) \qty(n+1)\\
				&= \qty(n+1)^2
			\end{align*}
			\begin{align*}
				\expval{a a^\dagger a^\dagger a}{n} &= \mel{n}{a a^\dagger a^\dagger}{n-1} \sqrt{n} \\
				&= \mel{n}{a a^\dagger}{n} \sqrt{n} \sqrt{n} \\
				&= \mel{n}{a}{n+1} \sqrt{n+1} n \\
				&= \braket{n}{n} \sqrt{n+1}  \sqrt{n+1}  n \\
				&= n \qty(n+1)
			\end{align*}
			\begin{align*}
				\expval{a^\dagger a a a^\dagger}{n} = n \qty(n+1)
			\end{align*}
			\begin{align*}
				\expval{a^\dagger a a^\dagger a}{n} = n^2
			\end{align*}
			\begin{align*}
				\expval{a^\dagger a^\dagger a a}{n} &= \mel{n}{a^\dagger a^\dagger a}{n-1} \sqrt{n} \\
				&= \mel{n}{a^\dagger a^\dagger}{n-2} \sqrt{n-1} \sqrt{n} \\
				&= \mel{n}{a^\dagger}{n-1} \sqrt{n-1}\sqrt{n-1}\sqrt{n} \\
				&= \braket{n}{n} \sqrt{n}\sqrt{n-1}\sqrt{n-1}\sqrt{n} \\
				&= n (n-1)
			\end{align*}
			Putting everything together
			\begin{align*}
				E_n^{(1)} &= \frac{\epsilon_4 \hbar^2}{4 m^2 \omega^2} \qty[(n+1)(n+2)  +  (n+1)^2  + n(n+1)  +  n(n+1)  +  n^2  + n(n-1)]\\
				&= \frac{\epsilon_4 \hbar^2}{4 m^2 \omega^2} \qty(6 n^2 + 6 n + 3)
			\end{align*}
			
			
			
			
			\item Linear harmonic oscillator of charge $q=+e$ ($e$ is positive) and mass $m$ perturbed by a uniform electric field $E$ in the $x$ direction.\\
			
			\underline{Ans.}\\
			%% page 48
			\begin{figure}
				%% TODO
				\centering
				\includegraphics[width=0.5\linewidth]{Pictures/not-found.jpg}
				\caption{Linear harmonic oscillator in electric field $E$}
			\end{figure}
		\begin{equation}
			V(x) = V(0) - q E x
		\end{equation}
		Assuming $V(0) = 0$
		\begin{equation}
			V(x) =  - q E x
		\end{equation}
			Here $V(x)$ is the electric potential energy of the oscillator. We will take this as a perturbation. The full Hamiltonian of the system is then
			\begin{equation}
				H = H_0 + V(x) = - \frac{\hbar^2}{2 m} \dv[2]{}{x} + \frac{1}{2} m \omega x^2 - q E x
			\end{equation}
			The first order correction to the energy of the $n^{th}$ level of the harmonic oscillator due to its interaction with the electric field is then
			\begin{equation}
				E_n^{(1)} = \expval{V}{n} = -e E \expval{x}{n} = 0
			\end{equation}
			because the integral is odd in the coordinate representation.\\
			
			We now calculate the energy correction in the $2nd$ order.
			\begin{equation}
				E_n^{(2)} = \sum_{\substack{m \\ m \neq n}} \frac{\abs{V_{m n} }^2}{E_n^{(0)} - E_m^{(0)}}
			\end{equation}
			Here
			\begin{equation}
				V_{m n} = \mel{m}{V}{n} = - e E \mel{m}{x}{n}
			\end{equation}
			Therefore
			\begin{equation}
				E_n^{(2)} = \left(-e E\right)^2 \sum_{\substack{m \\ m \neq n}} \frac{\abs{\mel{m}{x}{n}}^2}{E_n^{(0)} - E_m^{(0)}}
			\end{equation}
			Next
			\begin{equation}
				x = \left(\frac{\hbar}{2 m \omega}\right)^{1/2} \qty(a + a^\dagger)
			\end{equation}
			\begin{align*}
			\therefore \mel{m}{x}{n} 
			&= \left(\frac{\hbar}{2 m \omega}\right)^{1/2} \mel{m}{a + a^\dagger}{n} \\
			&= \left(\frac{\hbar}{2 m \omega}\right)^{1/2} \left[\sqrt{n} \braket{m}{n-1}  +  \sqrt{n+1}\braket{m}{n+1}\right] \\
			&= \left(\frac{\hbar}{2 m \omega}\right)^{1/2} \left[\sqrt{n} \delta_{m,n-1} + \sqrt{n+1} \delta_{m, n+1}\right]
			\end{align*}
			Hence
			\begin{align*}
				E_n^{(2)} 
				&= e^2 E^2 \sum_{\substack{m \\ m \neq n}} \frac{\abs{\mel{m}{x}{n}}^2}{E_n^{(0)} - E_m^{(0)} }\\
				&= e^2 E^2 \left[
				\frac{\abs{\mel{n-1}{x}{n}}^2}{E_n^{(0)}-E_{n-1}^{(0)}}
				+
				\frac{\abs{\mel{n+1}{x}{n}}^2}{E_n^{(0)}-E_{n+1}^{(0)}}
				\right]\\
				&= e^2 E^2 \left[\frac{n}{\hbar \omega} + \frac{n+1}{-\hbar\omega}\right] \left(\frac{\hbar}{2 m \omega}\right)\\
				&= - \frac{e^2 E^2}{2 m \omega^2}
			\end{align*}
			The correction to the energy of the $n^{th}$ level up to second order is then
			\begin{align*}
				E_n &=	E_n^{(0)}  + E_n^{(1)} + E_n^{(2)} \\
				&= \qty(n + \frac{1}{2}) \hbar\omega + 0 - \frac{e^2 E^2}{2 m \omega^2}\\
				&= \qty(n + \frac{1}{2}) \hbar\omega - \frac{e^2 E^2}{2 m \omega^2}
			\end{align*}
			
			%% page 50
			In the present situation the problem can be solved exactly merely by shifting the origin. This can be easily seen as follows:
			\begin{align*}
				H = H_0 + V 
				&= - \frac{\hbar^2}{2 m} \dv[2]{}{x} + \frac{1}{2} m \omega^2 x^2 - e E x\\
				&= -\frac{\hbar^2}{2 m}\dv[2]{}{x} + \frac{1}{2} m \omega^2 \qty(x^2 - \frac{2 e E}{m \omega^2}x)
			\end{align*}
			Now, we can write $\qty(x^2 - \frac{2 e E}{m \omega^2}x)$ as follows
			\begin{align*}
				\qty(x^2 - \frac{2 e E}{m \omega^2}x) 
				= x^2 - 2 \frac{e W}{m \omega^2} x + \qty(\frac{e E}{m \omega^2})^2 - \qty(\frac{e E}{m \omega^2})^2
				= \qty(x - \frac{e E}{m \omega^2})^2 - \qty(\frac{e E}{m \omega^2})^2
			\end{align*}
			Therefore, the Hamiltonian $H$ can be written as
			\begin{align*}
				H &= -\frac{\hbar^2}{2 m} \dv[2]{}{x}  + \frac{1}{2} m \omega^2 \qty(x - \frac{e E}{m \omega^2})^2 - \frac{1}{2} m \omega^2 \frac{e^2 E^2}{m^2 \omega^4} \\
				&= -\frac{\hbar^2}{2 m} \dv[2]{}{x} +  \frac{1}{2} m \omega^2 \qty(x - \frac{e E}{m \omega^2})^2 - \frac{e^2 E^2}{2 m \omega^2}
			\end{align*}
			Let $\xi = x - \frac{e E}{m \omega^2}$
			\begin{equation*}
				\therefore H = -\frac{\hbar^2}{2 m} \dv[2]{}{\xi} +  \frac{1}{2} m \omega^2 \xi^2 - \frac{e^2 E^2}{2 m \omega^2}
			\end{equation*}
			Thus, the exact eigenvalue spectrum is
			\begin{equation}
				E_n = \qty(n + \frac{1}{2}) \hbar \omega - \frac{e^2 E^2}{2 m \omega^2}\ ;\quad n=0,1,2,\ldots
			\end{equation}
			In the present problem, the second order perturbation theory to give the correct result.
			
			
			\item Calculate the shift in energy of the $1 s$ state (i.e. the ground state) of hydrogen if the proton is assumed to be a uniformly charged spherical shell of radius $10^{-15} m$ rather than a point charge. Use first order perturbation theory.\\
			
			\underline{Ans.}\\
			%% page 52
			\begin{figure}
				%%% TODO
				\centering
				\includegraphics[width=0.5\linewidth]{Pictures/not-found.jpg}
				\caption{hydrogen atom}
			\end{figure}
			Unperturbed system: proton is a point charge. The Hamiltonian  of the unperturbed system is
			\begin{equation}
				H_0 = - \frac{\hbar^2}{2 m} \nabla^2 + V_0(r) = - \frac{\hbar^2}{2 m} \nabla^2 - \frac{e^2}{4 \pi \epsilon_0 r}
			\end{equation}
			Perturbed system:\\
			
			\begin{figure}
				%%% TODO
				\centering
				\includegraphics[width=0.5\linewidth]{Pictures/not-found.jpg}
				\caption{hydrogen atom}
			\end{figure}
		The proton is a very thin shell of radius $a$. The value of $a$ is $10^{-15}m$. The potential for the perturbed system is
		\begin{equation}
			V(r) = \begin{cases}
			-\frac{e^2}{4 \pi \epsilon_0 r},\ r > a \\
			\frac{e^2}{4 \pi \epsilon_0 a}, \ r < a
			\end{cases}
		\end{equation}
			Therefore
			
			%% page 53
			\begin{align*}
				H &= - \frac{\hbar^2}{2 m} \nabla^2 + V(r) \\
				&= \begin{cases}
					- \frac{\hbar^2}{2 m} \nabla^2 - \frac{e^2}{4 \pi \epsilon_0 a} \quad \text{for} \ r<a \\
					- \frac{\hbar^2}{2 m} \nabla^2 - \frac{e^2}{4 \pi \epsilon_0 r} \quad \text{for} \ r>a \\
				\end{cases}
			\end{align*}
			The unperturbed Hamiltonian
			\begin{equation*}
				H_0 = - \frac{\hbar^2}{2 m} \nabla^2 - \frac{e^2}{4 \pi \epsilon_0 r} \quad \ \forall r
			\end{equation*}
			Therefore, perturbation $H^\prime$ is 
			\begin{align*}
				H^\prime 
				&= H - H_0 \\
				&= \begin{cases}
					- \frac{e^2}{4 \pi \epsilon_0 a} + \frac{e^2}{4 \pi \epsilon_0 r} \quad    &\text{for}\ r < a \\
					0 &\text{for}\ r > a
				\end{cases}
			\end{align*}
			Now, the ground state energy of the Hydrogen atom (taking the proton to be a point charge) is 
			\begin{equation*}
				E_0 = - \frac{e^2}{\qty(4 \pi \epsilon_0) 2 a_0} = - 16.6 eV
			\end{equation*}
			where
			\begin{equation*}
				a_0 = \frac{4\pi \epsilon_0 \hbar^2}{m e^2} = \text{Bohr radius} = 0.5 \times 10^{-10} m
			\end{equation*}
			
			The ground state wave function is
			\begin{equation*}
				\psi_{1s}^{(0)}\qty(r) = \frac{2}{\qty(4\pi)^{1/2} a_0^{3/2}} e^{-r / a_0}
			\end{equation*}
			The first-order correction is
			\begin{align*}
				E^{(1)} \equiv \Delta E
				&= \expval{H^\prime}{\psi_{1s}^{(0)}} \\
				&= \int \psi_{1s}^{(0)*}\qty(r) H^\prime \psi_{1s}^{(0)}\qty(r) \dd[3]{r}\\
				&= \frac{4}{4\pi a_0^3} 4\pi \int_0^a e^{-2r/a_0} \qty(\frac{e^2}{4\pi \epsilon_0 r} - \frac{e^2}{3\pi \epsilon_0 a}) r^2 \dd{r} \\
				&= \frac{4 e^2}{a_0^3 4\pi \epsilon_0} \int_{0}^{a} e^{-2 r/a_0} \qty(\frac{1}{r} - \frac{1}{a}) r^2 \dd{r}
			\end{align*}
			Here $a=10^{-15} m$. In the integral above
			\begin{equation*}
				\qty(\frac{r}{a_0})_{max} = \frac{a}{a_0} = \frac{10^{-15}}{0.5 \times 10^{-10}} = 2 \times 10^{-5} << 1
			\end{equation*}
			Therefore we can safely set $e^{-2 r/a_0} \approx 1$. The first order correction to the energy is then
			\begin{align*}
				\Delta E
				&= \frac{4 e^2}{4\pi \epsilon_0 a_0^3} \int_{0}^{a} \qty(\frac{1}{r} - \frac{1}{a}) r^2 \dd{r}\\
				&= \frac{4 e^2}{4\pi \epsilon_0 a_0^3} \qty(\frac{a^2}{2} - \frac{a^2}{3}) \\
				&= \frac{4 e^2 a^2}{4\pi \epsilon_0 a_0^3} \frac{1}{6}\\
				&= \frac{4}{3} \qty(\frac{e^2}{\qty(4\pi \epsilon_0 2 a_0)}) \frac{a^2}{a_0^2}
			\end{align*}
			\begin{equation*}
				\Delta E = \frac{4}{3} E_H \frac{a^2}{a_0^2}
			\end{equation*}
			with
			\begin{equation*}
				E_H \equiv \frac{e^2}{\qty(4\pi \epsilon_0) 2 a} = 13.6 eV
			\end{equation*}
			Numerically
			\begin{equation}
				\Delta E = \frac{4}{3} \qty(13.6 eV) \qty(\frac{10^{-15}}{0.5 \times 10^{-10}})^2 \sim 7 \times 10^{-9} eV
			\end{equation}
			The ground state has increased in energy, but the increase is very small.
			
			
			\item Calculate the shift in the ground state energy of the hydrogen atom if the proton is considered as a uniform sphere of radius $R$ instead than a point charge. \\
			
			\underline{Ans.}\\
			%% page 59
			
			\begin{figure}
				%%%% TODO figure
				\centering
				\includegraphics[width=0.5\linewidth]{Pictures/not-found.jpg}
				\caption{Hydrogen atom}
			\end{figure}
		
			\begin{align*}
				E_{1 s}^{(0)} = -\frac{e^2}{4\pi \epsilon_0 2 a_0} = - E_H = - 13.6 eV \\
				\psi_{1 s}^{(0)} = \frac{2}{\sqrt{4 \pi a_0^3} } e^{-r/a_0}\\
				a_0 = \frac{4 \pi \epsilon_0 \hbar^2}{m e^2}
			\end{align*}
			\underline{Perturbed system}
			
			\begin{figure}
				%%%% TODO figure. page 60
				\centering
				\includegraphics[width=0.5\linewidth]{Pictures/not-found.jpg}
				\includegraphics[width=0.5\linewidth]{Pictures/not-found.jpg}
				\caption{Hydrogen atom}
			\end{figure}
		
			\begin{align*}
				V\qty(r) = - \frac{e^2}{4 \pi \epsilon_0 r} \quad \text{for}\ r \geq R
			\end{align*}
			Let us calculate the potential (not the potential energy) of the electric field of the proton charge distribution for $r<R$
			
			Using Gauss's law for the sphere of radius $r<R$
			%% page 61
			\begin{align*}
				E 4 \pi r^2 = \frac{q_{enclosed}}{\epsilon_0} = \frac{1}{\epsilon_0} \frac{4}{3} \pi r^3 \rho = \frac{1}{\epsilon_0} \frac{4}{3} \pi r^3 \frac{e}{\frac{4}{3} \pi R^3} = \frac{e r^3}{\epsilon_0 R^3}
			\end{align*}
			\begin{align*}
				\therefore E = \frac{e r}{4\pi \epsilon_0 R^3}
			\end{align*}
			Let $v(r)$ be the potential. Therefore 
			\begin{align*}
				\vec{E} &= - \grad{ v} \\
				or, \ \dv{v}{r} &= -\frac{e r}{4 \pi \epsilon_0 R^3} \\
				\therefore v = - \frac{e r^2}{2 \qty(4 \pi \epsilon_0) R^3} + c
			\end{align*}
			Since $v(R) = \frac{e}{4 \pi \epsilon_0 R}$
			\begin{align*}
				\frac{e}{4\pi \epsilon_0 R} = - \frac{e}{2 \qty(4 \pi \epsilon_0) R}  + c \\
				\therefore c = \frac{3 e}{2 \qty(4 \pi \epsilon_0) R}
			\end{align*}
			
			\begin{align*}
				v(r) 
				&= \frac{- e r^2}{2 \qty(4 \pi \epsilon_0) R^3} + \frac{3 e}{2 \qty(4 \pi \epsilon_0) R} \\
				&= \frac{e}{2 \qty(4 \pi \epsilon_0) R} \qty(3 - \frac{r^2}{R^2}) \quad (r < R)
			\end{align*}
			Therefore, potential energy of the electron for $r<R$ is
			\begin{align*}
				V(r) = - e v(r) = - \frac{e^2}{2 \qty(4 \pi \epsilon_0) R} \qty(3 - \frac{r^2}{R^2}) \quad (r < R)
			\end{align*}
			Thus
			\begin{equation*}
				V(r) = \begin{cases}
				- \frac{e^2}{2 \qty(4 \pi \epsilon_0) R} \qty(3 - \frac{r^2}{R^2}) \quad (r < R) \\- \frac{e^2}{4 \pi \epsilon_0 r} \quad (r > R)
				\end{cases}
			\end{equation*}
			
			%% page 63
			Also
			\begin{align*}
				V_0 = - \frac{e^2}{4 \pi \epsilon_0 r} \ \forall r
			\end{align*}
			The perturbation is then
			\begin{align*}
				H^\prime 
				&= V - V_0 \\
				&= \begin{cases}
				- \frac{e^2}{2 \qty(4 \pi \epsilon_0) R} \qty(3 - \frac{r^2}{R^2}) + \frac{e^2}{4 \pi \epsilon_0 r} \quad (r < R) \\ 0 \quad (r > R)
				\end{cases}\\
				&= \begin{cases}
				\frac{e^2}{2 \qty(4 \pi \epsilon_0) R} \qty(-\frac{3}{2} + \frac{r^2}{2 R^2}  + \frac{R}{r}) \quad (r \leq R)
				\\ 0 \quad (r \geq R)
				\end{cases}
			\end{align*}
			Using this perturbation we can calculate the first order correction to the ground state energy in the previous problem.
			
			%% page 64
			\begin{align*}
				E^{(1)} = \delta E^{(1)} = \expval{H^\prime}{\psi_{1 s}^{(1)}}
			\end{align*}
			The ground state unperturbed wave function is
			\begin{align*}
				\psi_{1 s}^{(0)}\qty(r) = \frac{1}{\sqrt{r a_0^3}} e^{-r/a_0}
			\end{align*}
			which is independent of $\theta, \phi$. Here $a_0$ is the Bohr radius.
			
			\begin{align*}
				\therefore \delta E^{(1)} = \frac{4 \pi}{\pi a_0^3} \qty(\frac{e^3}{4 \pi \epsilon_0 R^3}) \int_{0}^{R} \qty(-\frac{3}{2} + \frac{r^2}{2 R}  +  \frac{R}{r}) e^{-2 r / a_0} r^2 \dd{r}
			\end{align*}
			Now $r_{max} = R = 10^{-15} m$ and $a_0 \approx 10^{-10} m$. Therefore $2 r  / a_0$ is very small, so $e^{-2 r / a_0} \approx 1$.\\
			
			The first order correction is then
			\begin{equation*}
				\delta E^{(1)} = \frac{4 \pi}{\pi a_0^3} \qty(\frac{e^2}{4\pi \epsilon_0 R}) \int_{0}^{R} \qty(-\frac{3}{2} + \frac{r^2}{2 R}  +  \frac{R}{r}) e^{-2 r / a_0} r^2 \dd{r}
			\end{equation*}
			Carrying out the integral and simplifying we get
			\begin{align*}
				\delta E^{(1)} 
				&= \frac{4}{5} \qty(\frac{e^2}{4\pi \epsilon_0 2 a_0}) \frac{R^2}{a_0^2} \\
				&= \frac{4}{5} E_H \qty(\frac{R^2}{a_0^2})
			\end{align*}
			where $E_H =\frac{e^2}{4\pi \epsilon_0 2 a_0}= 13.6 eV$
		
		
	
			
			\item Ground state of helium-type atoms.
			\begin{figure}
				%%%% TODO figure
				\centering
				\includegraphics[width=0.5\linewidth]{Pictures/not-found.jpg}
				\caption{Helium atom}
			\end{figure}
			The nucleus is taken to be a point charge with charge of $Z e$ ($e=+1.6 \times 10^{-19} C$).\\
			The Hamiltonian of the system is taken as
			\begin{align}
				H &= -\frac{\hbar^2}{2 m} \qty(\nabla_1^2  +  \nabla_2^2) - \frac{Z e^2}{4 \pi \epsilon_0} \qty(\frac{1}{r_1} + \frac{1}{r_2}) + \frac{e^2}{4 \pi \epsilon_0 \abs{\vec{r_1} - \vec{r_2}}}\\
				&= H_0 + V
			\end{align}
			Here 
			\begin{equation}
			 	V = \frac{e^2}{4 \pi \epsilon_0 \abs{\vec{r_1} - \vec{r_2}}}
			\end{equation}
			
			%% page 66
			where the unperturbed Hamiltonian is
			\begin{align*}
				H_0 &= \qty(-\frac{\hbar^2}{2 m} \nabla_1^2 - \frac{Z e^2}{4\pi \epsilon_0 r_1}) + \qty(-\frac{\hbar^2}{2 m} \nabla_2^2 - \frac{Z e^2}{4\pi \epsilon_0 r_2})\\
			i.e., \	H_0 &= H_0(1) + H_0(2)
			\end{align*}
			The perturbing potential $V$ is
			\begin{equation*}
				V = \frac{e^2}{4 \pi \epsilon_0 \abs{\vec{r_1} - \vec{r_2}}} = \frac{e^2}{4 \pi \epsilon_0 r_{1 2}}
			\end{equation*}
			The unperturbed ground state energy is
			\begin{align*}
				E_0 
				&= - \frac{Z^2 e^2}{\qty(4 \pi \epsilon_0) 2 a_0} - \frac{Z^2 e^2}{\qty(4 \pi \epsilon_0) 2 a_0}\\
				&= - 2 Z^2 E_H
			\end{align*}
			Where $E_H = \frac{e^2}{\qty(4 \pi \epsilon_0) 2 a_0} = 13.6 eV$ and $a_0 = \frac{4 \pi \epsilon_0  \hbar^2}{m e^2}$ is the Bohr radius.\\
			
			
			The corresponding unperturbed eigen function is the product of the eigenfunctions of each electron (neglecting anti-symmetry):
			\begin{align*}
				\ket{E_0^{(0)}} \doteq \psi_0 = \psi_{1 s}(1) \psi_{1 s}(2) 
				= \frac{1}{\sqrt{\pi}} \qty(\frac{Z}{a_0})^{3/2} e^{- Z r_1 / a_0} 
				\frac{1}{\sqrt{\pi}} \qty(\frac{Z}{a_0})^{3/2} e^{- Z r_2 / a_0}
			\end{align*}
			\begin{align*}
			i.e., \psi_0 = \frac{1}{\pi} \qty(\frac{Z}{a_0})^{3} e^{-Z\qty(r_1 + r_2) / a_0}
			\end{align*}
			
			The first-order correction to energy is
			
			%%% page 68
			\begin{align*}
				E^{(1)} 
				&= \expval{V}{E_0^{(0)}} \\
				&= \int \psi_0^{*}\qty(r_1 r_2) V \psi_0\qty(r_1 r_2) \dd[3]{r_1} \dd[3]{r_2} \\
				&= \frac{1}{\pi^2} \qty(\frac{Z}{a_0})^6 \int e^{-Z (r_1 + r_2)/a_0} \frac{e^2}{4\pi \epsilon_0 r_{1 2}} e^{-Z (r_1 + r_2)/a_0} \dd[3]{r_1} \dd[3]{r_2}\\
				&= \frac{e^2}{\qty(4\pi \epsilon_0) \pi^2} \qty(\frac{Z}{a_0})^6 \int e^{- 2 Z r_1/a_0} \frac{1}{r_{1 2}} e^{- 2 Z r_2/a_0} \dd[3]{r_1} \dd[3]{r_2}
			\end{align*}
			Let us now make changes in the variables of the integration in the following manner
			\begin{align*}
				\vec{\rho}_1 = \frac{2 Z}{a_0} \vec{r}_1 \\
				\vec{\rho}_2 = \frac{2 Z}{a_0} \vec{r}_2
			\end{align*}
			Therefore
			\begin{align*}
				r_{1 2} &= \abs{\vec{r}_1 - \vec{r}_2} = \frac{a_0}{2 Z} \abs{\vec{\rho}_1 - \vec{\rho}_2} = \frac{a_0}{2 Z} \rho_{1 2}\\
				\dd[3]{r_1} &= \qty(\frac{a_0}{2 Z})^3 \dd[3]{\rho_1} \\
				\dd[3]{r_2} &= \qty(\frac{a_0}{2 Z})^3 \dd[3]{\rho_2}
			\end{align*}
			With these change of variables $E^{(1)}$ can be written as
			\begin{align*}
				E^{(1)} = \frac{e^2}{\qty(4 \pi \epsilon_0) \pi^2} \qty(\frac{Z}{a_0})^6 \frac{2 Z}{a_0} \qty(\frac{a_0}{2 Z})^6  \int e^{-\rho_1} \frac{1}{\rho_{1 2}} e^{-\rho_2} \dd[3]{\rho_1} \dd[3]{\rho_2}
			\end{align*}
			Simplifying
			\begin{align*}
				E^{(1)} = \frac{e^2}{\qty(4 \pi \epsilon_0) \pi^2} \frac{2 Z}{a_0} \frac{1}{2^6} \int e^{-\qty(\rho_1 + \rho_2)} \frac{1}{\rho_{1 2}}  \dd[3]{\rho_1} \dd[3]{\rho_2}
			\end{align*}
			Now, we can show
			\begin{equation*}
				\int e^{-\qty(\rho_1 + \rho_2)} \frac{1}{\rho_{1 2}}  \dd[3]{\rho_1} \dd[3]{\rho_2} = 20 \pi^2
			\end{equation*}
			Therefore, $E^{(1)}$ becomes
			\begin{align*}
				E^{(1)} 
				&= \frac{e^2}{\qty(4 \pi \epsilon_0) \pi^2} \frac{2 Z}{a_0} \frac{1}{2^6} 20 \pi^2 \\
				&= \frac{e^2}{\qty(4 \pi \epsilon_0)} \cdot \frac{2 Z}{ a_0} \cdot \frac{1}{64} \cdot 20 \\
				&= \frac{5}{4} \cdot \frac{Z e^2}{\qty(4 \pi \epsilon_0) 2 a_0} \\
			\therefore E^{(1)}	&= \frac{5}{4} Z E_H
			\end{align*}
			Hence, upto first order in perturbation theory
			\begin{equation*}
				E = E_0 + E^{(1)} = - 2 Z^2 E_H + \frac{5}{4} Z E_H = - \qty(Z - \frac{5}{8}) 2 Z E_H
			\end{equation*}
			
			
			
			
			
		\end{enumerate}
	
	\section{Examples of Degenerate Perturbation Theory}
	\subsection{Stark Effect in Hydrogen Atom}
	%% page 71
	Stark effect is the splitting of atomic energy levels due to an applied electric field. Let the electric field be uniform and along the $z$-axis.
	\begin{figure}
		%% TODO
		\centering
		\includegraphics[width=0.5\linewidth]{Pictures/not-found.jpg}
		\caption{stark effect}
	\end{figure}
	The perturbation is 
	\begin{equation}
		H^\prime = \abs{\vec{F}} Z = \abs{q_e} E Z = e E Z
	\end{equation}
	Here $q_e = - e$ and $e= + 1.6\times 10^{-19} C$. The unperturbed Hamiltonian is
	\begin{align}
		H_0 = - \frac{\hbar^2}{2 m}\nabla^2 - \frac{e^2}{4\pi \epsilon_0 r}
	\end{align}
	The eigenvalues and eigenfunctions of $H_0$ are known.
	\begin{equation}
		H_0 \psi_{n l m}^{(0)}\qty(\vec{r}) = E_n^{(0)} \psi_{n l m}^{(0)}\qty(\vec{r})
	\end{equation}
	%% 72
	with
	\begin{align*}
		E_n^{(0)} = - \frac{e^2}{\qty(4 \pi \epsilon_0) 2 a_0} \frac{1}{n^2} \ ; \ n = 1,2,3, \ldots \\
		a_0 = \frac{\qty(4 \pi \epsilon_0) h^2}{m e^2} \quad \qty(m=\text{mass of electron})
	\end{align*}
	
	All levels (except the ground level) are degenerate \footnote{We neglect the spin degrees of freedom}. This is because for a given $n$
	\begin{align*}
		l &= 0,1,2,\ldots,(n-1) \\
		and\quad m &= -l,-l+1. \ldots, l-1, l
	\end{align*}
	The ground state wave function is 
	\begin{equation*}
		\psi_{1 0 0}^{(0)}\qty(r) = \frac{1}{\sqrt{\pi a_0^3}} e^{-r / a_0}
	\end{equation*}
	Which is independent of $\theta$ and $\phi$.\\
	
	The first-order correction to the ground state energy is
	\begin{equation*}
		E_{100}^{(1)} = \expval{H^\prime}{\psi_{100}} = 0
	\end{equation*}
	
	\underline{Level $n=2$}\\
	The first excited state of the unperturbed hydrogen atom is four-fold degenerate. For $n=2$, $l$ can have values $l=0,1$. For $l=0$, $m=0$, while for $l=1$, $m=1,0,-1$. The degenerate unperturbed eigenfunctions are
	$\psi^{(0)}_{2 0 0},  \psi^{(0)}_{2 1 0}, \psi^{(0)}_{2 1 1}, \psi^{(0)}_{2 1 -1}$  . A general unperturbed wave function is of the form
	\begin{equation}
		\psi_{n l m}^{(0)} = R_{n l}\qty(r) Y_{l m}\qty(\theta, \phi)
	\end{equation}
	We now have to diagonalize $H^\prime$ in the four dimensional eigen subspace of $E_2^{(0)}$. The eigenvalues of $H^\prime$ are the first order correction to the energy and the eigenvectors of $H^\prime$ are the correct zero-order approximation of the perturbed wave function.\\
	
	\underline{Matrix representation of $H^\prime$}\\
	Using the degenerate zeroth-order wave functions $\psi^{(0)}_{2 0 0},  \psi^{(0)}_{2 1 0}, \psi^{(0)}_{2 1 1}$ and $\psi^{(0)}_{2 1 -1}$  as basis we can work out the matrix element of $H^\prime$. We use the four wave functions in the following order
	\begin{align}
	\begin{aligned}
		\ket{1} &\equiv \ket{200} \doteq \psi_{200}^{(0)} \\
		\ket{2} &\equiv \ket{210} \doteq \psi_{210}^{(0)} \\
		\ket{3} &\equiv \ket{211} \doteq \psi_{211}^{(0)} \\
		\ket{4} &\equiv \ket{21-1} \doteq \psi_{21-1}^{(0)}
	\end{aligned}
		\label{chapter20.ex.degenerate.eqn1}
	\end{align}
	A general matrix element of $H^\prime$ is of the form 
	\begin{equation*}
		\mel{n l m}{H^\prime}{n^\prime l^\prime m^\prime}\quad \text{with}\ n=2
	\end{equation*}
	Before working out the matrix elements, consider the following:
	\begin{enumerate}
		\item 	Suppose $l=l^\prime$. This case includes all the diagonal elements $(m=m^\prime)$ and some non-diagonal elements $(m\neq m^\prime)$. In this case, the parity of the integrand of the matrix element is
		\begin{equation*}
			\qty(-1)^{2l + 1} = -1 \quad \text{odd}
		\end{equation*}
		since $H^\prime = e E Z = e E r \cos\theta \sim Y_{21}$ has parity $(-1)$. Therefore, the integral, i.e., the matrix element is zero.
		
		
		\item Next, note that $H^\prime = e E r\cos\theta$ is invariant under a rotation about the $z$-axis which only changes the azimuthal angle $\phi$ keeping $r$ and $\theta$ unchanged. Therefore
		\begin{equation*}
		\comm{H^\prime}{L_z} = 0
		\end{equation*}
	\end{enumerate}
	%% page 76
	Now, consider the matrix element of the commutator above
	\begin{align*}
		\mel{n l m}{\comm{H^\prime}{L_z}}{n l^\prime m^\prime} &= 0 \\
		\mel{n l m}{H^\prime L_z - L_z H^\prime}{n l^\prime m^\prime} &= 0 \\
		\qty(m^\prime - m)\mel{n l m}{H^\prime }{n l^\prime m^\prime} &= 0 
	\end{align*}
	i.e., if $m^\prime \neq m$, Then
	\begin{equation*}
		\mel{n l m}{H^\prime}{n l^\prime m^\prime} = 0
	\end{equation*}
	The upshot of the argument above is that $\mel{n l m}{H^\prime}{n l^\prime m^\prime} = 0$ if $l=l^\prime$ and if $m^\prime \neq m$. Thus all the matrix elements of $H^\prime$ except two are non-zero. These are $\mel{2 0 0}{H^\prime}{2 1 0}$ and $\mel{2 1 0}{H^\prime}{2 0 0}$ and these elements are complex conjugates of each other since $H^\prime$ is Hermitian.\\
	
	Using the order of the unperturbed wave function indicated above (eqn (\ref{chapter20.ex.degenerate.eqn1})), The matrix representation of $H^\prime$ in the eigen subspace of $E_2^{(0)}$ is
	
	\begin{equation}
	H^\prime = 
	\begin{blockarray}{ccccc}
	\ & \ket{1} & \ket{2} & \ket{3} & \ket{4}  \\
	\begin{block}{c[cccc]}
	\bra{1} & 0 & \mel{2 0 0}{H^\prime}{2 1 0} & 0 & 0  \\
	\bra{2} & \mel{2 1 0}{H^\prime}{2 0 0} & 0 & 0 & 0  \\
	\bra{3} & 0 & 0 & 0 & 0  \\ 
	\bra{4} & 0 & 0 & 0 & 0 \\
	\end{block}
	\end{blockarray}
	\end{equation}
	
	
	
	
	\subsubsection{calculation of $\mel{2 0 0}{H^\prime}{2 1 0}$}
	We have to calculate
	\begin{align*}
		\mel{2 0 0}{H^\prime}{2 1 0} 
		&= \int \psi_{2 0 0}^{(0)*} \qty(\vec{r}) e E Z \psi_{2 1 0}^{(0)}\qty(\vec{r})  \dd[3]{r} \\
		&= e E \int \psi_{2 0 0}^{(0)*} \qty(\vec{r}) r \cos \theta \psi_{2 1 0}^{(0)}\qty(\vec{r})  \dd[3]{r}
	\end{align*}
	Now
	\begin{align*}
		\psi_{2 0 0}^{(0)} 
		&= R_{2 0}\qty(r) Y_{0 0}\qty(\theta \phi) \\
		&= \frac{1}{\sqrt{4 \pi}} R_{2 0}\qty(r) \\
		&= \frac{1}{\sqrt{4 \pi}} \qty(\frac{1}{2 a_0})^{3/2} \qty(2 - r/a_0) e^{-r/2 a_0}
	\end{align*}
	
	\begin{align*}
		\psi_{2 1 0}^{(0)} 
		&= R_{2 1}\qty(r) Y_{1 0}\qty(\theta \phi) \\
		&= \frac{1}{\sqrt{3}} \qty(\frac{1}{2 a_0})^{3/2} \qty(\frac{r}{a_0}) e^{-r/2 a_0} Y_{1 0}\qty(\theta \phi)\\
		&= \frac{1}{\sqrt{3}} \qty(\frac{1}{2 a_0})^{3/2} \qty(\frac{r}{a_0}) e^{-r/2 a_0} \sqrt{\frac{3}{4 \pi}} \cos \theta\\
		&= \frac{1}{\sqrt{4\pi}} \qty(\frac{1}{2 a_0})^{3/2} \qty(\frac{r}{a_0}) e^{-r / 2 a_0} \cos\theta
	\end{align*}
	%% page 79
	\begin{align*}
		\mel{2 0 0}{H^\prime}{2 1 0} = e E \frac{1}{4 \pi} \frac{1}{\qty(2 a_0)^3 a_0} \int_{0}^{\infty} r^4 \qty(2 - r/a_0) e^{-r/a_0} \dd{r} \int_{\Omega} \cos^2\theta \sin\theta \dd{\theta} \dd{\phi}
	\end{align*}
	Now
	\begin{align*}
		\int_{0}^{2 \pi} \dd{\phi} &= 2 \pi \\
		\int_{0}^{\pi} \cos^2 \theta \sin \theta\dd{\theta} = \int_{-1}^{+1} \mu^2 \dd{\mu} = \frac{2}{3} \quad (\text{substituting } \mu=\cos\theta )
	\end{align*}
	Therefore
	\begin{align*}
		\mel{2 0 0}{H^\prime}{2 1 0} 
		&=  \frac{e E}{4 \pi \qty(8 a_0^3) a_0} \qty(2\pi) \frac{2}{3} \int_{0}^{\infty} r^4 \qty(2 - r/a_0) e^{-r/a_0} \dd{r} \\
		&= \frac{e E}{24 a_0^4} \int_{0}^{\infty} r^4 \qty(2 - r/a_0) e^{-r / a_0} \dd{r}
	\end{align*}
	Let $r/a_0 = x$
	\begin{align*}
		\mel{2 0 0}{H^\prime}{2 1 0} 
		&= \frac{e E}{24 a_0^4} \int_{0}^{\infty} a_0^4 x^4 \qty(2 - x) e^{-x} a_0\dd{x} \\
		&= \frac{e E a_0}{24} \int_{0}^{\infty}  x^4 \qty(2 - x) e^{-x} \dd{x} \\
		&= \frac{e E a_0}{24} \left[
		2 \int_{0}^{\infty} x^4 e^{-x} \dd{x}
		- \int_{0}^{\infty} x^5 e^{-x} \dd{x}
		\right]
		&= \frac{e E a_0}{24} \left[ 2 \times 4! - 5!\right] \\
		&= -3 e E a_0
	\end{align*}
	The gamma integral from appendix (\ref{appendix1.Integrals}) is used.
	
	Hence
	%% page 81
	
	\begin{equation}
		H^\prime \doteq 
		\left[
		\begin{matrix}
		0 & -3 e E a_0 & 0 & 0 \\
		-3 e E a_0 & 0 & 0 & 0 \\
		0 & 0 & 0 & 0 \\
		0 & 0 & 0 & 0 
		\end{matrix}
		\right]
	\end{equation}
	The eigenvalues of $H^\prime$ are the first-order corrections to energy of the unperturbed $n=2$ levels of the hydrogen atom.




	\subsubsection{Eigenvalues of $H^\prime$}
		
	The seculer equation is
	\begin{align*}
		\det [H^\prime - E I] &= 0 \\
		\mdet{-E & -3 e E a_0 & 0 & 0 \\
			-3 e E a_0 & -E & 0 & 0 \\
			0 & 0 & -E & 0\\
			0 & 0 & 0 & -E
	} &= 0 \\
	or,\ E^2 \qty(E^2 - q e^2 E^2 a_0^2) &= 0
	\end{align*}
	There are four roots of $E$, they are
	\begin{align*}
		E = - 3 e E a_0,\ 3 e E a_0,\ 0,\ 0
	\end{align*}
	Thus, the first-order corrections to the energy- level are
	\begin{align*}
		E_{2 1}^{(1)} &= - 3 e E a_0 \\		
		E_{2 2}^{(1)} &=  3 e E a_0 \\
		E_{2 3}^{(1)} &=  0 \\
		E_{2 4}^{(1)} &=  0 \\
	\end{align*}
	
	%%% page  82
	
	\begin{figure}
		%% TODO figure
		\centering
		\includegraphics[width=0.5\linewidth]{Pictures/not-found.jpg}
		\caption{Splitting of the energy levels}
	\end{figure}

	\subsubsection{Eigenvectors of $H$ in the zeroth order}
		The zeroth order eigen functions of $H$ are the eigenvectors of $H^\prime$. There are four eigenvalues of $H^\prime$ (two of them are equal, namely zero). We have to find each eigenvector.\\
		
		\underline{Eigenvalue $E_{n 1}^{(1)} = - 3 e E a_0 \ (n=2)$}\\
		Let the eigenvector be $(a_1, a_2, a_3, a_4)$. These components satisfy the equation
		\begin{equation*}
			\mqty[3 e E a_0 & - 3 e E a_0 & 0 & 0 \\
			-3 e E a_0 & 3 e E a_0 & 0 & 0\\
			0 & 0 & 3 e E a_0 & 0\\
			0 & 0 & 0 & 3 e E a_0
			]\mqty[a_1 \\ a_2 \\ a_3 \\ a_4] = 0
		\end{equation*}
		Therefore we have
		\begin{equation*}
			a_1 = a_2 \\
			a_3 = a_4 = 0\\
		\end{equation*}
		The normalization condition gives us $a_1 = a_2 = \frac{1}{\sqrt{2}}$.\\
		
		The components $(a_1, a_2, a_3, a_4)$ form the matrix representation of $\chi_{n 1}^{(0)}$ in the basis $\{\ket{1}, \ket{2}, \ket{3}, \ket{4}\}$ where
		\begin{align*}
			\ket{1} = \ket{2 0 0} = \psi_{2 0 0}^{(0)}\\
			\ket{2} = \ket{2 1 0} = \psi_{2 1 0}^{(0)}\\
			\ket{3} = \ket{2 1 1} = \psi_{2 1 1}^{(0)}\\
			\ket{4} = \ket{2 1 -1} = \psi_{2 1 -1}^{(0)}
		\end{align*}
		\begin{align*}
		\therefore \ \chi_{n 1}^{(0)} &= a_1 \psi_{2 0 0}^{(0)} + a_2 \psi_{2 1 0}^{(0)} \\		
		i.e., \ \chi_{n 1}^{(0)} &= \frac{1}{2} \qty(\psi_{2 0 0}^{(0)} + \psi_{2 1 0}^{(0)})
		\end{align*}
		
		
		
		\underline{Eigenvalue $E_{n 2}^{(1) = + 3 e E a_0}\ (n=2)$}\\
		The eigenvalue equation is
		\begin{equation*}
		\mqty[-3 e E a_0 & - 3 e E a_0 & 0 & 0 \\
		-3 e E a_0 & -3 e E a_0 & 0 & 0\\
		0 & 0 & -3 e E a_0 & 0\\
		0 & 0 & 0 & -3 e E a_0
		]\mqty[a_1 \\ a_2 \\ a_3 \\ a_4] = 0
		\end{equation*}
		We have
		\begin{align*}
		 a_3 = a_4 = 0 \\
		 a_1 = - a_2 = \frac{1}{\sqrt{2}}
		\end{align*}
		\begin{align*}
		\therefore \ \chi_{n 2}^{(0)} &= \frac{1}{2} \qty(\psi_{2 0 0}^{(0)} + \psi_{2 1 0}^{(0)})
		\end{align*}
		
		
		
		\underline{Eigenvalue $E_{n 3}^{(1)} = E_{n 4}^{(1)= 0} \ (n=2)$}\\
		The eigenvalue equation is
		\begin{equation*}
			\mqty[0 & - 3 e E a_0 & 0 & 0 \\
			-3 e E a_0 & 0 & 0 & 0\\
			0 & 0 & -3 e 0 & 0\\
			0 & 0 & 0 & 0
			]\mqty[a_1 \\ a_2 \\ a_3 \\ a_4] = 0
		\end{equation*}
		We have $ a_1 = a_2 = 0$ and $a_3$ and $a_4$ are arbitrary. Since we have two linearly independent eigenvectors, we can choose either
		\begin{align*}
			a_3 = 1 \quad \text{and} &\quad a_4 = 0 \\
			or\quad 			a_3 = 0 \quad \text{and} &\quad a_4 = 1
		\end{align*}
		Thus, the two linearly independent eigenvectors with degenerate eigenvalue $0$ are
		\begin{equation*}
			\mqty[0 \\ 0 \\ 1 \\ 0] \quad \text{and} \quad \mqty[0 \\ 0 \\ 0 \\ 1]
		\end{equation*}
		These two linearly independent eigenvectors are normalized and orthogonal. In the Hilbert space the eigenvectors are
		\begin{align*}
		\chi_{2 3}^{(0)} = \psi_{2 1 1}^{(0)}			\\
		\chi_{2 4}^{(0)} = \psi_{2 1 -1}^{(0)}
		\end{align*}
		Thus finally
		
		\begin{figure}
			%%% TODO figure
			\centering
			\includegraphics[width=0.5\linewidth]{Pictures/not-found.jpg}
			\caption{Hamiltonian and splitting of energy states}
		\end{figure}


	
	



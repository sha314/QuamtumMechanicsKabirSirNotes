%%%%%%%%%%%%%%%
% tensor package is used for index management
%
%
%

\chapter{sheet-20 : Time Independent Perturbation}

The major task in any practical application of quantum mechanics is to solve the eigenvalue equation of the Hamiltonian $H$ of the system. Considering the bound states, the eigenvalues of $H$ are discrete and corresponding to each eigenvalue there may be one or several linearly independent eigengectors. The eigenvalue equation, i.e., the time independent Schr\"{o}dinger equation is 
\begin{equation}
	H \ket{E_n} = E_n \ket{E_n}
	\label{chapter20.eqn1-intro}
\end{equation}

Except for few special cases the eigenvalue equation cannot be solved exactly. The eqaution then has to be solved numerically, or approximate methods have to be devised to solve the equation to any desired order of accuracy.

%%% page 2
Time dependent perturbation theory applies when $H$ is of the form
\begin{equation}
	H = H_0 + V
	\label{chapter20.eqn2-intro}
\end{equation}
where the eigenvalues and eigenvectors of $H_0$ are completely known and $V$ is an additional time-independent potential called the perturbation.\\


Let us denote the eigenvalues of $H_0$ as $E_n^{(0)}$ and the corresponding eigenvectors as $\ket{E_n^{(0)}}$  so that the eigenvalue equation for $H_0$ is written as
\begin{equation}
	H_0 \ket{E_n^{(0)}} = E_n^{(0)} \ket{E_n^{(0)}}
	\label{chapter20.eqn3-intro}
\end{equation}

We assume that all the eigenvalues and the eigenvectors of $H_0$ are already calculated.

\section{Non-degenerate Perturbation Theory}
We assume that the eigenvalues $E_n^{(0)}$ of $H_0$ are non-degenerate, i.e., there is only one linearly independent eigenvector $\ket{E_n^{(0)}}$ corresponding to $E_n^{(0)}$. Since the eigenvectors $\ket{E_n^{(0)}}$ form a complete set of vectors, we can express any vector in the Hilbert space as a linear combination of the eigenvectors of $H_0$. Further, eigenvectors belonging to different eigenvalues are orthogonal. We also normalize each of the eigenvectors of $H_0$. So these eigenvectors form a complete orthonormal set, i.e.,
\begin{equation}
	\braket{E_n^{(0)}}{E_m^{(0)}} = \delta_{n m}
	\label{chapter20.eqn4-non-degenerate}
\end{equation}
and
\begin{equation}
	\hat{1} = \sum_{k} \ket{E_k^{(0)}} \bra{E_k^{(0)}}
	\label{chapter20.eqn5-non-degenerate}
\end{equation}
Now we modify the eigenvalue equation for the full Hamiltonian $H$ (equation (\ref{chapter20.eqn1-intro})) as
\begin{equation}
	(H_0 + \lambda V) \ket{E_n}_\lambda = E_{n\lambda} \ket{E_n}_\lambda
	\label{chapter20.eqn6-non-degenerate}
\end{equation}
Where we have introduced a real parameter $\lambda$ whose value lies in the range $(0,1)$. The eigenvalue $E_{n \lambda}$ and the eigenvector $\ket{E_n}_\lambda$  in equation (\ref{chapter20.eqn6-intro}) are not quite the same as the corresponding quantities in equation (\ref{chapter20.eqn1-intro}). It is only in the limit the limit $\lambda\rightarrow 1$ would $E_{n\lambda}$ and  $\ket{E_n}_\lambda$ in equation (\ref{chapter20.eqn6-non-degenerate}) coincide with actual values. Furthermore,
\begin{align}
	\lim\limits_{\lambda \rightarrow 0} E_{n \lambda} &= E_n^{(0)} \\
	\lim\limits_{\lambda \rightarrow 0} \ket{E_n}_\lambda &= \ket{E_{n}^{(0)}}
\end{align}
Thus, as $\lambda \rightarrow 0$, the perturbation is switched off and as $\lambda \rightarrow 1$, the full perturbation $V$ is operative.\\

We will now set up a perturbative scheme for solving $E_{n,\lambda}$ and $\ket{E_n}_\lambda$ and at the end set $\lambda = 1$. First, we write $E_{n,\lambda}$ and $\ket{E_n}_\lambda$ as power series in $\lambda$
\begin{align}
\label{chapter20.eqn7-non-degenerate}
E_{n\lambda} &= E_n^{(0)} + \lambda E_n^{(1)} + \lambda^2 E_n^{(2)} + \ldots \\
\label{chapter20.eqn8-non-degenerate}
\ket{E_n}_\lambda &= \ket{E_n^{(0)}} + \lambda \ket{E_n^{(1)}} + \lambda^2 E_n^{(2)} + \ldots
\end{align}

Substituting equation (\ref{chapter20.eqn7-non-degenerate}) and (\ref{chapter20.eqn8-non-degenerate}) in equation (\ref{chapter20.eqn6-non-degenerate}) we have
\begin{equation}
\qty(H_0 + \lambda V)\qty(\ket{E_n^{(0)}} + \lambda \ket{E_n^{(1)}} + \lambda^2 E_n^{(2)} + \ldots) 
= \qty(E_n^{(0)} + \lambda E_n^{(1)} + \lambda^2 E_n^{(2)} + \ldots)\qty(\ket{E_n^{(0)}} + \lambda \ket{E_n^{(1)}} + \lambda^2 E_n^{(2)} + \ldots)
\end{equation}
or
\begin{align*}
H_0 \ket{E_n^{(0)}} + \lambda \qty(H_0 \ket{E_n^{(1)}}  + V\ket{E_n^{(0)}})  + &\lambda^2(H_0 \ket{E_n^{(2)}} + V\ket{E_n^{(1)}}) + \ldots  \\
= E_n^{(0)} \ket{E_n^{(0)}} + \lambda\qty(E_n^{(1)}\ket{E_n^{(0)}} + E_n^{(0)}\ket{E_n^{(1)}})  
&+ \lambda^2 \qty(E_n^{(2)} \ket{E_n^{(0)}}  + E_n^{(1)}\ket{E_n^{(1)}}  +  E_n^{(0)}\ket{E_n^{(2)}}) + \ldots
\end{align*}
We will solve this equation order by order in $\lambda$. So we equate the coefficient of equal power of $\lambda$ on both sides of the above equation. We  have up to order $\lambda^2$
\begin{align}
\label{chapter20.eqn8b-non-degenerate}
\lambda^0 : \quad &  H_0 \ket{E_{n}^{(0)}} = E_n^{(0)}\ket{E_{n}^{(0)}} \\
\label{chapter20.eqn9-non-degenerate}
\lambda^1 : \quad & H_0\ket{E_{n}^{(1)}} + V\ket{E_{n}^{(0)}} = E_n^{(0)}\ket{E_{n}^{(1)}} + E_n^{(1)}\ket{E_{n}^{(0)}} \\
\label{chapter20.eqn10-non-degenerate}
\lambda^2 : \quad & H_0\ket{E_n^{(2)}} + V\ket{E_{n}^{(1)}} = E_n^{(0)}\ket{E_{n}^{(2)}} + E_{n}^{(1)}\ket{E_{n}^{(1)}}  + E_{n}^{(2)}\ket{E_{n}^{(0)}}
\end{align}
Equation (\ref{chapter20.eqn8b-non-degenerate}) is considered solved because we have assumed that we know fully the eigenvalues and eigenvectors of $H_0$.

\subsection{First-Order correction to energy: $E_n^{(1)}$}
The first order correction to the unperturbed energy of the $n$-th level is $E_n^{(1)}$. This can be found from equation (\ref{chapter20.eqn9-non-degenerate}). We start by taking the product of equation (\ref{chapter20.eqn9-non-degenerate}) with $\bra{E_n^{(0)}}$. We get

\begin{equation}
\bra{E_n^{(0)}} H_0 \ket{E_n^{(1)}} + \bra{E_n^{(0)}} V \ket{E_{n}^{(0)}} = E_n^{(0)} \braket{E_n^{(0)}}{E_n^{(1)}} + E_n^{(1)} \braket{E_n^{(0)}}{E_n^{(0)}}
\label{chapter20.eqn11-non-degenerate}
\end{equation}
Now 
\begin{equation}
H_0 \ket{E_{n}^{(0)}} = E_n^{(0)} \ket{E_{n}^{(0)}}
\end{equation}
Since $H_0$ is hermitian
\begin{equation}
	\bra{E_n^{(0)}} H_0 = E_n^{(0)} = E_n^{(0)} \bra{E_n^{(0)}}
\end{equation}
Also
\begin{equation}
\braket{E_n^{(0)}}{E_n^{(0)}} = 1
\end{equation}
Therefore equation (\ref{chapter20.eqn11-non-degenerate}) becomes
\begin{align}
E_n^{(0)} \braket{E_n^{(0)}}{E_n^{(1)}} + \bra{E_n^{(0)}} V \ket{E_{n}^{(0)}} 
&= E_n^{(0)} \braket{E_n^{(0)}}{E_n^{(1)}} + E_n^{(1)} \nonumber \\
E_n^{(1)} &= \bra{E_n^{(0)}} V \ket{E_{n}^{(0)}} =\equiv V_{n n}
\end{align}
This is a fundamental result of time independent perturbation theory of non-degenerate levels. The first order correction to the $n$-th energy level is the expectation value of the perturbation potential in the unperturbed state.



\subsection{First-Order correction to eigenstate}
The ket $\ket{E_{n}^{(1)}}$ is the first-order correction to the zeroth-order eigenket $\ket{E_{n}^{(0)}}$. The ket $\ket{E_{n}^{(1)}}$ is also found from equation (\ref{chapter20.eqn9-non-degenerate}). First we write $\ket{E_{n}^{(1)}}$ as a linear combination of $\ket{E_{n}^{(0)}}$
\begin{align}
\label{chapter20.eqn13-non-degenerate}
\ket{E_{n}^{(1)}} 
&= \sum_{m} \ket{E_{m}^{(0)}} \braket{E_m^{(0)}}{E_n^{(1)}} \\
\label{chapter20.eqn14-non-degenerate}
&= \sum_{m} \ket{E_{m}^{(0)}} C^(1)_{m n}
\end{align}
Where we have defined
\begin{equation}
C^(1)_{m n} = \braket{E_m^{(0)}}{E_n^{(1)}}
\label{chapter20.eqn15-non-degenerate}
\end{equation}
Using equation (\ref{chapter20.eqn14-non-degenerate}) we write equation (\ref{chapter20.eqn9-non-degenerate}) as
\begin{align*}
	\sum_{m} H_0 \ket{E_{m}^{(0)}} C^{(1)}_{m n} + V \ket{E_{n}^{(0)}} 
	&= \sum_{m} E_n \ket{E_{m}^{(0)}} C^{(1)}_{m n} + E_{n}^{(1)} \ket{E_{n}^{(0)}} \\
	or, \quad 
	\sum_{m} \qty(E_{n}^{(0)}  -  E_{m}^{(0)}) \ket{E_{m}^{(0)}} C^{(1)}_{m n} 
	&= V \ket{E_n^{(0)}} - E_{n}^{(1)} \ket{E_{n}^{(0)}}
\end{align*}
Taking the scalar product with $\bra{E_k^{(0)}}$ we have
\begin{equation}
\qty(E_n^{(0)} - E_k^{(0)}) C^{(1)}_{k n} = \bra{E_k^{(0)}} V \ket{E_n^{(0)}} - E_n^{(1)} \delta_{k n}
\label{chapter20.eqn16-non-degenerate}
\end{equation}
If $k=n$, the left side is zero and we recover the result
\begin{equation}
E_n^{(1)}  = \bra{E_n^{(0)}} V \ket{E_n^{(0)}} , \quad k = n
\end{equation}
Thus, we cannot determine $C^{(1)}_{n n}$ from equation (\ref{chapter20.eqn16-non-degenerate}). This coefficient has to be determined from siderations of normalization of the eigenvectors as discussed later.\\

Next, if $k \neq n$, then equation (\ref{chapter20.eqn16-non-degenerate})ecomes
\begin{equation}
\qty(E_n^{(0)} -  E_{k}^{(0)}) C_{k n}^{(1)} = \bra{E_k^{(0)}} V \ket{E_n^{(0)}} , \quad k \neq n
\end{equation}
or
\begin{equation}
	C_{k n}^{(1)} =  \frac{\bra{E_k^{(0)}} V \ket{E_n^{(0)}}}{E_n^{(0)} -  E_{k}^{(0)}} , \quad k \neq n
	\label{chapter20.eqn17-non-degenerate}
\end{equation}
Using equation (\ref{chapter20.eqn14-non-degenerate}) and (\ref{chapter20.eqn17-non-degenerate}) the first order correction to the state is
\begin{align}
	\ket{E_{n}^{(1)}} 
	&= \sum_{k} \ket{E_k^{(0)}} C^{(1)}_{k n} \\
	\label{chapter20.eqn18-non-degenerate}
	&= C^{(1)}_{n n} \ket{E_{n}^{(0)}}  + \sum_{\substack{k \\ k \neq n}} \frac{\bra{E_k^{(0)}} V \ket{E_n^{(0)}}}{E_n^{(0)} -  E_{k}^{(0)}} \ket{E_k^{(0)}} \\
	\label{chapter20.eqn19-non-degenerate}
	&= C^{(1)}_{n n} \ket{E_{n}^{(0)}}  + \sum_{\substack{k \\ k \neq n}} \frac{V_{k n}}{E_n^{(0)} -  E_{k}^{(0)}} \ket{E_k^{(0)}}
\end{align}
	Where
	\begin{equation}
		V_{k n} \equiv \bra{E_k^{(0)}} V \ket{E_n^{(0)}}
	\end{equation}
	Therefo, upto first order in $\lambda$, the eigenstate $\ket{E_{n}}_\lambda$ is (see equation (\ref{chapter20.eqn8-non-degenerate}))
	\begin{align}
		\ket{E_n}_\lambda 
		&= \ket{E_n^{(0)}} + \lambda \ket{E_n^{(1)}} + \order{\lambda^2} \\
		\label{chapter20.eqn20-non-degenerate}
		&= \ket{E_n^{(0)}} + \lambda C^{(1)}_{n n} \ket{E_n^{(0)}} + \lambda \sum_{\substack{k \\ k \neq n}} \frac{V_{k n}}{E_n^{(0)} -  E_{k}^{(0)}} \ket{E_n^{(0)}} + \order{\lambda^2}
	\end{align}
	We want to normalize $\ket{E_n}_\lambda$ up to first order 
%	\begin{equation}
%		{}_{\lambda}\braket{E_n}{E_n}_{\lambda} = 1 + \order{\lambda^2}
%	\end{equation}
	\begin{equation}
		\tensor*[_{\lambda}]{\braket{E_n}{E_n}}{_{\lambda}} = 1 + \order{\lambda^2}
	\end{equation}
	using equation (\ref{chapter20.eqn20-non-degenerate}), the normalization condition can be written as (noting $\lambda$ is real)
	\begin{align}
		1 + \lambda C^{(1)}_{n n} + \lambda C^{(1)*}_{n n} + \order{\lambda^2}
		&= 1 + \order{\lambda^2} \\
		or, \quad C^{(1)}_{n n} +  C^{(1)*}_{n n} &= 0  \\
	i.e., \quad		\Re{C^{(1)}_{n n}} &= 0
	\end{align}
	Thus $C^{(1)}_{n n}$ is a purely imaginary number. We write
	\begin{equation}
		C^{(1)}_{n n} = \iu \alpha \quad (\alpha \in \real)
	\end{equation}

%%%% page 13
	Hence equation (\ref{chapter20.eqn20-non-degenerate}) can be written as
	\begin{align}
		\ket{E_n}_{\lambda} 
		&= \qty(1 + \iu \lambda \alpha) \ket{E_n^{(0)}} + \lambda \sum_{\substack{k \\ k \neq n}} \frac{V_{k n}}{E_n^{(0)} -  E_{k}^{(0)}} \ket{E_n^{(0)}} + \order{\lambda^2} \\
		&= e^{\iu \lambda \alpha} \ket{E_n^{(0)}} + \lambda \sum_{\substack{k \\ k \neq n}} \frac{V_{k n}}{E_n^{(0)} -  E_{k}^{(0)}} \ket{E_n^{(0)}} + \order{\lambda^2} \\
		e^{-\iu \lambda \alpha} \ket{E_n}_{\lambda}  &= \ket{E_n^{(0)}} + \lambda \sum_{\substack{k \\ k \neq n}} \frac{V_{k n}}{E_n^{(0)} -  E_{k}^{(0)}} \ket{E_n^{(0)}} + \order{\lambda^2}
	\end{align}
	Now $e^{-\iu \lambda \alpha}$ is an overall phase factor which does not affect the normalization of $\ket{E_n}_\lambda$ up to first order. This factor can be set equal to $1$ without loss of generality. So we take $\alpha = 0$, i.e.,
	\begin{equation}
		\braket{E^{(0)}_n}{E^{(1)}_n} \equiv C^{(1)}_{n n} = \iu \alpha = 0
	\end{equation}
	i.e., we can choose $\ket{E_n^{(1)}}$ to be orthonormal to $\ket{E_n^{(0)}}$.\\
	
	Thus, up to first order
	\begin{equation}
		\ket{E_n}_{\lambda}  = \ket{E_n^{(0)}} + \lambda \sum_{\substack{k \\ k \neq n}} \frac{V_{k n}}{E_n^{(0)} -  E_{k}^{(0)}} \ket{E_n^{(0)}} + \order{\lambda^2}
	\end{equation}
	Setting $\lambda = 1$ we get the desired eigenket of  the full Hamiltonian $H$ up to first order in the perturbing potential, i.e.,
	\begin{align}
		\ket{E_n}  
		&= \ket{E_n^{(0)}} + \ket{E_n^{(1)}} \\
		&= \ket{E_n^{(0)}} +  \sum_{\substack{k \\ k \neq n}} \frac{V_{k n}}{E_n^{(0)} -  E_{k}^{(0)}} \ket{E_n^{(0)}}
		\label{chapter20.eqn21-non-degenerate}
	\end{align}
	
	
	\subsection{Second-Order Correction to Energy: $E_n^{(2)}$}
	
	We can find the second order correction to the energy, i.e., $E^{(2)}_n$ from equation (\ref{chapter20.eqn10-non-degenerate}). First, multiply equation (\ref{chapter20.eqn10-non-degenerate}) by $\bra{E_n^{(0)}}$ 
	\begin{equation}
		\bra{E_n^{(0)}} H_0 \ket{E_n^{(2)}} + \bra{E_n^{(0)}} V \ket{E_n^{(1)}} = E_n^{(0)} \braket{E_n^{(0)}}{E_n^{(2)}} + E_n^{(1)} \braket{E_n^{(0)}}{E_n^{(1)}} + e_n^{(2)}
		\label{chapter20.eqn22-non-degenerate}
	\end{equation}
	Since
	\begin{equation}
		\bra{E_n^{(0)}} H_0 = E_n^{(0)} \bra{E_n^{(0)}}
	\end{equation}
	The first term on the left hand side of equation (\ref{chapter20.eqn22-non-degenerate}) cancels the first term on the right. Therefore, we have
	\begin{align}
		\bra{E_n^{(0)}} V \ket{E_n^{(1)}} 
		&=  E_n^{(1)} \braket{E_n^{(0)}}{E_n^{(1)}} + E_n^{(2)}\\
		i.e., \quad E_n^{(2)} =  \bra{E_n^{(0)}} V \ket{E_n^{(1)}}  - E_n^{(1)} \braket{E_n^{(0)}}{E_n^{(1)}}
		\label{chapter20.eqn23-non-degenerate}
	\end{align}
	Writing
	\begin{equation}
		\ket{E_n^{(1)}} = \sum_{m} \ket{E_m^{(0)}} \braket{E_m^{(0)}}{E_n^{(1)}}
	\end{equation}
	we have
	\begin{equation}
		E_n^{(2)} = \sum_{m} \bra{E_n^{(0)}} V \ket{E_m^{(0)}} \braket{E_m^{(0)}}{E_n^{(1)}}  -  E_n^{(1)} \braket{E_m^{(0)}}{E_n^{(1)}}
	\end{equation}
	We now isolate the term with $m=n$ in the summation.
	
	$\therefore$ we have
	\begin{equation}
		E_n^{(2)} = \bra{E_n^{(0)}} V \ket{E_m^{(0)}} \braket{E_m^{(0)}}{E_n^{(1)}} + \sum_{\substack{m\\ m \neq n}} \bra{E_n^{(0)}} V \ket{E_m^{(0)}} \braket{E_m^{(0)}}{E_n^{(1)}}  -  E_n^{(1)} \braket{E_m^{(0)}}{E_n^{(1)}}
		\label{chapter20.eqn24-non-degenerate}
	\end{equation}
	But
	\begin{equation}
		\bra{E_n^{(0)}} V \ket{E_m^{(0)}} = E_n^{(1)}
	\end{equation}
	So the first term cancels the third term in equation (\ref{chapter20.eqn24-non-degenerate}). We then have
	\begin{equation}
		E_n^{(2)} = \sum_{\substack{m\\ m \neq n}} \bra{E_n^{(0)}} V \ket{E_m^{(0)}} \braket{E_m^{(0)}}{E_n^{(1)}}
		\label{chapter20.eqn25-non-degenerate}
	\end{equation}
	Where the 
	%prime on the summation symbol means that the 
	term $m=n$ is excluded from the sum.

	Now, we have found previously (\ref{chapter20.eqn17-non-degenerate}) 
	\begin{equation}
		\braket{E_m^{(0)}}{E_n^{(1)}} \equiv C^{(1)}_{m n} = \frac{\bra{E_m^{(0)}} V \ket{E_n^{(0)}}}{E_n^{(0)} - E_m^{(0)}}
	\end{equation}
	substituting this in equation (\ref{chapter20.eqn25-non-degenerate}) we have
	\begin{equation}
		E_n^{(2)} = \sum_{\substack{m\\ m \neq n}} \frac{\bra{E_n^{(0)}} V \ket{E_m^{(0)}}\bra{E_m^{(0)}} V \ket{E_n^{(0)}}}{E_n^{(0)} - E_m^{(0)}}
		\label{chapter20.eqn26-non-degenerate} 
	\end{equation}
	This is the final expression for the second order correction $E_n^{(2)}$ for the $n-th$ level.
	
	Next, introducing the notation
	\begin{equation}
		V_{n m} \equiv \bra{E_n^{(0)}} V \ket{E_m^{(0)}}
	\end{equation}
	we can write equation (\ref{chapter20.eqn26-non-degenerate}) as 
	\begin{equation}
		E_N^{(2)} = \sum_{\substack{m\\ m \neq n}} \frac{V_{n m} V_{m n}}{E_n^{(0)} - E_m^{(0)}}
	\end{equation}
	Since $V$ is a hermitian operator
	\begin{equation}
		V_{m n} = V_{n m}^*
	\end{equation}
	So,
	\begin{equation}
		E_n^{(2)} = \sum_{\substack{m\\ m \neq n}} \frac{\abs{V_{nm}}^2}{E_n^{(0)} - E_m^{(0)}}
		\label{chapter20.eqn27-non-degenerate} 
	\end{equation}
	Note that the second order correction to the ground state energy is negative. Also, in the second order, the effect of an energy level above the $n$-th level is to push down the energy of the $n$-th level. The effect of a level below the $n$-th level is to push up the energy of the $n$-th level. It is as if, the levels are repelling each other in the $2$nd order perturbation

\section{Degenerate Perturbation Theory}
	%% page 19
	In perturbation theory we seek a solution of the eigenvalue equation of the Hamiltonian $H$, where
	\begin{equation}
		H = H_0 + H^\prime
		\label{chapter20.eqn1-degenerate} 
	\end{equation}
	We assume that the eigenvalues and eigenfunctions of the unperturbed Hamiltonian $H_0$ are known. We then ask how the energy and the wave function of the $n$-th level of the $H_0$ are modified when the perturbation $H^\prime$ is turned on.
	
	Suppose that the $n^{th}$ level of $H_0$ is $g_n$-fold degenerate. Therefore
	\begin{equation}
		H_0 \psi^{(0)}_{n \alpha} = E_n^{(0)} \psi^{(0)}_{n \alpha} \quad ; \alpha = 1,2,\ldots, g_n
		\label{chapter20.eqn2-degenerate} 
	\end{equation}
	The $g_n$ wave functions $\{\psi^{(0)}_{n \alpha}\ ; \alpha=1,2,\ldots, g_n\}$ are the linearly independent of each other and they are all orthogonal to the unperturbed wave functions belonging to other energy levels.
	
	% page 20
	We note that any linear combination of the vectors $\qty{\psi^{(0)}_{n \alpha}; \ \alpha=1,\ldots,g_n}$ is also an eigenvector of $H_0$ with the same eigenvalue $E_n^{(0)}$. Thus if we construct a vector $\chi^{(0)}_{n \beta}$ as
	\begin{equation}
		\chi^{(0)}_{n \beta} = \sum_{\alpha = 1}^{g_n} C_{\alpha \beta} \psi^{(0)}_{n\alpha}
		\label{chapter20.eqn3-degenerate} 
	\end{equation}
	
	Then $\chi^{(0)}_{n \beta}$ is also an eigenvector of $H_0$ with eigenvalue $E_n^{(0)}$:
	\begin{equation}
		H_0 \chi^{(0)}_{n \beta} = E_n^{(0)} \chi^{(0)}_{n \beta}
		\label{chapter20.eqn4-degenerate} 
	\end{equation}
	Now the vectors $\qty{\psi^{(0)}_{n \alpha}; \ \alpha=1,\ldots,g_n}$ need not be orthogonal. However, by using the Schmidt procedure, we can make the degenerate eigenvectors orthonormal by taking suitable linear combinations if they are not orthogonal to start with. This procedure can be applied to all vectors belonging to every level.
	
	% page 20a
	Thus, we will assume that all vectors whether belonging to the same leel or not are normalized and orthogonal to each other, i.e.,
	\begin{equation}
		\braket{\psi^{(0)}_{n\alpha}}{\psi^{(0)}_{m\beta}} = \delta_{n m}\delta_{\alpha \beta}
		\label{chapter20.eqn5-degenerate} 
	\end{equation}
	Further, the eigenvectors of $H_0$ space the entire Hilbert space, i.e., they form a complete set of states. The completeness condition can be written as
	\begin{equation}
		\hat{1} = \sum_{k} \sum_{\alpha = 1}^{g_k} \ket{\psi^{(0)}_{k\alpha}} \bra{\psi^{(0)}_{k\alpha}}
		\label{chapter20.eqn6-degenerate} 
	\end{equation}
	The "full" eigenvalue equation for the $n^{th}$ level is written as
	\begin{equation}
		H \psi_{n \alpha} = E_{n \alpha} \psi_{n \alpha} \quad ; \quad \alpha = 1,2,\ldots,g_n
		\label{chapter20.eqn7-degenerate} 
	\end{equation}
	In order to facilitate counting of different orders, we may write
	\begin{equation}
		H = H_0 + \lambda H^\prime
	\end{equation}
	Where $\lambda$ is a real parameter which we set equal to one at the end of our calculations.
	% page 20b
	The eigenvalues $E_{n \alpha}$ and the eigenvector $\psi_{n\alpha}$ are now functions of $\lambda$. In the limit $\lambda \rightarrow 0$ $E_{n\alpha}$ tends to $E_n^{(0)}$, i.e.,
	\begin{equation}
		\lim\limits_{\lambda \rightarrow 0} E_{n\alpha} = E_n^{(0)}
		\label{chapter20.eqn8-degenerate} 
	\end{equation}
	However, there is a difficulty in taking the corresponding limits for $\psi_{n\alpha}$. Since there are $g_n$ linearly independent unperturbed eigenfunctions corresponding to $E_n^{(0)}$, we do not know to which particular eigenfunction will $\psi_{n\alpha}$ tend to when $\lambda \rightarrow 0$. Suppose
	
	\begin{equation}
		\psi_{n\alpha} \quad \substack{\rightarrow \\ \lambda=0} \quad \chi_{n\alpha}^{(0)}
	\end{equation}
	Where $\chi_{n\alpha}^{(0)}$ is some linear combination of $\qty{\psi_{n \alpha}^{(0)}, \alpha=1,2,\ldots,g_n}$.
	
	Now we write
	\begin{equation}
		\psi_{n \alpha} = \chi_{n\alpha}^{(0)} + \lambda \psi_{n \alpha}^{(1)} + \lambda^2 \psi_{n \alpha}^{(2)}
		\label{chapter20.eqn9-degenerate} 
	\end{equation}
	where $\chi_{n\alpha}^{(0)}$ is as yet some undetermined linear combination of $\qty{\psi_{n \alpha}^{(0)}, \alpha=1,2,\ldots,g_n}$. We also write the perturbed energy $E_{n\alpha}$ as 
	\begin{equation}
		E_{n\alpha} = E_n^{(0)} + \lambda E_{n\alpha}^{(1)} + \lambda^2 E_{n\alpha}^{(2)}
		\label{chapter20.eqn10-degenerate}
	\end{equation}
	where we have used the fact that $E_{n\alpha}^{(0)} = E_n^{(0)}$ for all $\alpha$.\\
	
	Next, we substitute (\ref{chapter20.eqn9-degenerate}) and (\ref{chapter20.eqn10-degenerate}) in equation (\ref{chapter20.eqn7-degenerate}). We have

	\begin{equation}
		\qty(H_0 + \lambda H^\prime) \qty(\chi_{n\alpha}^{(0)} + \lambda \psi_{n \alpha}^{(1)} + \ldots) 
		= \qty(E_n^{(0)} + \lambda E_{n\alpha}^{(1)} + \ldots) \qty(\chi_{n\alpha}^{(0)} + \lambda \psi_{n \alpha}^{(1)} + \ldots) 
	\end{equation}
	Equating the coefficient of equal powers of $\lambda$ on both sides of this equation we obtain the zeroth order equation
	\begin{equation}
		H_0 \chi_{n\alpha}^{(0)} = E_n^{(0)} \chi_{n\alpha}^{(0)}
	\end{equation}
	which is equation (\ref{chapter20.eqn4-degenerate}) written earlier. In the first order we have
	
	%%	page 22
	
	
	\begin{equation}
		\qty(H_0 - E_n^{(0)}) \psi_{n \alpha}^{(1)} = \qty(E_{n\alpha}^{(1)} - H^\prime) \chi_{n\alpha}^{(0)}
		\label{chapter20.eqn11-degenerate}
	\end{equation}
	
	Now, we write
	\begin{equation}
		\psi_{n \alpha}^{(1)} = \sum_{k\beta} C_{k\beta, n\alpha} \psi_{k \beta}^{(0)}
		\label{chapter20.eqn12-degenerate}
	\end{equation}
	and
	\begin{equation}
		\chi_{n\alpha}^{(0)} = \sum_{\beta=1}^{g_n} a_{\beta\alpha} \psi_{n\beta}^{(0)}
		\label{chapter20.eqn13-degenerate}
	\end{equation}
	
	where the indices $\alpha$ and $\beta$ refer explicitly to degeneracy. Substituting (\ref{chapter20.eqn12-degenerate}) and(\ref{chapter20.eqn13-degenerate}) in equation (\ref{chapter20.eqn11-degenerate}) we find
	\begin{align}
		\qty(H_0 - E_n^{(0)}) \sum_{k,\beta} C_{k\beta, n\alpha} \psi_{k \beta}^{(0)} 
		&= \qty(E_{n\alpha}^{(1)} - H^\prime) \sum_{\beta=1}^{g_n} a_{\beta\alpha} \psi_{n\beta}^{(0)} \\
		or,\ \sum_{k,\beta} C_{k\beta, n\alpha} \qty(E_k^{(0)} - E_n^{(0)}) \psi_{k \beta}^{(0)} 
		&= \sum_{\beta=1}^{g_n} a_{\beta\alpha} \qty(E_{n\alpha}^{(1)} - H^\prime) \psi_{n\beta}^{(0)} 
	\end{align}
	Taking the scalar product with $\psi_{m\gamma}^{(0)}$ and using orthogonality $\braket{\psi_{m\gamma}^{(0)}}{\psi_{k\beta}^{(0)}} = \delta_{m k} \delta_{\gamma \beta}$, we have
	
	\begin{equation}
		C_{m\gamma, n\alpha} \qty(E_m^{(0)} - E_n^{(0)}) = \sum_{\beta=1}^{g_n} a_{\beta\alpha} \qty(E_{n\alpha}^{(1)}\delta_{mn}\delta_{\gamma\beta}  -  H^\prime_{m\gamma,n\beta})
		\label{chapter20.eqn14-degenerate}
	\end{equation}
	where we have written
	\begin{equation}
		H^\prime_{m\gamma,n\beta} = \bra{\psi_{m\gamma}^{(0)}} H^\prime \ket{\psi_{n\beta}^{(0)}}
		\label{chapter20.eqn15-degenerate}
	\end{equation}
	
	\subsubsection{First Order Correction to Energy}
	%% page 25 (pdf) or 23(hand written)
	
	First, let us choose $m=n$ in equation (\ref{chapter20.eqn14-degenerate}). Then the left hand side of this equation is zero. We then have
	\begin{equation}
		\sum_{\beta=1}^{g_n} \qty(H^\prime_{n\gamma,n \beta}  -  E_{n\alpha}^{(1)}\delta_{\gamma \beta}) a_{\beta \alpha} = 0
		\label{chapter20.eqn16-degenerate}
	\end{equation}
	Simplifying the notation by writing
	\begin{equation}
		H^\prime_{n\gamma, n\beta} = H^{\prime (n)}_{\gamma\beta}
	\end{equation}
	We write equation (\ref{chapter20.eqn16-degenerate}) as
	% page 26
	
	\begin{equation}
		\sum_{\beta=1}^{g_n} \qty(H^{\prime (n)}_{\gamma \beta}  -  E_{n\alpha}^{(1)}\delta_{\gamma \beta}) a_{\beta \alpha} = 0
		\label{chapter20.eqn17-degenerate}
	\end{equation}
	Equation (\ref{chapter20.eqn16-degenerate}) is a set of $g_n$ linear equations for unknowns $\qty{a_{1\alpha}, a_{2\alpha}, \ldots, a_{g_n \alpha}}$ corresponding to $E_{n\alpha}^{(1)}$. The value of $E_{n\alpha}^{(1)}$ are not known a priori.\\
		
		However, we note that, for a solution of equation (\ref{chapter20.eqn17-degenerate}) to exist, the determinant formed by the coefficient of $a_{\beta\alpha}$ must vanish, i.e.,
		\begin{equation}
			\det \qty[H^{\prime (n)}_{\gamma \beta}  -  E^{(1)}_{n\alpha} \delta_{\gamma \beta}] = 0
			\label{chapter20.eqn18-degenerate}
		\end{equation}
		This is called the secular equation, which is a polynomial of degree $g_n$ in $E_{n\alpha}^{(1)}$. It has $g_n$ real roots $E_{n 1}^{(1)}, E_{n 2}^{(1)}, \ldots, E_{n, g_n}^{(1)}$. If all these roots are distinct, the degeneracy is completely removed to first order in the perturbation. On the other hand, if some or all roots of equation (\ref{chapter20.eqn18-degenerate}) are identical, the degeneracy is only partially (or not at all) removed. The residual degeneracy may then either be removed in higher order perturbation theory, or it may persist in all orders.\\
		
		Next, substituting each of the roots $E_{n\alpha}^{(1)}, \ \alpha=1,2,\ldots,g_n$ in equation (\ref{chapter20.eqn17-degenerate}) we can solve for coefficients $a_{1\alpha}, a_{2 \alpha}, \ldots, a_{g_n \alpha}$. In fact, one of the coefficients remain undetermined and the other coefficients are found in terms of the undetermined one. This is because the set of equation given by equation (\ref{chapter20.eqn17-degenerate}) are homogeneous. The undetermined coefficient is then obtained up to a phase by requiring that the eigenvector $a_{\beta\alpha}; \beta = 1,2,\ldots, g_n$ be normalized to unity.
		\begin{equation}
			a^{*}_{1\alpha} a{1\alpha}  +  a^{*}_{2\alpha} a{2\alpha} + \ldots + a^{*}_{g_n\alpha} a{g_n\alpha} = 1
		\end{equation}
		that is,
		\begin{equation}
			\sum_{\beta = 1}^{g_n}	a^{*}_{2\alpha} a{2\alpha} = 1 \quad ; \alpha=1,2,\ldots, g_n
		\end{equation}
		The correct zeroth order wave function is then found using equation (\ref{chapter20.eqn13-degenerate}), i.e.,
		\begin{equation}
			\chi_{n\alpha}^{(0)} = \sum_{\beta = 1}^{g_n} a_{\beta\alpha} \psi_{n\beta}^{(0)}
		\end{equation}
		The functions $\chi_{n\alpha}^{(0)}$ are eigenvectors of $H^\prime$ in the eigen subspace of $E_n^{(0)}$ with eigenvalue $E_{n\alpha}^{(1)}$, i.e.,
		\begin{equation}
			H^\prime \chi_{n\alpha}^{(0)}= E_{n\alpha}^{(1)} \chi_{n\alpha}^{(0)} \quad ; \alpha = 1,2,\ldots, g_n
		\end{equation}
		and the coefficients $a_{\beta \alpha}\ , \beta=1,2, \ldots, g_n$ form the $g_n$ component representation of the eigenvector $\chi_{n\alpha}^{(0)}$ using the basis $\qty{\psi_{n\beta}^{(0)}\ , \beta =1,2, \ldots,g_n}$.
		
		Thus
		\begin{align}
			\chi_{n\alpha}^{(0)} &= \mqty[a_{1\alpha} \\ a_{2\alpha} \\ \vdots \\ a_{g_n\alpha}] 
		\end{align}
		
		Thus, in summary, the first-order corrections to the $n^{th}$ degenerate level of $H_0$ with energy $E_n^{(0)}$ are obtained by diagonalizing $H^\prime$ in the eigen subspace of $E_n^{(0)}$. The eigenvalues of $H^\prime$ are the corrections to the energy and the corresponding eigenvectors of $H^\prime$ are the zeroth order approximation of the wavefunction.\\
		
		Once the correct zeroth-order wavefunctions $\chi_{n\alpha}^{(0)}\ , \alpha=1,2, \ldots, g_n$, have been determined, the first order correction $\psi_{n \alpha}^{(1)}$ to the wavefunction and second-order energy correction $E_{n\alpha}^{(2)}$ can be obtained in a way similar to non-degenerate perturbation theory.
		
		
		
		%% page 31 Ex
		\section{Examples}
		\begin{enumerate}
			\item 
			Calculate the first order energy shifts for the first three states of the infinite square well of width $a$ in one dimension due to the perturbation $V(x) = V_0 \frac{x}{a}$.\\
			
			\underline{Ans}\\
			
			
			
			\item 
			This is new
			
			
			
			
		\end{enumerate}
	
	


